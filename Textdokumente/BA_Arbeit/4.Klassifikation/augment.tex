Hier wird auf die Erstellung der Augmented Vectors der Direct Cascade Netzwerke für die Klassifikation eingegangen. Jedes der drei Netze hat 
eine eigene Berechnung davon. Es wird zuerst das COD-, dann das 1DC- und zum Schluss das 2DC-Netzwerk betrachtet. 

Bei allen Netzwerken wird der Input des Netzwerkes und die Prediction gebraucht, um diesen Vector zu erstellen. Der Input ist entweder der 
Datensatz selbst oder der vorherige Augmented Vector. Nur beim ersten Mal ist es der Datensatz, da danach der Augmented Vector existiert. 
Der Augmented Vector wächst dabei bei jeder Iteration von einem neuen Netzwerk an, da darüber das Wissen aller vorherigen auf das neue 
übertragen wird. 
Die Prediction ist das Ergebnis, welches aus der Inferenz des fertig trainierten Netzwerkes kommt. Im folgenden bedeuten die Buchstaben N, W, 
H und C Datensamples, Bildbreite, Bildhöhe und Channel. 

Das COD-Netzwerk hat als Input den Datensatz mit folgendem Shape: (N, W*H). 
Die Prediction hat immer den Shape (N, 10). 
Diese beiden Sachen werden in der zweiten Dimension konkateniert. Dies ergibt die Formel 4.1 und damit den Augmented Vector. 

\begin{equation}
    AugVec(Input(N, W*H), Prediction(N, 10)) = (N, (W*H).10)
\end{equation}

Das 1DC-Netzwerk hat als Input hingegen den Datensatz in folgendem Shape: (N, W*H, C). Da der Channel nur eindimensional ist, wird dieser 
zuerst entfernt und dann die Berechnung nach der Formel 4.1 durchgeführt. Zum Schluss wird die Channeldimension wieder hinzugefügt. 
Beide bisher behandelten Netzwerke haben somit einen um N*10 Einträge linear wachsenden Augmented Vector. 

Das 2DC-Netzwerk hat einen komplexeren Input mit dem Shape: (N, W, H, C). Dies muss verbunden werden mit der Prediction die in der Form (N, 10) 
vorliegt. Dafür wird für jedes N zehn Arrays gebaut, die alle die Form (W, H, C) haben. Diese haben von eins bis zehn den Inhalt der Prediction. 
Dies wird dann auf der Channeldimension konkateniert. Daraus resultiert die Formel 4.2. 

\begin{multline}
    AugVec(Input(N, W, H, C), Prediction(N, 10)) = Input(N, W, H, C.ConVec)\\
    ConVec(W, H, C)[0-9] = Prediction(10)[0-9]
\end{multline}

Dabei ist der ConVec der Vector in dem die Predictionwerte von eins bis zehn jeweils in der Form (W, H, C) enthalten sind. Dies skaliert bei 
den Netziterationen im Speicherplatz aber in der Form, wie es in Formel 4.3 zu sehen ist.  

\begin{equation}
    AugVecNew = N*W*H*C_{old} + N*W*H*10
\end{equation}

Daraus ergibt sich, dass der Arbeitsspeicherplatzbedarf mit einer Steigung von dem Zehnfachen des Datensatzes zunimmt. Bei Daten, die in einem Sample 
bereits 8192 Bytes benötigen, ist klar, dass diese Variante des Bauens des Augmented Vectors nicht durchgeführt werden sollte, da dieser zu 
schnell zu groß wird. Aus diesem Grund wird das 2DC-Netzwerk im Folgenden nicht mehr verwendet. 
