Hier werden kurz die Direct Cascade Netze für die Klassifikation vorgestellt. Die Besonderheit eines solchen Netzwerkes ist es, dass es 
immer nur ein einziges Hiddenlayer existiert und die Netze so iteriert werden, dass sie das Wissen der vorherigen mitnehmen. 

\begin{table}[h!]
    \caption{Direct Cascade Networks}
        \label{tab:classvor}
    \begin{tabular}{l|l|l|l|l}
        \textbf{Name} & \textbf{Hiddenlayer} & \textbf{Nodes/Filter} & \textbf{Aktivierung} & \textbf{Input} \\
        \hline
        ClassOneDense & Linear & 512 & Relu & 1 \\
        1DConv & 1DConv & 32 & Relu & 1 \\
        2DConv & 2DConv & 32 & Relu & 2
    \end{tabular}
\end{table}

Dabei ist der Input die Dimension in der die Bilddaten vorliegen und nur in der ersten Zeile sind es Nodes, sonst sind es die Filter. 
Bei beiden Convolution-Netzen wird ein Kern der Größe 3 beziehungsweise 3x3 verwendet mit einem solchen Padding, dass die Größe der Daten 
dabei nicht verändert wird. 

\iffalse
Das ClassOneDense-Netzwerk besitzt ein Linearlayer mit 512 Nodes als Hiddenlayer 
und der Aktivierungsfunktion Relu. Es nimmt als Input eindimensionale Bilddaten. 

Das 1DConv-Netzwerk besitzt ein 1DConv-Layer als Hiddenlayer mit 32 Filtern, einer Kerngröße von 3 und solchem Padding, dass sich die Größe 
der Daten nicht verändert. Zudem wird die Aktivierungsfunktion Relu genutzt und eindimensionale Bilddaten. 

Das 2DConv-Netzwerk besitzt ein 2DConv-Layer als Hiddenlayer mit 32 Filtern, einer Kerngröße von 3x3 und einem Padding, sodass die Datengröße 
nicht verändert wird. Es wird als Aktivierungsfunktion Relu genutzt und als Input zweidimensionale Bilddaten. 
\fi
