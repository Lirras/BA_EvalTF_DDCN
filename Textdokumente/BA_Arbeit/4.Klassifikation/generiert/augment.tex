In diesem Abschnitt wird die Erstellung der Augmented Vectors für die Direct Cascade Netzwerke im Klassifikationskontext beschrieben. Jedes der 
drei betrachteten Netzwerke – COD, 1DC und 2DC – verwendet dabei eine eigene Methode zur Generierung dieser Vektoren. Die Darstellung erfolgt in 
der genannten Reihenfolge: zunächst das COD-, anschließend das 1DC- und abschließend das 2DC-Netzwerk.

Allen Netzwerken gemeinsam ist, dass sowohl der Input des Netzwerks als auch dessen Vorhersage (Prediction) zur Berechnung des Augmented Vectors 
herangezogen werden. Als Input dient entweder der ursprüngliche Datensatz oder – ab der zweiten Iteration – der jeweils zuletzt erzeugte Augmented 
Vector. Nur bei der ersten Iteration wird der originale Datensatz verwendet, da zu diesem Zeitpunkt noch kein Augmented Vector existiert.

Mit jeder weiteren Iteration wächst der Augmented Vector, da er Informationen aus allen vorhergehenden Netzwerken integriert und so das Wissen 
schrittweise an das nächste Netzwerk weitergibt. Die Prediction entspricht dabei dem Ausgabeverhalten des trainierten Netzwerks auf Basis der 
jeweils aktuellen Eingabe.

Im Folgenden bezeichnen die Variablen N, W, H und C die Anzahl der Datensamples, die Bildbreite, die Bildhöhe sowie die Anzahl der Kanäle 
(Channels).

Für das COD-Netzwerk gilt: Der Input liegt in der Form (N,WxH) vor, d.h., die Bilddaten wurden vorab flach in ein Vektorformat 
überführt. Die Prediction hat die Form (N,10), entsprechend der zehn Klassenzugehörigkeiten. Beide Matrizen werden entlang der zweiten 
Dimension (Feature-Dimension) miteinander konkateniert, was zur Definition des Augmented Vectors gemäß Gleichung 4.1 führt.

\begin{equation}
    AugVec(Input(N, W*H), Prediction(N, 10)) = (N, (W*H).10)
\end{equation}

Beim 1DC-Netzwerk liegt der Input in der Form (N,WxH,C) vor. Da der Kanal in diesem Fall eindimensional ist, wird die Channel-Dimension 
zunächst entfernt. Anschließend erfolgt die Berechnung des Augmented Vectors gemäß Gleichung 4.1. Nach der Berechnung wird die Channel-Dimension 
wieder ergänzt, um die ursprüngliche Struktur beizubehalten. In beiden bisher betrachteten Netzwerken (COD und 1DC) wächst der Augmented Vector 
mit jeder Iteration linear um $N * 10$ Einträge.

Das 2DC-Netzwerk hingegen arbeitet mit einem komplexeren Eingabeformat: (N,W,H,C). Die Prediction liegt auch hier in der 
Form (N,10) vor. Zur Fusion von Input und Prediction werden für jedes Sample NN zehn zusätzliche Arrays erzeugt, jeweils in der 
Form (W,H,C). Diese Arrays enthalten jeweils die Werte der zehn Klassenwahrscheinlichkeiten der Prediction, wobei jeder Kanal mit 
dem entsprechenden Skalarwert befüllt wird. Anschließend erfolgt eine Konkatenation entlang der Channel-Dimension. Dieses Verfahren führt zur 
Definition des Augmented Vectors gemäß Gleichung 4.2.

\begin{multline}
    AugVec(Input(N, W, H, C), Prediction(N, 10)) = Input(N, W, H, C.ConVec)\\
    ConVec(W, H, C)[0-9] = Prediction(10)[0-9]
\end{multline}

Der sogenannte ConVec bezeichnet den Vektor, in dem die einzelnen Werte der Prediction – entsprechend den Klassen eins bis zehn – jeweils in 
Form eines Arrays der Dimension (W,H,C) abgelegt sind. Bei jeder Iteration des Netzwerks führt dieses Verfahren jedoch zu einem 
speicherintensiven Wachstum des Augmented Vectors. Die Skalierung des Speicherbedarfs erfolgt gemäß der in Gleichung 4.3 dargestellten Beziehung.

\begin{equation}
    AugVecNew = N*W*H*C_{old} + N*W*H*10
\end{equation}

Daraus ergibt sich, dass der Arbeitsspeicherbedarf bei jeder Iteration mit einer Steigerungsrate wächst, die dem Zehnfachen der Größe des 
jeweiligen Datensatzes entspricht. Bei Datensamples, die bereits im Ausgangszustand eine Größe von 8192 Bytes pro Datenpunkt aufweisen, führt 
dieses Vorgehen zu einem sehr stark linear ansteigenden Speicherverbrauch. Daher ist die hier beschriebene Methode zur Erstellung des Augmented Vectors aus 
praktischer Sicht nicht sinnvoll einsetzbar. Aus diesem Grund wird das 2DC-Netzwerk im weiteren Verlauf der Arbeit nicht mehr berücksichtigt.
