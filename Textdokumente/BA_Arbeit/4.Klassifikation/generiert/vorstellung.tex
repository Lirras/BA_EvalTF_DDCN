In diesem Kapitel werden die Eigenschaften und Erkenntnisse im Bereich der Klassifikation detailliert untersucht. Zunächst werden die Direct 
Cascade Netzwerke vorgestellt und gemeinsam mit dem Deep Cascade Netzwerk analysiert. Dabei wird untersucht, wie sich die Menge der verfügbaren 
Target-Daten auf das Trainingsergebnis auswirkt. Zudem wird erläutert, weshalb das 2DC-Netzwerk trotz potenziell besserer Ergebnisse nicht weiter 
betrachtet wird. Ein weiterer Schwerpunkt liegt auf der detaillierten Analyse der Augmented Vectors bei den Direct Cascade Netzwerken. 
Abschließend werden die Unterschiede und Auswirkungen beim Wechsel zwischen Cascade TF, Cascade und Complete Netzwerken diskutiert.

Die Direct Cascade Netzwerke für die Klassifikation zeichnen sich dadurch aus, dass jeweils nur ein einzelnes Hidden Layer verwendet wird und die 
Netzwerke iterativ aufgebaut sind, wobei das Wissen der vorherigen Iterationen übernommen wird. 

\begin{table}[h!]
    \centering    
    \begin{tabular}{l|l|l|l|l}
        \textbf{Name} & \textbf{Hiddenlayer} & \textbf{N/F} & \textbf{Aktivierung} & \textbf{Inputdim} \\
        \hline
        COD & Linear & 512 & Relu & 1 \\
        1DC & 1DConv & 32 & Relu & 1 \\
        2DC & 2DConv & 32 & Relu & 2
    \end{tabular}
    \caption{\small{In dieser Übersicht sind alle Direct Cascade Netzwerke für die Klassifikation mit den wesentlichen internen Parametern 
    dargestellt. "Hidden Layer" bezeichnet die Art des verwendeten versteckten Layers. "N/F" gibt die Anzahl der Neuronen (Nodes) beziehungsweise 
    Filter im jeweiligen Layer an. Unter "Aktivierung" ist die im Hidden Layer eingesetzte Aktivierungsfunktion zu verstehen, und "Inputdim" 
    beschreibt die Dimensionalität der Eingabedaten.}}
        \label{tab:classvor}
\end{table}

In Tabelle \ref{tab:classvor} sind sämtliche Direct Cascade Klassifikationsnetzwerke aufgeführt. Diese umfassen die Modelle 
Classification\_one\_Dense (COD), OneDLilConv (1DC) sowie little\_conv (2DC).

Der Input entspricht der Dimensionalität, in der die Bilddaten vorliegen. Lediglich in der ersten Zeile wird die Anzahl der Neuronen (Nodes) 
angegeben, während in den übrigen Zeilen die Anzahl der Filter angegeben ist. Bei beiden Convolutional-Netzwerken wird ein Kernel der Größe 3 
bzw. 3x3 verwendet, wobei durch entsprechendes Padding die räumlichen Dimensionen der Daten erhalten bleiben. Für das Training wird in diesen 
Netzwerken eine Batch-Größe von jeweils 128 Datensamples verwendet.
