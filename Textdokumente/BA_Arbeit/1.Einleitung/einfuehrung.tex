% Wo sind wir gerade?
% Cascade Verfahren entdeckt + festgestellt, dass die recht gut sein könnten.
% Deep Cascade entdeckt + Constructive Networks
% TF angefangen

Selbst diejenigen, die nichts mit Informatik zu tun haben, kennen heute KI. 
Dahinter stecken neuronale Netze. Diese werden meistens in einem Zug komplett gebaut und trainiert. 
Das dauert lange, sodass das Kaskadierungsverfahren der Cascade Correlation \cite{cascor} entwickelt wurde. 
Dabei wurde bemerkt, dass es in Ordnung funktioniert und das schon bei kleineren Netzen. Deshalb wurden darauf aufbauend 
weitere Netzwerke und Kaskadierungsverfahren entwickelt \cite{cascade_network_architectures}, \cite{Constructive_Cascade}, 
\cite{deep_cascade_learning}. Es wurden bei den Kaskadennetzwerken ebenso festgestellt, dass es mit diesem Aufbau der Netze 
einfach ist, einen Wechsel zwischen unterschiedlichen Daten und Aufgaben durchzuführen \cite{phd_deep_cascade}, \cite{transfer_learning}, 
\cite{survey_transfer}. In dem Bereich des Wechsels zwischen unterschiedlichen Daten ist diese Arbeit anzusiedeln, denn sie überprüft, wie 
gut die einzelnen Netzwerkvarianten sind und geht auf die Probleme von TF und Cascade ein. 

% Was fehlt denn noch in der Einführung? 

% Hier kann doch nur eine hinführung zum Thema sein. 

% oder eine Zusammenfassung dessen, was ich tat und was hier vorkommt. 

% Hinführung + Grob, was man tat.
