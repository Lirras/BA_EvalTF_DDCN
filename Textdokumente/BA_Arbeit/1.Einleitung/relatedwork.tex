Diese Arbeit baut auf den CasCor-Algorithmus\cite{cascor}, der wegen der langen Trainingsdauer entwickelt wurde, auf und nutzt diesen, 
um Direct Cascade \cite{cascade_network_architectures} 
durchzuführen, wie es in der Thesis von Marquez \cite{phd_deep_cascade} verwendet wurde. Zudem wird Deep Cascade verwendet als 
Vergleichsbasis \cite{deep_cascade_learning}. Dabei sind Deep Cascade Netzwerke solche, die iterativ aufgebaut werden \cite{Constructive_Cascade}. 
Es wird nur Domain-Wechsel angewandt, während es noch den Taskwechsel \cite{transfer_learning} gibt. Zudem gibt es bei Transfer Learning drei 
Probleme \cite{survey_transfer}, die hier auch bearbeitet werden. 

% Alle wissenschaftlichen Arbeiten, die hier dazu gehören und verwandt sind aufzählen. 

% Also die Direct Cascade Arbeiten von Ritter und Littmann zum Beispiel. 

% Ebenso Cascor von Fahlmann und Lebiere.

% Deep Cascade Learning von Enrico S. Marquez. 

% Wie sieht das mit dem Bachelorvortrag aus? Wieviel später kommt der?
% Passt das hier von der Länge und dem Inhalt? 

% Aus Zorn und Hass bestehend, errichtete er großes Unheil, indem er ging anstatt zu kämpfen und dadurch eventuell bleiben zu können. 
% Er ging, weil es das kleinere Übel für die Gruppe war, denn er hatte sie dennoch lieb. Aber seine eigene Heimat war zu einem Ort von 
% Leid geworden und er wusste, warum er nichts dagegen unternehmen konnte. Es waren Freunde, die ihn angegriffen und über ihn gepottet hatten. 
% Die Freunde und die Heimat verloren jede freundschaftliche Bedeutung. Dies war der wirkliche Schmerz. Er wollte nicht noch einmal hin, doch er 
% wusste, dass er es nochmal tun musste und es weitergehen würde. Die Zerstörung seines Selbst. Und sie wissen nicht, was sie taten, was sie 
% wirklich taten. Wie zerstörerisch sie waren, ist ihnen immer noch unbekannt. Wie kann er sie noch lieben, wenn sie ihn verachteten und seine 
% auch kommunizierten Grenzen konsequent überschritten? Wenn seine Worte für sie keine Bedeutung haben? 

% Gott, war das bei dir nicht genauso als du unter den Menschen wandeltest? Als du von allen abgelehnt wurdest, waren deine Worte und 
% augezeigten Grenzen für deine Nächsten wichtig? Haben sie sich daran gehalten oder gab es so etwas nicht? 
% Wurdest du so ignoriert? Konntest du aus dem Gefängnis der Angst ausbrechen? Kennst du das Problem überhaupt? Ich kann mich nicht daran 
% erinnern diesbezüglich irgendetwas in der Bibel gelesen zu haben. Wie soll man denn den anderen sagen, dass es reicht, dass es zuviel ist, 
% wenn man vor Angst zu unfähig ist, gegen sie zu halten und zu argumentieren? Wie kann es sein, dass niemand irgendeinen Punkt findet, über 
% den sie mit mir reden können? Liegt es daran, dass ich zu versteckt bin, dass ich nichts preisgeben kann? 
% Warum muss ich selbst die Gespräche anfangen und dann interessiert es die meisten nicht einmal das Thema um das es geht? 
% Wieso muss ich zuerst mein Problem besiegen, um es bekämpfen zu können? 
% Wenn ich selbst nicht andere anspreche, dann werde ich in Ruhe gelassen und komme in eine gefährliche Isolation, dessen Ende grausam ist. 
% Gott, Jesus - Ich kann nicht mehr! Ich will nicht, dass es so ist, aber aus eigener Kraft kann ich es nicht ändern. 
% Bitte, Hilf mir! - Wieso klingt das so egoistisch? Hilf meinem Umfeld auch, sowie allen Menschen. Jeder hat seinen Rucksack zu tragen. 
% Jede braucht deine Hilfe bei den unterschiedlichsten Problemen. 
