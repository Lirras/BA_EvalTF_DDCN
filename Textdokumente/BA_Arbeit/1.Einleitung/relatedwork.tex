Diese Arbeit baut auf den CasCor-Algorithmus \cite{cascor}, der wegen der langen Trainingsdauer entwickelt wurde, auf und nutzt diesen, 
um Direct Cascade \cite{cascade_network_architectures} 
durchzuführen, wie es in der Thesis von Marquez \cite{phd_deep_cascade} verwendet wurde. Zudem wird Deep Cascade verwendet als 
Vergleichsbasis \cite{deep_cascade_learning}. Dabei sind Deep Cascade Netzwerke solche, die iterativ aufgebaut werden \cite{Constructive_Cascade}. 
Es wird nur Domain-Wechsel angewandt, während es noch den Taskwechsel \cite{transfer_learning} gibt. Zudem gibt es bei Transfer Learning drei 
Probleme \cite{survey_transfer}, die hier auch bearbeitet werden. 

% Alle wissenschaftlichen Arbeiten, die hier dazu gehören und verwandt sind aufzählen. 

% Also die Direct Cascade Arbeiten von Ritter und Littmann zum Beispiel. 

% Ebenso Cascor von Fahlmann und Lebiere.

% Deep Cascade Learning von Enrico S. Marquez. 

% Wie sieht das mit dem Bachelorvortrag aus? Wieviel später kommt der?
% Passt das hier von der Länge und dem Inhalt? 
