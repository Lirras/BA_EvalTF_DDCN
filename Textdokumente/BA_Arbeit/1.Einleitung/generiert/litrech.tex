Neuronale Netzwerke werden für eine Vielzahl von Aufgaben eingesetzt, die ein rechenintensives Vorgehen erfordern und dabei eine Architektur 
nutzen, die dem menschlichen Gehirn nachempfunden ist. Neben klassischen Anwendungen im Bereich der Computer Vision existieren auch 
Einsatzgebiete in der Steuerung und Planung, wie sie beispielsweise bei strategischen Spielen wie Go relevant sind 
\cite{deep_neural_networks_scientific_models}.

Ein weiteres Anwendungsfeld stellt die Klassifikation dar, etwa bei der Erkennung handgeschriebener Ziffern \cite{handwritten_digit}. 
Zudem werden neuronale Netzwerke zur Approximation von Funktionen eingesetzt, typischerweise im Rahmen von Regressionsaufgaben \cite{Gen_Reg}.

In der Praxis ergeben sich jedoch verschiedene Herausforderungen. Eine häufige Problematik ist die unzureichende Verfügbarkeit großer, 
qualitativ hochwertiger Datensätze für bestimmte Aufgabenstellungen \cite{survey_transfer}. Darüber hinaus kann das Training eines 
leistungsfähigen neuronalen Netzwerks äußerst zeitintensiv sein \cite{cascor}. Generell gilt für alle neuronalen Modelle, dass ihre 
Vorhersagen mit einer gewissen Fehlerrate behaftet sind und somit nicht vollständig korrekt sind \cite{EvoClassAndReg}.

Transfer Learning (TF) bezeichnet ein maschinelles Lernparadigma, bei dem bereits erworbenes Wissen aus einer Quellaufgabe (Source) genutzt 
wird, um die Lernleistung auf einer Zielaufgabe (Target) zu verbessern \cite{transfer_learning}. Dabei erlernt ein neuronales Netzwerk 
zunächst eine Aufgabe und überträgt anschließend relevante Erkenntnisse auf eine andere, meist verwandte Aufgabe \cite{phd_deep_cascade}.

Dieses Vorgehen lässt sich mit dem menschlichen Lernverhalten vergleichen, bei dem über Eselsbrücken bereits bestehendes Wissen herangezogen 
wird, um neues Wissen effizienter zu erwerben. Ein Netzwerk, das Transfer Learning betreibt, vollzieht folglich einen kontextuellen Wechsel 
zwischen verschiedenen Lernaufgaben. Dabei ergeben sich insbesondere drei zentrale Forschungsfragen:

\begin{enumerate}
    \item What to transfer
    \item How to transfer
    \item When to transfer
\end{enumerate}
\cite{survey_transfer}

Die erste zentrale Forschungsfrage im Transfer Learning betrifft die Beschaffenheit und Auswahl der Quelldaten, also jener Datenbasis, auf der 
das übertragbare Wissen – bildlich gesprochen die "Eselsbrücke" – aufgebaut wird.

Die zweite Fragestellung bezieht sich auf die Rolle der Lernalgorithmen, insbesondere darauf, wann der Übergang vom Lernen auf der Quell- zur 
Zielaufgabe erfolgen soll. In der Praxis wird dieser Übergang häufig durch Fehlermetriken gesteuert: Sobald das Modell einen bestimmten 
Schwellenwert (Threshold) unterschreitet und die Fehlerquote als ausreichend niedrig eingestuft wird, erfolgt der Transfer. Der optimale 
Zeitpunkt für diesen Wechsel kann jedoch experimentell bestimmt und variiert werden, um maximale Leistung zu erzielen. Es ist jedoch wichtig 
zu beachten, dass Transfer Learning nicht zwangsläufig zu einer Leistungssteigerung führt. In bestimmten Fällen kann die Übertragung von Wissen 
sogar zu einer Verschlechterung der Modellleistung führen – ein Phänomen, das als Negative Transfer bezeichnet wird \cite{survey_transfer}.

Transfer Learning setzt in mindestens einem von zwei Bereichen einen Wechsel voraus: der Domain und/oder dem Task. Die Domain umfasst Merkmale 
wie die Verteilung, Repräsentation oder den strukturellen Aufbau der Daten (z.B. Formate, Dimensionen), während der Task sich auf die konkrete 
Lernaufgabe bezieht, wie beispielsweise Bildklassifikation oder Segmentierung \cite{survey_transfer}.

In komplexeren Szenarien kann der Transfer nicht direkt von der Quell- zur Zieldomain erfolgen. Stattdessen wird eine intermediäre, sogenannte 
Brückendomain (Bridge Domain) eingeführt, über die der Wissenstransfer schrittweise erfolgt. Dieses Vorgehen wird als Bridge Transfer Learning 
bezeichnet und findet Anwendung, wenn die Unterschiede im Feature Space zwischen Quelle und Ziel zu groß sind oder die direkte Übertragung zu 
negativem Transfer führen würde \cite{bridge_transfer, survey_transfer}.

Während Deep Learning grundsätzlich in der Lage ist, komplexere Aufgaben zu lösen als sogenanntes Shallow Learning, ist es gleichzeitig durch 
einen erheblich höheren Trainingsaufwand gekennzeichnet. Wenn dieser hohe Rechenaufwand nicht praktikabel ist, kommt Cascade Learning zum 
Einsatz \cite{cascor}.

Im klassischen Deep Transfer Learning wird in der Regel ein vortrainiertes Netzwerk genutzt, wobei der finale Feature-Vektor als Ausgangspunkt 
für den Transfer dient. Dies beruht auf der Annahme, dass in dieser Repräsentation die bedeutendsten Informationen verdichtet vorliegen – ein 
Ansatz, der als Feature Representation Transfer bezeichnet wird \cite{survey_transfer}.

Im Gegensatz dazu verfolgt das Cascade Learning einen iterativen Ansatz: In jeder Trainingsiteration wird der aktuell generierte Feature-Vektor 
in den Transferprozess einbezogen, um herauszufinden, welcher Repräsentationsvektor tatsächlich den größten Informationsgehalt besitzt. Dieser 
Ansatz führt in der Regel zu einer effizienteren und weniger komplexen Netzwerkstruktur im Vergleich zum klassischen Deep Transfer Learning 
\cite{phd_deep_cascade}.

Cascade-Netzwerke zeichnen sich dadurch aus, dass ihre Architektur nicht vor Beginn des Trainings vollständig definiert ist, sondern 
schrittweise während des Trainingsprozesses aufgebaut wird. Hierfür kommen sogenannte Constructive Algorithms zum Einsatz, deren wesentlicher 
Vorteil darin liegt, dass keine exakte Festlegung der Netzwerkgröße im Vorfeld erforderlich ist \cite{Constructive_Cascade}. Das Netzwerk wächst 
lediglich so weit, wie es zur Lösung der jeweiligen Aufgabe notwendig ist, wodurch eine übermäßige Modellkomplexität vermieden und gleichzeitig 
die Trainingszeit reduziert wird \cite{Constructive_Cascade, cascor}.

Ein prominentes Beispiel für einen solchen Ansatz ist das Cascade Correlation (CasCor). Bei diesem Verfahren wird das Netzwerk schrittweise 
durch das Hinzufügen einzelner Neuronen (Perzeptrons) erweitert. Jedes neue Perzeptron wird so trainiert, dass es eine möglichst hohe 
Korrelation mit dem verbleibenden Fehler aufweist. Nach Abschluss des Trainings wird das betreffende Neuron fixiert, das heißt, seine 
Gewichtungen werden eingefroren und bleiben im weiteren Verlauf unverändert. Anschließend wird überprüft, ob das Netzwerk die angestrebte 
Fehlergrenze erreicht hat. Ist dies nicht der Fall, wird ein weiteres Perzeptron ergänzt und der Prozess wiederholt sich, bis die gewünschte 
Modellgüte erreicht ist \cite{cascor}.

Dieser iterative Aufbau sowie das Einfrieren bereits gelernter Gewichtungen ermöglichen ein effizientes Training, bei dem keine Backpropagation 
durch das gesamte Netzwerk erforderlich ist. Lediglich die Gewichte des neu hinzugefügten Neurons werden jeweils angepasst. Da jedes neue 
Neuron direkt mit dem Ausgang des Netzwerks verbunden ist, trägt es unmittelbar zur Fehlerkorrektur bei \cite{cascor}.

Ein verwandtes Konzept stellt das Direct Cascade Learning dar. Hierbei erfolgt keine Zwischenverarbeitung der Ausgaben bereits trainierter 
Neuronen. Stattdessen dient der Output eines Perzeptrons direkt als Input für das nachfolgende, wie es beispielsweise in den Architekturen 
Cascade QEF und Cascade LLM umgesetzt ist \cite{cascade_network_architectures,cascade_llm_networks}.

Obwohl Transfer Learning (TF) bislang vorrangig mit klassischen tiefen Lernverfahren kombiniert wurde, existieren erste Ansätze, die TF mit 
Cascade Learning verknüpfen. Ein Beispiel hierfür ist das Cascade Transfer Learning (CTL), das Cascade Correlation mit Transfer Learning 
kombiniert \cite{phd_deep_cascade}. Zwar blieb die Leistung von CTL leicht hinter der von klassischen Fine-Tuning-Verfahren zurück, jedoch 
zeichnete sich CTL durch eine deutlich geringere Speicheranforderung aus \cite{phd_deep_cascade}.

Neben CasCor existieren weitere Cascade-Lernalgorithmen – insbesondere solche, die dem Direct Cascade-Prinzip folgen –, die bislang noch nicht 
im Kontext von Transfer Learning eingesetzt wurden und somit Potenzial für zukünftige Forschung bieten.
