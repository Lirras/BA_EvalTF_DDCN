Kapitel 2 enthält eine Einführung in die grundlegenden Konzepte dieser Arbeit. Es werden die Kaskadierungsverfahren, das Prinzip des 
TFs, das verwendete Rechensystem, die eingesetzten Early-Stopping-Metriken sowie die für das Training verwendeten Datensätze 
vorgestellt. Darüber hinaus erfolgt eine Übersicht über alle durchgeführten Tests einschließlich ihrer jeweiligen Kennungen und Zielsetzungen.

Kapitel 3 widmet sich den systematisch auftretenden Phänomenen beim domänenübergreifenden TF. Im Fokus stehen hierbei der 
Leistungsabfall unmittelbar nach dem Transfer sowie das Overfitting auf den Source-Datensatz. Zur Analyse wird ein 
Deep-Cascade-Klassifikationsnetzwerk eingesetzt. Zudem werden in diesem Kapitel die Trainingszeiten aller betrachteten Netzarchitekturen 
verglichen und die Ergebnisse interpretiert.

Kapitel 4 behandelt ausschließlich beobachtete Effekte, die spezifisch im Kontext der Klassifikationsaufgabe auftreten. Dazu zählen die 
Auswirkungen unterschiedlicher Target-Datenmengen, die Erstellung sogenannter Augmented Vectors sowie der Vergleich 
unterschiedlicher Netzarchitekturen bei identischer Layer-Konfiguration. Ziel ist es, zu klären, warum TF im Rahmen von 
Kaskadierung bei Klassifikationsaufgaben nicht zuverlässig funktioniert.

Kapitel 5 fokussiert sich auf regressionsspezifische Beobachtungen. Hierfür werden sowohl Szenarien mit großen als auch mit kleinen Target-Datenmengen 
untersucht, wobei ein Vergleich zwischen Cascade TF, klassischer Kaskadierung und vollständig trainierten Netzwerken erfolgt. 
Dabei zeigt sich, dass TF in bestimmten Konstellationen zu besseren Ergebnissen führt. Da in diesem Kontext Cascade TF 
funktional einsetzbar ist, werden hier zusätzlich die Auswirkungen verschiedener Early-Stopping-Metriken betrachtet.

Kapitel 6 bietet eine zusammenfassende Darstellung der zentralen Erkenntnisse dieser Arbeit sowie einen Ausblick auf potenzielle weiterführende 
Forschungsfragen in diesem Themenfeld.
