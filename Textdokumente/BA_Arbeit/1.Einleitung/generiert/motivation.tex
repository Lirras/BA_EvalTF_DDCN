In vielen Anwendungsfällen, für die der Einsatz von KI grundsätzlich sinnvoll wäre, stehen lediglich sehr kleine Datensätze 
zur Verfügung. Diese Datenmengen sind oft so begrenzt, dass ein neuronales Netzwerk nicht ausreichend Informationen erhält, um ein robustes und 
generalisierbares Modell zu erlernen. % – selbst bei verlängerter Trainingsdauer bleibt die Modellleistung unzureichend.

Zusätzlich stellt die lange Trainingszeit klassischer neuronaler Netzwerke eine weitere Herausforderung dar. Beide Probleme – unzureichende 
Datenverfügbarkeit und hoher Trainingsaufwand – sollen in dieser Arbeit adressiert und verbessert werden.

Zur Bewältigung der Problematik kleiner Datensätze wird ein auf TF basie-render Ansatz verfolgt. Dabei lernt das Modell zunächst 
auf einer verwandten, aber besser verfügbaren Datenbasis, um dieses Vorwissen anschließend auf die eigentliche Zielaufgabe zu übertragen. Dieses 
Vorgehen orientiert sich an kognitiven Lernprozessen beim Menschen, bei denen bereits erlernte Konzepte genutzt werden, um neue, ähnliche 
Inhalte zu erschließen. % – vergleichbar mit der Funktionsweise von Eselsbrücken.

Die Netzwerkarchitektur wird derart gestaltet, dass jeweils nur ein geringer Teil des Modells gleichzeitig trainiert wird. Dadurch soll der 
Trainingsaufwand reduziert und eine signifikante zeitliche Effizienzsteigerung erzielt werden. Zur Umsetzung dieses Ansatzes kommen die 
Kaskadierungsverfahren Deep Cascade und Direct Cascade zum Einsatz.

Es soll untersucht werden, ob Netzwerke mit der Direct Cascade Architektur mit TF eine ausreichende 
Generalisierungsfähigkeit aufweist, sodass die Modell-leistung auf bislang unbekannten Datensätzen nicht signifikant abnimmt.
