
Liste aller hier vorkommenden Netzen mit ihren Kürzeln:

\begin{enumerate}
    \item ConvMaxPool (CMP)
    \item 1DConv (1DC)
    \item 2DConv (2DC)
    \item ClassOneDense (COD)
    \item RegressionTwo (Regr2)
    \item OneLayer (1Lay)
\end{enumerate}

Davon sind ConvMaxPool und RegressionTwo Deep Cascade Netzwerke, während alle anderen Direct Cascade Netzwerke sind. 
Ebenso sind nur RegressionTwo und OneLayer Regressionsnetze, während der Rest Klassifikationsnetze sind. 

Alle Netze werden mit dem Adam-Optimizer mit der Lernrate 1e-3 gelernt. Klassifikationsnetze haben als Loss den 
CategoricalCrossEntropy und Softmax als Aktivierungsfunktion, während die Regressionsnetze MeanSquaredError und Linear als 
Aktivierungsfunktion vorweisen. 

Mit allen Direct Cascade Netzwerken wurden zusätzlich Early Stopping Metriken durchgeführt mit MAEM, LM und ACCM. 

Für alle Klassifikationsnetze gilt, dass sie mit fünf verschiedenen Größen des Targetdatensatzes trainiert wurden. Die Ausnahme ist das 
2DC-Netzwerk, welches nur mit sehr wenigen Source- und Targetdaten trainiert werden kann, da es technisch auf derselben Hardware mit mehr 
Daten unmöglich ist. 

Bei den Regressionsnetzen wird jeweils einmal mit vielen und wenigen Targetdaten trainiert. 

Es wurde mit allen Netzwerken ein Vergleich sowohl zwischen mit TF und ohne als auch zwischen ohne TF und Kompletten angefertigt. 
Komplett heißt hier, dass es ein Netzwerk ohne TF und ohne Kaskadierung ist und dieses deshalb in einem komplett trainiert wird. 

Mit allen Netzwerken wurde der Zeitpunkt für das TF frei ausgetestet. 

Alle Direct Cascade Netzwerke haben jeweils nur ein Hidden Layer. In manchen Fällen sind sie noch mit einem Hilfslayer, um den Wechsel 
zwischen Filterlayern und Linearlayern zu bewerkstelligen. 

Für alle Netzwerke wurde derselbe Seed für die Initialisierung der Weights genutzt. 

In Tabelle 2.1 sind die Tests bezüglich Klassifikation und in Tabelle 2.2 die für die Regression. 
In beiden Tabellen sind die Tests, die sich mit der Zeitdauer befassen mit der Endung Time. Dabei gilt, dass CasTF Kaskadierung mit TF, Cas allein 
Kaskadierung ohne TF und Comp bedeutet, dass es weder TF noch Kaskadierung gab. ACCM, LM und MAEM sind die Tests bezüglich der Early-Stopping 
Metriken. 
Vor dem ersten Schrägstrich steht, wann TF gemacht wurde, welches mit TF im Eintrag gekennzeichnet ist. Wenn kein TF gemacht wurde, ist dieser 
erste Bereich nicht existent. Dahinter steht die Datenmenge 
des Targetdatensatzes und danach die Menge an Epochen pro Trainingsiteration. Wenn es noch etwas viertes gibt, dann zeigt dieses an, wieviele 
Epochen in Zehnern es insgesamt gab. 

\begin{table}[!ht]
    \centering
    \begin{tabular}{l|l|l|l}
        \textbf{CMP} & \textbf{COD} & \textbf{1DC} & \textbf{2DC} \\
        \hline
        TF0/732/10 & CasTFTime & CasTFTime & CasTFTime \\
        TF1/732/10 & CasTime & CasTime & CasTime \\
        TF2/732/10 & CompTime & CompTime & CompTime \\
        TF3/732/10 & TF2/732/10 & TF2/732/10 & TF2/732/10 \\
        TF4/732/10 & TF2/7k/10 & TF2/7k/10 & \\
        TF5/732/10 & TF2/21k/10 & TF2/21k/10 & \\
        732/10 & TF2/36k/10 & TF2/36k/10 & \\
        CasTFTime & TF2/51k/10 & TF2/51k/10 & \\
        CasTime & TF10/732/10/30 & TF10/732/10/30 & \\
        CompTime & 732/10/30 & 732/10/30 & \\
        TF2/7k/10 & Comp/732//30 & Comp/732//30 & \\
        TF2/21k/10 & ACCM/732/10 & ACCM/732/10 & \\
        TF2/36k/10 & LM/732/10 & LM/732/10 & \\
        TF2/51k/10 & & & \\
        & & & \\
    \end{tabular}
    \caption{\label{tab:classtests} Liste aller Klassifikationstests}
\end{table}

\begin{table}[!ht]
    \centering
    \begin{tabular}{l|l}
        \textbf{Regr2} & \textbf{1Lay} \\
        \hline
        TF0/240/25 & CasTFTime \\
        TF1/240/25 & CasTime \\
        TF4/240/25 & CompTime \\
        CasTFTime & TF11/8k/10/21 \\
        CasTime & 8k/10/11 \\
        CompTime & Comp/8k//8 \\
        TF4/8k/10/8 & TF11/240/10/21 \\
        8k/10/8 & 240/10/11 \\
        Comp/8k//8 & Comp/240//8 \\
        TF4/240/10/8 & MAEM/240/10 \\
        240/10/8 & LM/240/10 \\
        Comp/240//8 & TF4/206/10/8/ts \\
        TF4/206/10/8/ts & 206/10/8/ts \\
        206/10/8/ts & Comp/206//8/ts \\
        Comp/206//8/ts &
    \end{tabular}
    \caption{\label{tab:regrtests} Liste aller Regressionstests}
\end{table}

Eine Referenz zu einer dieser Listen ist CMP:TF0/732/10. Diese bedeutet, dass es um den Test mit der Kennung TF0/732/10 des CMP-Netzwerkes geht. 
Bei diesem gibt es die Besonderheit, dass TF nach dem ersten Layer gemacht wird, dieses Layer jedoch im 
Gegensatz zum restlichen Netzwerk nur mit einer Epoche trainiert wird.
Selbiges gilt für Regr2:TF0/240/25.

Die Tests mit der Endung ts sind diejenigen, die einen explizit großen Testdatensatz haben. 
