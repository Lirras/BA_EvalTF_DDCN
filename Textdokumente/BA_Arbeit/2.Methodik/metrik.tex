Es wurden drei Metriken erstellt. 
Die Accuracy- (ACCM), Loss- (LM) und MAE-Metrik (MAEM). MAE heißt dabei Mean Absolute Error. 
Alle drei Metriken sind für Early Stopping und entscheiden, wieviele Epochen genutzt werden. 
Die Accuracy-Metrik bricht immer dann ab, wenn die Validation-Accuracy mindestens um 10\% 
schlechter ist als die Trainingsaccuracy, da dann in dem Netzwerk Overfitting herrscht.

Die Loss- und die MAE-Metrik brechen beide dann ab, wenn der Validation-Wert der aktuellen 
Epoche schlechter ist als in der Epoche davor. Dies hat zur Folge, dass die Netze in lokale 
Minima hineinlaufen und nicht wieder herauskommen. Dabei unterliegt die Anzahl der Netze für 
das Direct Cascade keiner Metrik.

% Evtl eine Metrik bauen, die über einen Max-Value für ACC geht und dann den Abbruch macht, wenn es besser 
% wird oder schlechter als dieser Wert(Muss rel. enger Bereich sein)
% Ebenso für Min-Value für Loss und MAE
