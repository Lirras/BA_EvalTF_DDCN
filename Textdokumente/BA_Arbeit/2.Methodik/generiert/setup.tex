Alle Experimente wurden auf einem Erazer Gaming Notebook P15601 mit Windows 10 durchgeführt. Die neuronalen Netzwerke wurden ausschließlich auf 
der CPU ausgeführt und ausschließlich im Netzbetrieb trainiert. Das System ist mit einem Intel Core i5 der 9. Generation ausgestattet, welcher 
über 4 physische Kerne und 8 logische Prozessoren verfügt. Die Taktrate liegt im Bereich von 2,4 bis 5,1 GHz. Der Arbeitsspeicher umfasst 15,8 GB 
mit einer Geschwindigkeit von 2667 MHz.

Die Implementierung erfolgte unter Verwendung von PyCharm als Entwicklungsumgebung sowie der Keras-Bibliothek für das neuronale 
Netzwerk-Framework. Die Dokumentation wurde mithilfe von BibTeX erstellt, und die Visualisierungen basieren auf der Matplotlib-Bibliothek.

Als Quell-Datensätze (Source-Daten) wurden MNIST und Boston Housing verwendet, während SVHN und California Housing als Ziel-Datensätze 
(Target-Daten) dienen. Die Ziel-Datensätze wurden manuell reduziert, um bewusst eine unzureichende Datenmenge zu simulieren, sodass alternative 
Lernmethoden notwendig sind.

Es ist zu beachten, dass keiner der verwendeten Datensätze in normalisierter Form als Eingabe in die Modelle eingespeist wurde.
