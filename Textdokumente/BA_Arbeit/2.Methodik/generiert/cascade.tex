In diesem Abschnitt wird das Konzept von Kaskadennetzwerken erläutert, einschließlich ihrer charakteristischen Merkmale. Ein Kaskadennetzwerk ist 
eine spezielle Architekturform neuronaler Netzwerke, bei der das Netzwerk schrittweise – in sogenannten Kaskaden – aufgebaut wird. Im Gegensatz zu 
konventionellen neuronalen Netzwerken, bei denen die gesamte Netzwerkstruktur (Anzahl und Art der Layer sowie der Subnetzwerke) im Vorfeld 
vollständig definiert ist, erfolgt die Konstruktion eines Kaskadennetzwerks iterativ während des Trainingsprozesses.

Dabei wird jeweils nur der neu hinzugefügte Netzwerkabschnitt trainiert, während die zuvor trainierten Komponenten eingefroren werden. 
Das bedeutet, dass die Gewichtungen dieser eingefrorenen Layer und Subnetzwerke nach ihrer Trainingsphase nicht mehr angepasst werden. Diese 
Vorgehensweise führt dazu, dass Interaktionen zwischen den verschiedenen Netzwerkstufen – insbesondere zwischen bereits trainierten und neu 
hinzukommenden Komponenten – nicht erlernt werden können. Daher erzielen Kaskadennetzwerke bei gleicher Datenbasis in der Regel eine 
geringere Modellgüte als klassische Netzwerke mit vollständig trainierbarer Struktur.

Ein wesentlicher Vorteil der Kaskadenarchitektur liegt jedoch in der Trainingsgeschwindigkeit. Da jeweils nur ein Teil des Netzwerks aktiv 
trainiert wird und die eingefrorenen Gewichte unverändert bleiben, entfallen die Berechnungen zur Gradientenaktualisierung für den Großteil des 
Netzwerks. Dies reduziert den Rechenaufwand pro Trainingsschritt erheblich und führt zu einer insgesamt schnelleren Trainingsphase.
