Transferlernen (Transfer Learning, TL) basiert auf dem Prinzip der Wissensübertragung zwischen unterschiedlichen, jedoch verwandten 
Lernsituationen – vergleichbar mit dem Konzept einer "Eselsbrücke". Es existieren verschiedene Formen des Transferlernens, wobei 
insbesondere zwischen Task Transfer (Aufgabenwechsel) und Domain Transfer (Domänenwechsel) unterschieden wird. Wird keine dieser Varianten 
angewendet, handelt es sich nicht um Transferlernen im engeren Sinne.

In dieser Arbeit wird ausschließlich der Domain Transfer betrachtet, da ausschließlich dieser zum Einsatz kommt. Dabei wird ein Modell, das auf 
einem bestimmten Datensatz (Source-Domain) trainiert wurde, zur Verbesserung der Leistung auf einem anderen, inhaltlich verwandten Datensatz 
(Target-Domain) genutzt – unter Beibehaltung der gleichen Modellarchitektur. Dieser Ansatz wird als Transductive Transfer Learning bezeichnet 
\cite{survey_transfer}. Ziel ist es, das in der Source-Domain erworbene Wissen auf die Target-Domain zu übertragen.

Ein zentrales Problem des Transferlernens besteht darin, geeignete Antworten auf die drei grundlegenden Fragen zu finden: What to transfer, How 
to transfer und When to transfer \cite{survey_transfer}. Diese Aspekte sind bislang nicht abschließend geklärt und stellen weiterhin offene 
Forschungsfragen dar.

Da sowohl Klassifikations- als auch Regressionsmodelle evaluiert werden, kommen jeweils zwei Datensätze als Source und Target zum Einsatz:

\textbf{Klassifikation:}
\begin{enumerate}
    \item Source-Datensatz: Modified National Institute of Standards and Technology (MNIST) \cite{handwritten_digit}
    \item Target-Datensatz: Street View House Numbers (SVHN) \cite{house_numbers}
\end{enumerate}

Beide Datensätze müssen zur strukturellen Kompatibilität für das Transferlernen angepasst werden. MNIST-Bilder (ursprünglich 28x28 Pixel, 
einfarbig) werden auf 32x32 Pixel skaliert. Der SVHN-Datensatz, ursprünglich in Farbe (RGB), wird in Graustufen konvertiert. Dies gewährleistet, 
dass beide Eingabedatensätze die gleiche Dimensionalität aufweisen und somit als Input für dasselbe neuronale Netz genutzt werden können.

\textbf{Regression:}
\begin{enumerate}
    \item Source-Datensatz: Boston Housing Prices (BOST) \cite{Boston_housing}
    \item Target-Datensatz: California Housing Prices (CALI) \cite{California_housing}
\end{enumerate}

Beide Regressionsdatensätze müssen inhaltlich reduziert werden, da nur solche Merkmale berücksichtigt werden sollen, die eine semantisch und 
strukturell sinnvolle Entsprechung in beiden Domänen aufweisen. Im BOST-Datensatz verbleiben nur die drei Merkmale:

\begin{enumerate}
    \item RM: durchschnittliche Anzahl von Zimmern pro Wohnung
    \item AGE: Anteil der vor 1940 erbauten Häuser
    \item LSTAT: Anteil der Bevölkerung mit niedrigem sozioökonomischem Status
\end{enumerate}

Hinweis: Die ursprünglich im BOST-Datensatz enthaltene Variable mit ethnisch diskriminierendem Inhalt wird vollständig entfernt.

Im CALI-Datensatz verbleiben:
\begin{enumerate}
    \item MedInc: mittleres Einkommen pro Häuserblock
    \item HouseAge: durchschnittliches Alter der Gebäude
    \item RoomsPerHousehold: abgeleitet durch Division von AveRooms (durchschnittliche Zimmeranzahl im Block) durch Households (Anzahl der Haushalte)
\end{enumerate}

Diese Ableitung stellt die strukturelle Entsprechung zur RM-Variable im BOST-Datensatz dar.

Eine inhaltliche Verbindung besteht voraussichtlich zwischen LSTAT (BOST) und MedInc (CALI), da sozioökonomischer Status und Einkommen 
erfahrungsgemäß korrelieren. Da LSTAT jedoch antiproportional zur Einkommenshöhe ist, wird diese Variable invertiert, um eine proportionale 
Beziehung zur MedInc-Variable herzustellen – eine Voraussetzung für effektives Transferlernen.

Die Harmonisierung der Altersvariablen gestaltet sich komplexer: AGE im BOST-Datensatz beschreibt den Anteil der vor 1940 erbauten Häuser, 
während HouseAge im CALI-Datensatz das durchschnittliche Alter der Gebäude angibt. Um AGE in eine vergleichbare Form zu bringen, wird auf 
Basis einer angenommenen maximalen Anzahl von 100 Häusern pro Block und einem maximalen Alter von 85 Jahren (bezogen auf das Jahr 2025) eine 
Umrechnung vorgenommen. Die entsprechende Formel lautet:
\begin{equation}
    \frac{AGE * 85}{Maximalanzahl}
\end{equation}

Durch die beschriebenen Anpassungen weisen alle verwendeten Source- und Target-Datensätze strukturelle Kompatibilität zueinander auf. Damit ist 
die Frage "Was wird transferiert?" im Sinne des Transferlernens hinreichend beantwortet.

Die nächste zu klärende Dimension ist das "How to Transfer". Der Transfer erfolgt in allen Fällen ohne Modifikation der Gewichtungen zuvor 
trainierter Netzwerkschichten. Das neuronale Netzwerk wird zunächst vollständig auf dem Source-Datensatz trainiert. Anschließend erfolgt der 
Übergang zum Target-Datensatz, ohne dass das Modell selbst oder dessen Gewichtungen verändert werden.

Dabei unterscheiden sich die beiden eingesetzten Kaskadierungsansätze hinsichtlich der Art des Transfers:
\begin{enumerate}
    \item Beim Deep Cascade-Verfahren bleibt die Netzinputgröße unverändert, und lediglich die Eingabedatenquelle (Source → Target) wird gewechselt. 
    Der Transfer vollzieht sich somit ausschließlich über die Änderung des Input-Datensatzes bei gleichbleibender Architektur und 
    Parameterbelegung. Dabei ist zu bedenken, dass dieses Netz immer größer wird.
    \item Beim Direct Cascade-Verfahren erfolgt der Transfer implizit über die kontinuierliche Erweiterung des Eingaberaums. Der sogenannte 
    Augmented Input wird bei jedem Schritt vergrößert, sodass die auf dem Source-Datensatz erlernten Repräsentationen in den nachfolgenden 
    Stufen weiterhin einfließen.
\end{enumerate}
    
    
Die Frage "Wann ist Transfer Learning sinnvoll?" (When to Transfer) kann bislang nicht eindeutig beantwortet werden. Aus diesem Grund wird in 
den Experimenten mit unterschiedlichen Zeitpunkten für den Transfer gearbeitet – sowohl mit einem frühen als auch mit einem späteren Übergang 
zum Target-Datensatz.
