Hier wird erklärt, was ein Kaskadennetzwerk ist und welche Besonderheiten es dabei gibt. 
Ein Kaskadennetzwerk ist ein Netzwerk, welches in Kaskaden, also Schrittweise, aufgebaut wird. Während bei einem klassischen Netz 
vorher festgelegt wird, wieviele und welche Layer und Unternetzwerke dieses haben wird, ist es bei einem Kaskadennetzwerk nicht so. Ein solches 
Netzwerk 
wird erst während dem Training aufgebaut und immer erweitert. Deshalb werden, im Gegensatz zu den klassischen Netzen, nur der aktuelle neue 
Teil trainiert, während der Rest nicht mehr verändert wird. Die vorher gelernten Layer und Unternetzwerke werden nach dem Training gefreezt. 
Dadurch 
werden die Gewichte der gefreezten Layer und Unternetzwerke nicht mehr verändert. Die Kaskadennetzwerke lernen dadurch das, was zwischen den 
Layern und Unternetzwerken 
gelernt wird, nicht und sind deshalb etwas schlechter als die klassischen Netzwerke bei gleich vielen Daten. Aber, weil immer nur das 
Aktuelle gelernt wird, sollten Kaskadennetzwerke im Training sehr viel schneller sein. Dies liegt daran, dass die Gewichte der vorherigen Layer 
kein sich verändertes Bild von den nachfolgenden Gewichten in jeder Epoche haben und sich nicht aktualisieren müssen. 
