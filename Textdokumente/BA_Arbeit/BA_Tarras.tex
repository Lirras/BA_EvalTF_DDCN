\documentclass[ngerman]{report}

\usepackage[
        bibencoding=utf8, 
        style=alphabetic
    ]{biblatex}

\usepackage{graphicx}
\usepackage{amsmath}
\usepackage{caption}
\usepackage{hyperref}
\usepackage[headsepline]{scrlayer-scrpage}
\pagestyle{scrheadings}

\setlength{\footheight}{24.0pt}

\ihead{\headmark}
\automark{section}

\cfoot{Der hier hinterliegende Code, die Plots und die Textausarbeitung ist zu finden unter: 
    \url{https://github.com/Lirras/BA_EvalTF_DDCN}.}

\bibliography{bibliography}

\title{Evaluierung von Transferlernen mit Deep Direct Cascade Networks}

\author{Simon Tarras}

\begin{document}
    \maketitle
    \tableofcontents

    \chapter{Einleitung}  % 1.Einleitung
    \section{Einführung}
    % Ausblick auf die Arbeit selbst
    Künstliche Intelligenz (KI) ist mittlerweile selbst außerhalb der Informatik einer breiten Öffentlichkeit bekannt. Die zugrunde liegende 
Technologie basiert in der Regel auf künstlichen neuronalen Netzen, die üblicherweise in einem einzigen Schritt vollständig konstruiert und 
trainiert werden. Dieser Prozess ist jedoch zeit- und rechenintensiv, weshalb alternative Verfahren wie die Cascade-Correlation-Methode 
entwickelt wurden \cite{cascor}. Erste Untersuchungen zeigten, dass diese Kaskadierungsstrategie bereits bei kleineren Netzarchitekturen 
zufriedenstellende Ergebnisse liefert. Darauf aufbauend wurden verschiedene erweiterte Netzwerkstrukturen und Kaskadierungsverfahren 
entwickelt \cite{cascade_network_architectures}, \cite{Constructive_Cascade}, \cite{deep_cascade_learning}.

Ein wesentlicher Vorteil kaskadierender Netzwerke liegt in ihrer modularen Struktur, die eine flexible Anpassung an unterschiedliche 
Datensätze und Aufgabenstellungen ermöglicht \cite{phd_deep_cascade}, \cite{transfer_learning}, \cite{survey_transfer}. Die vorliegende 
Arbeit ist im Kontext dieser Transferfähigkeit einzuordnen. Ziel ist es, die Leistungsfähigkeit verschiedener Netzwerkarchitekturen zu 
evaluieren und spezifische Herausforderungen im Zusammenhang mit Transferlernen (TF) und Kaskadierung zu analysieren.

    \section{Motivation}
    In vielen Anwendungsfällen, für die der Einsatz von KI grundsätzlich sinnvoll wäre, stehen lediglich sehr kleine Datensätze 
zur Verfügung. Diese Datenmengen sind oft so begrenzt, dass ein neuronales Netzwerk nicht ausreichend Informationen erhält, um ein robustes und 
generalisierbares Modell zu erlernen. % – selbst bei verlängerter Trainingsdauer bleibt die Modellleistung unzureichend.

Zusätzlich stellt die lange Trainingszeit klassischer neuronaler Netzwerke eine weitere Herausforderung dar. Beide Probleme – unzureichende 
Datenverfügbarkeit und hoher Trainingsaufwand – sollen in dieser Arbeit adressiert und verbessert werden.

Zur Bewältigung der Problematik kleiner Datensätze wird ein auf TF basierender Ansatz verfolgt. Dabei lernt das Modell zunächst 
auf einer verwandten, aber besser verfügbaren Datenbasis, um dieses Vorwissen anschließend auf die eigentliche Zielaufgabe zu übertragen. Dieses 
Vorgehen orientiert sich an kognitiven Lernprozessen beim Menschen, bei denen bereits erlernte Konzepte genutzt werden, um neue, ähnliche 
Inhalte zu erschließen. % – vergleichbar mit der Funktionsweise von Eselsbrücken.

Die Netzwerkarchitektur wird derart gestaltet, dass jeweils nur ein geringer Teil des Modells gleichzeitig trainiert wird. Dadurch soll der 
Trainingsaufwand reduziert und eine signifikante zeitliche Effizienzsteigerung erzielt werden. Zur Umsetzung dieses Ansatzes kommen die 
Kaskadierungsverfahren Deep Cascade und Direct Cascade zum Einsatz.

    % Evtl. Übersicht
    \section{Related Work}
    % Literaturrecherche
    Diese Arbeit baut auf den CasCor-Algorithmus \cite{cascor}, der wegen der langen Trainingsdauer entwickelt wurde, auf und nutzt diesen, 
um Direct Cascade \cite{cascade_network_architectures} 
durchzuführen, wie es in der Thesis von Marquez \cite{phd_deep_cascade} verwendet wurde. Zudem wird Deep Cascade verwendet als 
Vergleichsbasis \cite{deep_cascade_learning}. Dabei sind Deep Cascade Netzwerke solche, die iterativ aufgebaut werden \cite{Constructive_Cascade}. 
Es wird nur Domain-Wechsel angewandt, während es noch den Taskwechsel \cite{transfer_learning} gibt. Zudem gibt es bei Transfer Learning drei 
Probleme \cite{survey_transfer}, die hier auch bearbeitet werden. 

% Alle wissenschaftlichen Arbeiten, die hier dazu gehören und verwandt sind aufzählen. 

% Also die Direct Cascade Arbeiten von Ritter und Littmann zum Beispiel. 

% Ebenso Cascor von Fahlmann und Lebiere.

% Deep Cascade Learning von Enrico S. Marquez. 

% Wie sieht das mit dem Bachelorvortrag aus? Wieviel später kommt der?
% Passt das hier von der Länge und dem Inhalt? 

    \newline
Neuronale Netzwerke werden für eine Vielzahl von Aufgaben eingesetzt, die ein rechenintensives Vorgehen erfordern und dabei eine Architektur 
nutzen, die dem menschlichen Gehirn nachempfunden ist. Neben klassischen Anwendungen im Bereich der Computer Vision existieren auch 
Einsatzgebiete in der Steuerung und Planung, wie sie beispielsweise bei strategischen Spielen wie Go relevant sind 
\cite{deep_neural_networks_scientific_models}.

Ein weiteres Anwendungsfeld stellt die Klassifikation dar, etwa bei der Erkennung handgeschriebener Ziffern \cite{handwritten_digit}. 
Zudem werden neuronale Netzwerke zur Approximation von Funktionen eingesetzt, typischerweise im Rahmen von Regressionsaufgaben \cite{Gen_Reg}.

In der Praxis ergeben sich jedoch verschiedene Herausforderungen. Eine häufige Problematik ist die unzureichende Verfügbarkeit großer, 
qualitativ hoch-wertiger Datensätze für bestimmte Aufgabenstellungen \cite{survey_transfer}. Darüber hinaus kann das Training eines 
leistungsfähigen neuronalen Netzwerks äußerst zeitintensiv sein \cite{cascor}. Generell gilt für alle neuronalen Modelle, dass ihre 
Vorhersagen mit einer gewissen Fehlerrate behaftet sind und somit nicht vollständig korrekt sind \cite{EvoClassAndReg}.

Bei unzureichender Verfügbarkeit von Trainingsdaten wird TF eingesetzt. 
TF bezeichnet ein maschinelles Lernparadigma, bei dem bereits erworbenes Wissen aus einer Quellaufgabe (Source) genutzt 
wird, um die Lernleistung auf einer Zielaufgabe (Target) zu verbessern \cite{transfer_learning}. Dabei erlernt ein neuronales Netzwerk 
zunächst eine Aufgabe und überträgt anschließend relevante Erkenntnisse auf eine andere, meist verwandte Aufgabe \cite{phd_deep_cascade}.

Dieses Vorgehen lässt sich mit dem menschlichen Lernverhalten vergleichen, bei dem über Eselsbrücken bereits bestehendes Wissen herangezogen 
wird, um neues Wissen effizienter zu erwerben. Ein Netzwerk, das TF betreibt, vollzieht folglich einen kontextuellen Wechsel 
zwischen verschiedenen Lernaufgaben. Da-bei ergeben sich insbesondere drei zentrale Forschungsfragen:

\begin{enumerate}
    \item What to transfer
    \item How to transfer
    \item When to transfer
\end{enumerate}
\cite{survey_transfer}

Die erste zentrale Forschungsfrage im TF betrifft die Beschaffenheit und Auswahl der Source- und Target-Daten zwischen denen TF 
angewandt werden soll. 
% , also jener Datenbasis, auf der 
% das übertragbare Wissen – bildlich gesprochen die "Eselsbrücke" – aufgebaut wird.

Die zweite Fragestellung bezieht sich auf die konkrete Umsetzung des TF in der Praxis. Dabei wird untersucht, auf welche Weise neuronale Netze 
das Wissen aus den Source-Daten extrahieren und wie dieses Wissen im weiteren Verlauf verarbeitet und genutzt wird.

Die letzte Fragestellung bezieht sich darauf, wann der Übergang vom Lernen auf der Source- zur 
Target-Aufgabe erfolgen soll. In der Praxis wird dieser Übergang häufig durch Fehlermetriken gesteuert: Sobald das Modell einen be-stimmten 
Schwellenwert unterschreitet und die Fehlerquote als ausreichend nied-rig eingestuft wird, erfolgt der Transfer. Der optimale 
Zeitpunkt für diesen Wechsel kann jedoch experimentell bestimmt und variiert werden, um maximale Leistung zu erzielen. Es ist jedoch wichtig 
zu beachten, dass TF nicht zwangsläufig zu einer Leistungssteigerung führt. In bestimmten Fällen kann die Übertragung von Wissen 
sogar zu einer Verschlechterung der Modellleistung führen – ein Phänomen, das als Negative Transfer bezeichnet wird \cite{survey_transfer}.

TF setzt in mindestens einem von zwei Bereichen einen Wechsel voraus: der Domain oder dem Task. Die Domain umfasst Merkmale 
wie die Verteilung, Repräsentation oder den strukturellen Aufbau der Daten (z.B. Formate, Dimensionen), während der Task sich auf die konkrete 
Lernaufgabe bezieht, wie beispielsweise Bildklassifikation oder Segmentierung \cite{survey_transfer}.

In komplexeren Szenarien kann der Transfer nicht direkt von der Source- zur Target-Domain erfolgen. Stattdessen wird eine intermediäre, sogenannte 
Brückendomain eingeführt, über die der Wissenstransfer schrittweise erfolgt. Dieses Vorgehen wird als Bridge Transfer Learning 
bezeichnet und findet Anwendung, wenn die Unterschiede zwischen Source und Target zu groß sind oder die direkte Übertragung zu 
negativem Transfer führen würde \cite{bridge_transfer, survey_transfer}.

Während Deep Learning grundsätzlich in der Lage ist, komplexere Aufgaben zu lösen als sogenanntes Shallow Learning, ist es gleichzeitig durch 
einen erheb-lich höheren Trainingsaufwand gekennzeichnet. Wenn dieser hohe Rechenauf-wand nicht praktikabel ist, kommt Cascade Learning zum 
Einsatz \cite{cascor}.

Im klassischen Deep Transfer Learning wird in der Regel ein vortrainiertes Netzwerk genutzt, wobei der finale Feature-Vektor als Ausgangspunkt 
für den Transfer dient. Dies beruht auf der Annahme, dass in dieser Repräsentation die bedeutendsten Informationen verdichtet vorliegen – ein 
Ansatz, der als Feature Representation Transfer bezeichnet wird \cite{survey_transfer}.

Im Gegensatz dazu verfolgt das Cascade Learning einen iterativen Ansatz: In jeder Trainingsiteration wird der aktuell generierte Feature-Vektor 
in den Transferprozess einbezogen, um herauszufinden, welcher Repräsentationsvektor tatsächlich den größten Informationsgehalt besitzt. Dieser 
Ansatz führt in der Regel zu einer effizienteren und weniger komplexen Netzwerkstruktur im Vergleich zum klassischen Deep Transfer Learning 
\cite{phd_deep_cascade}.

Cascade-Netzwerke zeichnen sich dadurch aus, dass ihre Architektur nicht vor Beginn des Trainings vollständig definiert ist, sondern 
schrittweise während des Trainingsprozesses aufgebaut wird. Hierfür kommen sogenannte Constructive Algorithms zum Einsatz, deren wesentlicher 
Vorteil darin liegt, dass keine exakte Festlegung der Netzwerkgröße im Vorfeld erforderlich ist \cite{Constructive_Cascade}. Das Netzwerk wächst 
lediglich so weit, wie es zur Lösung der jeweiligen Aufgabe notwendig ist, wodurch eine übermäßige Modellkomplexität vermieden und \\gleichzeitig 
die Trainingszeit reduziert wird \cite{Constructive_Cascade, cascor}.

Ein prominentes Beispiel für einen solchen Ansatz ist der Cascade Correlation Algorithmus (CasCor). Bei diesem Verfahren wird das Netzwerk schritt-weise 
durch das Hinzufügen einzelner Neuronen (Perzeptrons) erweitert. Jedes neue Perzeptron wird so trainiert, dass es eine möglichst hohe 
Korrelation mit dem verbleibenden Fehler aufweist. Nach Abschluss des Trainings wird das betreffende Neuron fixiert, das heißt, seine 
Gewichtungen werden eingefroren und bleiben im weiteren Verlauf unverändert. Anschließend wird überprüft, ob das Netzwerk die angestrebte 
Fehlergrenze erreicht hat. Ist dies nicht der Fall, wird ein weiteres Perzeptron ergänzt und der Prozess wiederholt sich, bis die gewünschte 
Modellgüte erreicht ist \cite{cascor}.

Dieser iterative Aufbau sowie das Einfrieren bereits gelernter Gewichtungen ermöglichen ein effizientes Training, bei dem keine Backpropagation 
durch das gesamte Netzwerk erforderlich ist. Lediglich die Gewichte des neu hinzugefügten Neurons werden jeweils angepasst. Da jedes neue 
Neuron direkt mit dem Ausgang des Netzwerks verbunden ist, trägt es unmittelbar zur Fehlerkorrektur bei \cite{cascor}.

Ein verwandtes Konzept stellt das Direct Cascade Learning dar. Hierbei erfolgt keine Zwischenverarbeitung der Ausgaben bereits trainierter 
Neuronen. Stattdessen dient der Output eines Perzeptrons direkt als Input für das nachfolgende, wie es beispielsweise in den Architekturen 
Cascade QEF und Cascade LLM umgesetzt ist \cite{cascade_network_architectures,cascade_llm_networks}.

Obwohl TF bislang vorrangig mit klassischen tiefen Lernverfahren kombiniert wurde, existieren erste Ansätze, die TF mit 
Cascade Learning verknüpfen. Ein Beispiel hierfür ist das Cascade Transfer Learning (CTL), das CasCor mit TF 
kombiniert \cite{phd_deep_cascade}. Zwar blieb die Leistung von CTL leicht hinter der von klassischen Fine-Tuning-Verfahren zurück, jedoch 
zeichnete sich CTL durch eine deutlich geringere Speicheranforderung aus \cite{phd_deep_cascade}.

Neben CasCor existieren weitere Cascade-Lernalgorithmen – insbesondere solche, die dem Direct Cascade-Prinzip folgen –, die bislang noch nicht 
im Kontext von TF eingesetzt wurden und somit Potenzial für zukünftige Forschung bieten.


    \chapter{Methodik}  % 2.Methodik
    \section{Transferlernen}  % Irgendwo hier hin schreiben, dass es ohne Normalisierung ist!
    Transferlernen (TF) ist das Prinzip des Lernens über einer Eselsbrücke. 
Es gibt mehrere Varianten, wie TF verwendet werden kann. Nur wenn keine davon genutzt wird, wird nicht von TF gesprochen. 
Hier wird nur der Domain-Wechsel vorgestellt werden, da nur dieser genutzt wird. Ein Domain-Wechsel ist hier der Wechsel 
zwischen zwei verschiedenen Datensätzen, während diegleichen Netzarten genutzt wird. 
Dies wird Transductive Transferlernen\cite{survey_transfer} genannt. 
Das Wissen vom ersten Datensatz wird auf den zweiten übertragen. Der erste Datensatz ist dabei die Source, der Zweite das Target. 
Es gibt dabei drei Stellschrauben, bei denen nicht klar ist, was besser ist: What, How, When to Transfer \cite{survey_transfer}. 
Da es sowohl eine Klassifikation als auch eine Regression ausgetestet wird, werden jeweils zwei Source- und Targetdatensätze benötigt. 
Für Klassifikation wird die Source der Modified National Institute of Standards and Technology \cite{handwritten_digit} (MNIST) Datensatzes  
und der 
Street View House Numbers (SVHN) \cite{house_numbers} der Targetdatensatz sein. Beide müssen für das Transfer ein wenig 
verändert werden. Der MNIST wird von 28x28 Pixel auf 32x32 erweitert, während der SVHN von farbig auf schwarz-weiß verändert wird. 
Dies ist notwendig, da beide Datensätze als Input denselben Shape, also diegleichen Dimensionalitäten vorweisen, haben müssen. 
Bei der Regression ist der Sourcedatensatz der Boston Housing Prices (Bost) \cite{Boston_housing} und der Targetdatensatz der 
California Housing Prices (Cali) \cite{California_housing}. 

Beide Datensätze müssen stark reduziert werden. Von den Acht beziehungsweise Dreizehn Spalten bleiben nur Drei übrig. Dies hat den Grund, dass 
nur Spalten als Sinnvoll geachtet werden, die ein passendes gegenüber haben. Der Bost-Datensatz hat allerdings ein ethnisches Problem, da 
dieser eine Spalte enthält, die diskriminierend ist. Diese wird entfernt. 
Die einzigen Spalten des Bost-Datensatzes, die übrig bleiben sind: RM, AGE, LSTAT. RM ist die durchschnittliche Zimmeranzahl pro Wohnung, AGE 
ist die Anzahl der Häuser, die vor 1940 bewohnt wurden und LSTAT ist der prozentuale Anteil der Bevölkerung mit niedrigerem Status. 
Der Datensatz Cali behält nur die Spalten MedInc und HouseAge. MedInc ist das durchschnittliche Einkommen des Häuserblocks und HouseAge das 
durchschnittliche Alter. Aus den Spalten AveRooms und Households wird die durchschnittliche Anzahl von Zimmers pro Haushalt berechnet. AveRooms 
ist dabei die durchschnittliche Anzahl an Räumen innerhalb eines Häuserblocks, während Households die Anzahl der Haushalte innerhalb des 
Häuserblocks ist. Dadurch ist die berechnete Spalte zu der RM-Spalte von Bost passend. 
Da LSTAT und MedInc wahrscheinlich abhängig sind, da es vermutet wird, dass diejenigen Menschen, die einen niedrigeren Status vorweisen, weniger 
Einnahmen haben. Deshalb dürfte es über diese beiden Spalten möglich zu sein TF zu nutzen. Allerdings sind sie zueinander antiproportional, 
weshalb die LSTAT Spalte invertiert wird, damit es zur Proportionalität kommt. Komplexer ist die Berechnung des Alters der Häuser, da 
AGE nur die Anzahl der Häuser, die vor 1940 gebaut wurden, beeinhaltet, aber HouseAge das durchschnittliche Alter des Häuserblocks ist. 
Die Maximalanzahl der betrachteten Häuser im Bost-Datensatz ist einhundert und das Alter der Häuser vor 1940 ist 85, wenn man auf 2025 rechnet. 
Dadurch kann AGE auf die Art von HouseAge mit folgender Formel umgerechnet werden: 
\begin{equation}
    \frac{AGE * 85}{Maximalanzahl}
\end{equation}
Dadurch sind alle Source- und Targetdatensätze zueinander kompatibel. Damit ist ausreichend geklärt, mit was TF verwendet wird. 

Die nächste Frage, die geklärt werden muss ist das How to transfer. Dies wird jeweils ohne Veränderung der Weights der Netze gemacht. 
Es wird das neuronale Netz zuerst auf dem Sourcedatensatz trainiert und dann ohne irgendetwas zu tun auf den Targetdatensatz gewechselt, 
welcher auf demselbem Netz oder einem gleichen Netz wie zuvor ist. Wenn es dasselbe Netz ist, dann verändert sich nur aus welchem 
Datensatz der Input kommt, was bei Deep Cascade ist. Während bei dem gleichen Netz der Input immer vergrößert wird und das TF über diese 
Vergrößerung passiert, was bei Direct Cascade ist. 

Wann TF sinnvoll ist, ist nicht klar, weshalb es mal mit früherem und späteren TF probiert wird. 

    \section{Kaskadierung}
    Im Folgenden wird untersucht, welche Auswirkungen auftreten, wenn auf das Kaskadieren verzichtet wird. Da ein zufriedenstellendes Ergebnis 
nur ohne Verwendung von TF zu erwarten ist, erfolgt das Training direkt auf dem Target-Datensatz. Die Architektur der 
vollständigen Netzwerke wird so angepasst, dass die Gesamtanzahl der Hidden Layer der Summe der Hidden Layer aller einzelnen Netzwerke im 
Direct-Cascade-Verfahren entspricht. Dabei wird die Gesamtanzahl der Trainings-Epochen beibehalten, um eine vergleichbare Trainingsdauer 
sicherzustellen.

\begin{figure}[htpb]
    \includegraphics[height=5cm]{../../Plots/ba_plots/classnocascade/1dc.png}
    \includegraphics[height=5cm]{../../Plots/ba_plots/classnocascade/cod.png}
    \caption{\label{fig:nocascade} 
    \small{Die dargestellten Ergebnisse zeigen Testläufe ohne Kaskadierung. Konkret sind links die Resultate für das 1DC-Netzwerk (1DC:Comp/732//30) 
    und rechts für das COD-Netzwerk (COD:Comp/732//30) dargestellt. Es ist ersichtlich, dass eines der Modelle in einem lokalen Maximum stecken bleibt. 
    Zudem lässt sich aus den Ergebnissen ableiten, welcher maximale Accuracy-Wert unter der gegebenen, begrenzten Datenmenge realistischerweise 
    erreicht werden kann. Dieser Wert wird ausschließlich durch den Einsatz der vollständigen, hier beschriebenen Netzwerkarchitekturen erzielt.}}
\end{figure}

In Abbildung \ref{fig:nocascade} fällt auf, dass während der meisten Epochen kein Lerneffekt eintritt. Zudem ist Overfitting 
erkennbar – ein zu erwartendes Verhalten angesichts der geringen Menge an Trainingsdaten. Obwohl in diesem Experiment kein TF eingesetzt wurde, 
zeigt einer der beiden Plots in der Mitte einen plötzlichen Anstieg der Accuracy. Dieser Anstieg wurde durch eine minimale Verbesserung des 
Trainingswerts bei gleichzeitig minimaler Verschlechterung des Validierungswerts ausgelöst; beide Änderungen lagen im Bereich von 
Zehntelprozenten. Dies deutet auf das Erreichen eines lokalen Maximums im Trainingsverlauf hin.

Im Vergleich dazu bleibt das andere Netzwerk dauerhaft auf dem Niveau dieses lokalen Maximums – beide Resultate zeigen exakt identische Werte. 
Aus Abbildung \ref{fig:nocascade} lässt sich zudem ableiten, dass unter den gegebenen Bedingungen eine maximale Accuracy von etwa 40\% erreichbar 
ist. Dieser Wert stellt das globale Maximum dar, da er die bestmögliche Performanz auf den Trainingsdaten widerspiegelt.

Weder das reine Kaskadieren noch die Kombination aus Kaskadierung und TF erreichen vergleichbare Ergebnisse – beide Varianten 
erreichen lediglich eine maximale Accuracy von etwa 20\% und damit nur etwa die Hälfte der möglichen Leistung.

Diese Beobachtungen legen nahe, dass die Ursache für die stark eingeschränkte Klassifikationsleistung im Direct-Cascade-Verfahren mit 
TF bereits in der Art der Kaskadierung selbst zu suchen ist. Mögliche Gründe hierfür könnten in der wiederholten Anwendung der 
Categorical-Crossentropy-Verlustfunktion liegen, die sich gegenseitig negativ beeinflussen könnte. Alternativ könnte auch die 
Softmax-Aktivierungsfunktion oder das konkrete Vorgehen beim Kaskadieren die Ursache darstellen.

Letztlich erweist sich der Einsatz von TF in Kombination mit dem Direct-Cascade-Verfahren unter Verwendung der in dieser Arbeit 
genutzten Datensätze und der beschriebenen Augmented Vectors als nicht zielführend, da die erzielten Accuracy-Werte selbst bei umfangreichen 
Trainingsdaten 60\% nicht überschreiten und im Vergleich zu einem vollständig trainierten Netzwerk signifikant schlechter ausfallen.

    \subsection{Deep Cascade}
    Hier wird die Variante des Deep Cascade vorgestellt. 
Die Deep Cascade Netze werden iterativ während dem Training aufgebaut. Es bleibt dabei ein einziges Netz. Es wird zuerst 
definiert, welcher Optimizer und welcher Loss in dem Netz genutzt wird. 

\begin{figure}[htpb]
    \includegraphics[height=10cm]{../../Graphiken/deepcascade_2.png}
    \caption{\label{fig:deepcascade} Vorstellung Deep Cascade Aufbau}
\end{figure}

Sobald dies beides gemacht wurde, wird im Netz das erste Layer definiert. Dieses wird ergänzt durch ein Output Layer und dann trainiert. 
Wenn das Training beendet wird, wird das Output Layer gelöscht und ein neues Layer hinzugefügt, wie es in Figure 2.1 gezeigt wird. Zudem wird 
das gerade trainierte Layer gefreezt, damit dieses keine weiteren Aktualisierungen mehr bekommt. 
Dann wiederholt sich das Training, das Löschen, das Freezing und weitere Hinzufügen von Layern. 
An einer beliebigen Stelle kann TF gemacht werden, indem, statt in der Trainingsphase den Sourcedatensatz zu nutzen, der Targetdatensatz 
genutzt wird. 

% Sollte ich nicht vorher Kaskadierung erklären? Oder geht das hier? Hier könnte ich auch Graphen bauen. Ist glaube ich sogar besser, wenn 
% ich es tue...

    \subsection{Direct Cascade}
    In diesem Abschnitt wird die Kaskadierungsvariante des Direct Cascade Netzwerks vorgestellt. Das Netzwerk ist hierbei vollständig vorab 
definiert und besteht aus einem einzelnen Hidden Layer sowie einem Output Layer. Die Gesamtstruktur setzt sich aus mehreren identischen 
Subnetzwerken zusammen, zwischen denen während des Trainings ein Wissenstransfer stattfindet.

\begin{figure}[htpb]
    \centering
    \includegraphics[height=10cm]{../../Graphiken/direct_cascade.png}
    \caption{\label{fig:directcascade} 
    \small{Hier wird das Direct Cascade Verfahren dargestellt. Dieses Verfahren verwendet mehrere einzelne Netzwerke 
    (hier als Modelle bezeichnet), die jeweils nur wenige Hidden Layer aufweisen, in der Regel lediglich einen. Nach der 
    Initialisierung wird jedes Modell einmal ohne weiteres Training angewendet, und dessen Ausgangssignal wird mit dem ursprünglichen 
    Eingabesignal kombiniert. Diese Kombination bildet den neuen Eingabedatensatz für das nachfolgende Modell. Durch diese sukzessive 
    Verknüpfung der Ausgaben mit den Eingaben kann das Verfahren eine Wissensweitergabe und -integration zwischen den einzelnen Modellen 
    realisieren.}}
\end{figure}

Der Ablauf beginnt, wie in Abbildung \ref{fig:directcascade} dargestellt, mit dem vorbereiteten Quell-Datensatz (Sourcedatensatz), der als 
Eingabe in die erste Instanz des Netzwerks gegeben wird. Diese Netzwerkinstanz wird anschließend trainiert. Nach Abschluss des Trainings 
erfolgt eine einmalige Anwendung des fixierten Netzwerks, deren Ergebnis die Vorhersage (Prediction) darstellt. Diese Prediction wird mit 
dem ursprünglichen Eingabesignal desselben Netzwerks kombiniert, wodurch ein sogenannter Augmented Vector entsteht. Die genaue Bildung dieses 
Augmented Vectors variiert dabei leicht je nach spezifischer Implementierung des jeweiligen Direct Cascade Netzwerks und wird an späterer 
Stelle detaillierter erläutert.

Der Augmented Vector dient als Input für die nächste Instanz des Netzwerks. Dieser Zyklus aus Netzwerkinstanz, Training, Prediction und 
Augmented Vector Berechnung wird beliebig oft wiederholt. Durch die Einbindung der Prediction in den Augmented Vector kann das Netzwerk 
Wissen aus den zuvor trainierten Instanzen übernehmen und integrieren.

Eine Transfer-Learning-Phase (TF) kann jederzeit innerhalb eines Trainingsschritts durchgeführt werden, indem anstelle des Quell-Datensatzes 
ein Ziel-Datensatz (Targetdatensatz) als Input verwendet wird. Dabei können beliebig viele Netzwerkinstanzen vor und nach der 
Transfer-Learning-Phase genutzt werden. Der einzige Unterschied besteht darin, dass der Augmented Vector mit jedem weiteren Netzwerk etwas 
größer wird, da er sowohl das Wissen aller bisher trainierten Netzwerke als auch die ursprünglichen Eingabedaten enthält.

Für die Implementierung bedeutet dies, dass von Beginn an sowohl der Quell- als auch der Ziel-Datensatz in das feste Netzwerk eingespeist 
werden müssen. Dies ist notwendig, um die Prediction auf dem Ziel-Datensatz – die während der Trainingsphase mit dem Quell-Datensatz generiert 
wurde – im Augmented Vector zu integrieren. Somit wird sichergestellt, dass die während des Trainings auf dem Quell-Datensatz gelernten 
Netzwerkkomponenten auch bei der Anpassung an den Ziel-Datensatz berücksichtigt werden.

    \section{Setup}
    Alle Test wurden auf einem Erazer Gaming Notebook P15601 unter Windows 10 durchgeführt.
Die Neuronalen Netze laufen dabei ausschließlich auf 
der CPU und wurden nur trainiert, während der Rechner am Stromnetz angeschlossen war. 
Dieser Rechner hat einen intel Core i5 der neunten Generation mit 4 Kernen auf 8 
logischen Prozessoren. Die Betriebsgeschwindigkeit liegt bei 2,4-5,1 GHz und die 
RAM-Größe liegt bei 15,8 GB bei einer Geschwindigkeit von 2667 MHz. 

Es wurde mit PyCharm und der library Keras programmiert. Die Texte sind mit BibTex 
erstellt worden und die Plots mit der MatPlotLib library.

Dabei sind MNIST und Bost die Sourcedaten und SVHN und Cali die Targetdaten. Jeder Targetdatensatz 
wird händisch verkleinert, da es darum geht, nicht genügend Daten für sie allein zu haben und deshalb eine andere 
Methode genutzt werden muss.

    % \section{Durchführungen}
    % Es wurden sowohl für Klassifikation als auch für Regression drei verschiedene Ansätze der Kaskadierung genutzt. 
Ebenso wurde Direct TF mit Domainwechsel durchgeführt. 

Die drei Ansätze sind Deep Cascade, Direct Cascade und eine Kaskadierung von einem Netz im Netz mit mehreren Inputs.

Bei Deep Cascade wird ein Netz Layer für Layer aufgebaut und jedes Layer einzelnd trainiert und gefreezt. 
Bei Direct Cascade werden ganze Netze trainiert und dessen Prediction als zusätzlichen Input für das nächste Netz zu nutzen.
Bei der dritten Variante wird ein Netz trainiert, dann auf einem Teilnetz davon die Prediction gemacht, um mit dieser das 
ganze Netz außer das vorher erwähnte Teilnetz zu trainieren. 
Nur Deep Cascade wird genauer betrachtet, denn die beiden anderen Ansätze sind nur zum Vergleichen da.

Es wurde bei jedem gleichbleibende Epochenanzahlen, zufällige und von einer Metrik abhängige durchgeführt. 

Bei allen Neuronalen Netzen wurden die dafür benötigten Daten in ein Trainings-, Validation- und Testdatensatz aufgeteilt. 
Ebenfalls wurde MNIST auf 32x32 erweitert, sowie SVHN in graue Bilder mit einem Channel verändert. 
Es wurde erweitert, da keine Daten unnötig verloren werden sollten. Die Reduzierung von SVHN liegt daran, dass MNIST nur 
Schwarz-Weiß-Bilder sind und es nicht möglich ist, dies in bunte Bilder zu verändern.
Die Veränderungen der Datensätze kommt daher, dass sie technisch gleich aussehen müssen, da sie sonst nicht als Input 
desselben Netzes genutzt werden können.

Für die Regressionsnetze müssen alle Spalten weggenommen werden, die kein Gegenüber im anderen Datensatz besitzen. 
Somit fielen die Spalten: Verbrechensrate, Anteil der Wohngebiete über 25000 Fuß, Nicht-Einzelhandelanteil der Gewerbeflächen, 
Flussgrundstück, Stickoxidkonzentration, Entfernung zu Arbeitsvermittlungszentren, Erreichbarkeit von Autobahnen, 
Vollwertsteuersatz, Schüler-Lehrer-Verhältnis und die Anzahl von Schwarzen im Bosten weg, während im California die 
folgenden Spalten wegfielen: Längengrad, Breitengrad, Schlafzimmer und Bevölkerung. Aus der Gesamtanzahl der Räume und der Haushalte 
wird die durchschnittliche Anzahl an Räumen pro Wohnung errechnet.
Diese Spalten haben alle keinen Gegenüber im anderen Datensatz und eine Spalte ist aus ethnischen Gründen nicht nutzbar, was daran liegt, 
dass der Datensatz aus den Siebzigern stammt.
Übrig blieben von Boston nur noch die durchschnittliche Anzahl der Räume pro Wohnung, die Menge der Häuser, die vor 1940 
erbaut worden sind und der prozentuale Anteil der Bevölkerung mit niedrigem Status.
Bei California blieben das Errechnete und das Hausalter, sowie das durchschnittliche Einkommen. 
Die Anzahl der Räume pro Wohnung passen offensichtlich zueinander, während der prozentuale Anteil der Bevölkerung mit niedrigem Status 
antiproportional zu dem durchschnittlichen Einkommen ist. Dies wird vorher zu einer Proportionalität umgewandelt.
Als etwas komplizierter erweist sich das Alter. Mit Prozentrechnung kann man aber das ungefäre Alter der Häuser aus dem 
Boston Datensatz abschätzen. Da immer eine Häuseranzahl von einhundert betrachtet wird, ist dies die Gesamtmenge und folgende Formel 
löst das Problem: 
\begin{equation}
    \frac{Hausanzahl * Hausalter}{Gesamtmenge}
\end{equation}
Die Hausanzahl ist hier die Menge der Häuser, die vor 1940 erbaut worden sind. Das Hausalter bezieht sich auf das Alter der eben 
erwähnten Häuser und ist auf Heute angedacht; sind also 85 Jahre.

Die Hypothese war, dass man mit TF bei zu wenig Daten eine verhältnismäßig gute Performanz der Netze erwarten kann, sowie, 
dass durch einen Kaskadierungsansatz das Training der Netze sehr kurz ist.

Generell wird zuerst eine Weile auf dem Sourcedatensatz trainiert und dann auf den Targetdatensatz gewechselt ohne die 
bisherigen Netze zu verändern. Bis auf den Direct Cascade Ansatz werden auch die Inputs während des ganzen Prozesses nicht 
verändert.
Bei Direct Cascade werden die Inputs immer größer, denn die Prediction des vorherigen Netzes wird zum Input hinzugefügt.

    \section{Metrik}
    Es wurden drei Metriken erstellt. 
Die Accuracy- (ACCM), Loss- (LM) und MAE-Metrik (MAEM). MAE heißt dabei Mean absolute Error. 
Alle drei Metriken sind für Early Stopping und entscheiden, wieviele Epochen genutzt werden. 
Die Accuracy-Metrik bricht immer dann ab, wenn die Validation-Accuracy mindestens um 10\% 
schlechter ist als die Trainingsaccuracy, da dann in dem Netzwerk Overfitting herrscht.

Die Loss- und die MAE-Metrik brechen beide dann ab, wenn der Validation-Wert der aktuellen 
Epoche schlechter ist als in der Epoche davor. Dies hat zur Folge, dass die Netze in lokale 
Minima hineinlaufen und nicht wieder herauskommen. Dabei unterliegt die Anzahl der Netze für 
das Direct Cascade keiner Metrik.

% Evtl eine Metrik bauen, die über einen Max-Value für ACC geht und dann den Abbruch macht, wenn es besser 
% wird oder schlechter als dieser Wert(Muss rel. enger Bereich sein)
% Ebenso für Min-Value für Loss und MAE

    \section{Liste der Tests}
    
Liste aller hier vorkommenden Netzen mit ihren Kürzeln:

\begin{enumerate}
    \item ConvMaxPool (CMP)
    \item 1DConv (1DC)
    \item 2DConv (2DC)
    \item ClassOneDense (COD)
    \item RegressionTwo (Regr2)
    \item OneLayer (1Lay)
\end{enumerate}

Davon sind ConvMaxPool und RegressionTwo Deep Cascade Netzwerke, während alle anderen Direct Cascade Netzwerke sind. 
Ebenso sind nur RegressionTwo und OneLayer Regressionsnetze, während der Rest Klassifikationsnetze sind. 

Alle Netze werden mit dem Adam-Optimizer mit der Lernrate 1e-3 gelernt. Klassifikationsnetze haben als Loss den 
CategoricalCrossEntropy und Softmax als Aktivierungsfunktion, während die Regressionsnetze MeanSquaredError und Linear als 
Aktivierungsfunktion vorweisen. 

Mit allen Direct Cascade Netzwerken wurden zusätzlich Early Stopping Metriken durchgeführt mit MAEM, LM und ACCM. 

Für alle Klassifikationsnetze gilt, dass sie mit fünf verschiedenen Größen des Targetdatensatzes trainiert wurden. Die Ausnahme ist das 
2DC-Netzwerk, welches nur mit sehr wenigen Source- und Targetdaten trainiert werden kann, da es technisch auf derselben Hardware mit mehr 
Daten unmöglich ist. 

Bei den Regressionsnetzen wird jeweils einmal mit vielen und wenigen Targetdaten trainiert. 

Es wurde mit allen Netzwerken ein Vergleich sowohl zwischen mit TF und ohne als auch zwischen ohne TF und Kompletten angefertigt. 
Komplett heißt hier, dass es ein Netzwerk ohne TF und ohne Kaskadierung ist und dieses deshalb in einem komplett trainiert wird. 

Mit allen Netzwerken wurde der Zeitpunkt für das TF frei ausgetestet. 

Alle Direct Cascade Netzwerke haben jeweils nur ein Hidden Layer. In manchen Fällen sind sie noch mit einem Hilfslayer, um den Wechsel 
zwischen Filterlayern und Linearlayern zu bewerkstelligen. 

Für alle Netzwerke wurde derselbe Seed für die Initialisierung der Weights genutzt. 

In Tabelle 2.1 sind die Tests bezüglich Klassifikation und in Tabelle 2.2 die für die Regression. 
In beiden Tabellen sind die Tests, die sich mit der Zeitdauer befassen mit der Endung Time. Dabei gilt, dass CasTF Kaskadierung mit TF, Cas allein 
Kaskadierung ohne TF und Comp bedeutet, dass es weder TF noch Kaskadierung gab. ACCM, LM und MAEM sind die Tests bezüglich der Early-Stopping 
Metriken. 
Vor dem ersten Schrägstrich steht, wann TF gemacht wurde, welches mit TF im Eintrag gekennzeichnet ist. Wenn kein TF gemacht wurde, ist dieser 
erste Bereich nicht existent. Dahinter steht die Datenmenge 
des Targetdatensatzes und danach die Menge an Epochen pro Trainingsiteration. Wenn es noch etwas viertes gibt, dann zeigt dieses an, wieviele 
Epochen in Zehnern es insgesamt gab. 

\begin{table}[!ht]
    \centering
    \begin{tabular}{l|l|l|l}
        \textbf{CMP} & \textbf{COD} & \textbf{1DC} & \textbf{2DC} \\
        \hline
        TF0/732/10 & CasTFTime & CasTFTime & CasTFTime \\
        TF1/732/10 & CasTime & CasTime & CasTime \\
        TF2/732/10 & CompTime & CompTime & CompTime \\
        TF3/732/10 & TF2/732/10 & TF2/732/10 & TF2/732/10 \\
        TF4/732/10 & TF2/7k/10 & TF2/7k/10 & \\
        TF5/732/10 & TF2/21k/10 & TF2/21k/10 & \\
        732/10 & TF2/36k/10 & TF2/36k/10 & \\
        CasTFTime & TF2/51k/10 & TF2/51k/10 & \\
        CasTime & TF10/732/10/30 & TF10/732/10/30 & \\
        CompTime & 732/10/30 & 732/10/30 & \\
        TF2/7k/10 & Comp/732//30 & Comp/732//30 & \\
        TF2/21k/10 & ACCM/732/10 & ACCM/732/10 & \\
        TF2/36k/10 & LM/732/10 & LM/732/10 & \\
        TF2/51k/10 & & & \\
        & & & \\
    \end{tabular}
    \caption{\label{tab:classtests} Liste aller Klassifikationstests}
\end{table}

\begin{table}[!ht]
    \centering
    \begin{tabular}{l|l}
        \textbf{Regr2} & \textbf{1Lay} \\
        \hline
        TF0/240/25 & CasTFTime \\
        TF1/240/25 & CasTime \\
        TF4/240/25 & CompTime \\
        CasTFTime & TF11/8k/10/8 \\
        CasTime & 8k/10/8 \\
        CompTime & Comp/8k/10/8 \\
        TF3/8k/10/8 & TF11/240/10/20 \\
        8k/10/8 & 240/10 \\
        Comp/8k/8 & Comp/240/8 \\
        TF3/240/8 & MAEM/240/10 \\
        240/8 & LM/240/10 \\
        Comp/240/8 & TF4/206/10/8/ts \\
        TF4/206/10/8/ts & 206/10/8/ts \\
        206/10/8/ts & Comp/206/10/8/ts \\
        Comp/206/10/8/ts &
    \end{tabular}
    \caption{\label{tab:regrtests} Liste aller Regressionstests}
\end{table}

Eine Referenz zu einer dieser Listen ist CMP:TF0/732/10. Diese bedeutet, dass es um den Test mit der Kennung TF0/732/10 des CMP-Netzwerkes geht. 
Bei diesem gibt es die Besonderheit, dass TF nach dem ersten Layer gemacht wird, dieses Layer jedoch im 
Gegensatz zum restlichen Netzwerk nur mit einer Epoche trainiert wird.
Selbiges gilt für Regr2:TF0/240/25.

Die Tests mit der Endung ts sind diejenigen, die einen explizit großen Testdatensatz haben. 


    \chapter{Allgemeine Resultate}  % 3.Allgemeines
    \section{Ploterklärung}
    In diesem Unterkapitel werden alle Arten der Plots einmal vorgestellt und auf alle Eigenheiten eingegangen, damit diese verstanden werden. 

Es wird hier sowohl auf die Achsenbeschriftung als auch auf die Texte innerhalb der Plots eingegangen. 

\begin{figure}[htpb]
    \includegraphics[height=5cm]{../../Plots/ba_plots/convmaxpool/2TFtr.png}
    \includegraphics[height=5cm]{../../Plots/ba_plots/regr2/regr2train.png}
    \caption{\label{fig:ploterkl} 
    \small{Es sind hier die Plots von zwei Deep Cascade Netzen. Links für die Klassifikation und rechts für die Regression. 
    Zu sehen sind die Tests: CMP:TF2/732/10 und Regr2:TF4/240/25. Beides dient als Beispielplot dafür, wie diese mit TF für 
    gewöhnlich aussehen. }}
\end{figure}

Dazu wird Figure \ref{fig:ploterkl} betrachtet. In beiden Teilen stehen drei Zeilen Text auf die nun einzelnd eingegangen wird. Die Erste sagt aus, 
wieviele Epochen pro Layer oder Netzwerk trainiert worden sind. Die Zweite beschreibt wieviele Datensamples des Trainingssets des 
Targetdatensatzes im Training nach TF genutzt worden sind und die dritte Zeile zeigt die gesamte Trainingsdauer in Sekunden an. 

Wenn es um die Accuracy geht, was bei Klassifikation der Fall ist, dann steht ACC auf der senkrechten Achse und beim Funktionsnamen dabei. 
Die senkrechte Achse ist dann bei 100\%, wenn sie bei 1 ist. In Figure \ref{fig:ploterkl} ist links ein Beispielplot für diesen Fall. 

Für die Regression, geht es um den MAE. Dies steht wiederum in den Namen der Funktionen und der senkrechten Achse. Diese Achse ist in 1000\$ pro 
Einheit. Dabei ist es besser, je geringer der Wert ist. 

    \section{ConvMaxPool}
    Anhand des ConvMaxPool-Netzwerks werden alle allgemeinen Resultate und Auffälligkeiten beschrieben. 
Dies ist ein Deep Cascade Classification Netzwerk und wird deshalb iterativ aufgebaut. 
Das Netz ist ein Convolution-Network mit Padding, sodass die Dimensionen während der Convolution-Layer nicht verringert werden. 
Es wird die Aktivierungsfunktion relu genutzt. 

\begin{figure}[htpb]
    \centering
    \includegraphics[height=5cm]{../../Graphiken/convmaxpool.png}
    \caption{\label{fig:convmaxpool} 
    \small{Diese Layer in genau der Reihenfolge, wie hier von oben nach unten, stecken hinter dem CMP-Netzwerk.}}
\end{figure}

\iffalse
\begin{enumerate}
    \item Conv2D (32, (3, 3), same, relu)
    \item MaxPool2D (2, 2)
    \item Conv2D (64, (3, 3), same, relu)
    \item MaxPool2D (2, 2)
    \item Flatten
    \item Dense (10, softmax)
\end{enumerate}
\fi

Alle Layer des ConvMaxPool Netzwerks sind in Figure 3.2 in korrekter Reihenfolge zu sehen. Dabei ist die erste Zahl eines Convolution Layer 
die Anzahl der genutzten Filter, während das folgende Tuple die Kerngröße beschreibt. Ebenfalls steht die Kerngröße bei den MaxPool Layern dort. 
Das Flatten- und das Linear-Layer sind der Output-Block. Das Linear-Layer benötigt zehn Nodes, da es zehn Klassen gibt. Jedes Hidden Layer wird 
mit zehn Epochen trainiert. Es gibt keine Early-Stopping Metrik und es wird derselbe Seed für alle Tests genutzt. 

    \subsection{Veränderungen bei TF}
    Hier wird etwas sehr offensichtliches betrachtet. Dies passiert jedes Mal, wenn TF verwendet wird. 
Der Graph, der die Trainings- und Validationdaten nutzt, hat immer einen Einbruch in der Performanz an der Stelle an 
der TF gemacht wird. Dies ist in der Figure 3.3 deutlich bei Epoche zwanzig zu sehen.

\begin{figure}[htpb]
    \centering
    \includegraphics[height=5cm]{../../Plots/ba_plots/convmaxpool/convmaxpooltrain.png}
    \caption{\label{fig:convmaxpooltrain} 
    \small{Hier zu sehen ist CMP:TF2/732/10. Der Hauptaugenmerk liegt hier bei Epoche 20, denn zu diesem Zeitpunkt wurde TF angewendet. 
    Es kommt zum Einbruch der Performanz.}}
\end{figure}

Dieser Einbruch passiert jedes Mal nach TF. Dies liegt daran, dass das Netz bisher die Targetdaten noch nie gesehen hat und bisher 
auf eine andere Domain mit dem Sourcedatensatz trainiert hat. Das Netz kennt nur das Wissen aus dem Sourcedatensatz und kann nur dieses 
anwenden. Wenn man aber das Testset, welches nur über die Targetdaten geht auf das ganze Netzwerk betrachtet, kommt Figure 3.4 heraus. 

\begin{figure}[htpb]
    \centering
    \includegraphics[height=5cm]{../../Plots/ba_plots/convmaxpool/convmaxpooltest.png}
    \caption{\label{fig:convmaxpooltest} 
    \small{Dies ist der Testdatenplot von CMP:TF2/732/10. Dieser enthält die Targetdaten, die das Netz nicht 
    während dem Training sieht. Immer nachdem ein Layer fertig trainiert wurde, werden einmal die Testdaten evaluiert. Deshalb ist der Punkt 
    an dem TF gemacht wird, hier bei 2 Networks. Performanz wird zu dem Zeitpunkt besser.}}
\end{figure}

Der Wechsel ist hierbei bei Netzwerk 2. Es ist eindeutig zu erkennen, dass es nach TF besser wird. Dies hat den Grund, dass das Netzwerk 
ab diesem Zeitpunkt auf den Trainingsdaten trainiert, die zum Testdatenset passen, da dieses seine Daten nur aus dem Targetdatensatz bezieht. 

    % \subsection{Stabilisierung}
    % Bei fast allen Varianten des TF mit Kaskadennetzwerken kommt es nach TF zu einer Stabilisierung der Performanz des Netzes. 
Diese wird hier genauer vorgestellt und erklärt. 

\begin{figure}[htpb]
    \includegraphics[height=5cm]{../../Plots/ba_plots/convmaxpool/convmaxpooltrainbig.png}
    \includegraphics[height=5cm]{../../Plots/ba_plots/convmaxpool/convmaxpooltestbig.png}
    \caption{\label{fig:cmptrbig} Stabilisierung der Performanz}
\end{figure}

Es ist in Figure 3.4 links deutlich zu erkennen, dass es bereits nach zehn Epochen nach TF zu einer Stabilisierung der Accuracy kommt. 
Selbst bei späteren Epochen mit mehr Layern verbessert sich diese nur noch wenig. Dies bestätigt auch der Plot von den Testdaten, dessen 
Werte schlagartig nach TF besser wird, aber danach nur noch wenig steigen. 

Dasselbe verhält sich auch bei Regressionsnetzwerken. Nach TF kommt es schnell dazu, dass sich der Mean Absolute Error nicht weiter verringert. 
Ein Beispielplot ist in Figure 3.5 zu sehen. 

\begin{figure}[htpb]
    \includegraphics[height=5cm]{../../Plots/ba_plots/regr2/regr2train.png}
    \caption{\label{fig:regr2tr} Stabilisierung bei Regression}
\end{figure}

Diese Stabilisierung der Performanz ist sehr schnell nach TF. Dies fällt dann auf, wenn direkt auf dem Targetdatensatz gelernt wird. 
% Targetdatensatz only Plot -> Diesen auch für das nächste Unterkapitel nutzen. 

\begin{figure}[htpb]
    \includegraphics[height=5cm]{../../Plots/ba_plots/convmaxpool/woconvmaxpoolbig.png}
    \caption{\label{fig:cmpwotf} Ohne TF}
\end{figure}

Wie in Figure 3.6 zu sehen, ist diese Stabilisierung ohne TF nach gleich vielen Epochen bezüglich des Datensatzes und es bleibt auch bei diesem 
Wert. Dies liegt an der händisch definierten Learningrate, die die Deep Cascade Netze haben. Die Learningrate für Klassifikation ist hierbei: 

\begin{equation}
    (10^{-4} * 10)^{\frac{10}{epoch + 10}}
\end{equation}

Die zweite Learningrate ist für die Regressionsnetzwerke. 

\begin{equation}
    (10^{-4} * 100)^{\frac{10}{epoch + 10}}
\end{equation}

Beide Learningraten haben die Eigenschaft, dass sie immer kleiner werden, wodurch diese Stabilisierung kommt. 

% Hier irgendwo die Learningrate hinzufügen



    \subsection{Overfitting auf Sourcedatensatz}
    Wenn es unterschiedlich lang auf dem Sourcedatensatz trainiert wird, fällt auf, dass das Netz unterschiedlich gut auf dem Targetdatensatz ist. 
Da es sowieso ausgestestet werden muss, wann TF genutzt wird, wird nun das ConvMaxPool-Netzwerk genommen und nach jedem Layer TF angewandt. 
Das Ergebnis davon ist in Figure 3.5 zu sehen. 

\begin{figure}[htpb]
    \includegraphics[height=5cm]{../../Plots/ba_plots/convmaxpool/wotr.png}
    \includegraphics[height=5cm]{../../Plots/ba_plots/convmaxpool/1TFtr.png}
    \includegraphics[height=5cm]{../../Plots/ba_plots/convmaxpool/2TFtr.png}
    \includegraphics[height=5cm]{../../Plots/ba_plots/convmaxpool/epochTFtr.png}
    % \includegraphics[height=5cm]{../../Plots/ba_plots/convmaxpool/4TFtr.png}
    \caption{\label{fig:layertf} TF bei unterschiedlichen Layern}
\end{figure}

Auffällig ist es, dass hier die beste Performanz ohne TF ist. Bereits nach nur einer Epoche im ersten Layer, welches auf dem Sourcedatensatz 
trainiert wird, bricht die Accuracy ein. Dies zeigt, dass TF bei Klassifikation und Deep Cascade Netzwerken sinnfrei ist. Das 
Trainingsset der Trainingsdaten ist bei TF nie auch nur annähernd an den Bereich kommt, in dem es bei ohne TF ist. Daraus folgt, dass es 
bereits zu Overfitting auf dem Sourcedatensatz gekommen ist. Dadurch kann nicht mehr so gut auf dem Targetdatensatz gelernt werden. Dieses 
Overfitting passiert sogar bereits, wenn nur eine Epoche auf dem Sourcedatensatz gelernt wird, was die letzte Graphik von Figure 3.4 zeigt. 
Ebenso ist es offensichtlich, dass es bei jedem Graph zu Overfitting auf dem Trainingsset des Targetdatensatzes kam, da dieser um 60\% höhere 
Accuracy als das Validationset und dem Testdatenset vorweist. 

Bei einem Regressionnetzwerk, wie dem Deep Cascade Netzwerk RegressionTwo kommt es, wie in Figure 3.6 zu sehen, nicht so schnell zu Overfitting. 
Weder auf dem Sourcedatensatz noch auf dem Targetdatensatz. 

\begin{figure}[htpb]
    \includegraphics[height=5cm]{../../Plots/ba_plots/regr2/woregr2tr.png}
    \includegraphics[height=5cm]{../../Plots/ba_plots/regr2/1TFtr.png}
    \caption{\label{fig:regr2tf} TF bei Regression}
\end{figure}

Dieses Overfitting-Problem hat nur die Klassifikation. Dies muss an der Loss-Function, die für Klassifikation benutzt wird, liegen. Also 
am CategoricalCrossEntropy. Dahinter ist folgende Formel: 

\begin{equation}
    CCE = -\frac{1}{M} \sum_{k=1}^{K}\sum_{m=1}^{M} y_m^k * \log(h_w(x_m, k))
\end{equation} 

Dabei ist M die Anzahl der Datensamples, K die Anzahl der Klassen, $y_m^k$ das Target label, x der Input und $h_w$ die 
Gewichte \cite{rwcrossentropy}. 

    \section{Zeitnahme}
    In diesem Abschnitt werden alle Netzwerke hinsichtlich ihrer Trainingsdauer untersucht. Die Trainingszeit wird in jedem Plot angegeben und ist im Wesentlichen abhängig von der Anzahl der verwendeten Trainingsdaten sowie der maximal erlaubten Epochenanzahl. Daher werden für die 
Klassifikations- und Regressionsnetzwerke jeweils gleiche Mengen an Trainingsdaten und identische Gesamtepochenzahlen verwendet.

Beim Einsatz von Transfer Learning (TF) kommen stets der kleinste Target-Datensatz sowie der größte Source-Datensatz zum Einsatz. Ohne TF wird 
ausschließlich der kleinste Target-Datensatz verwendet. Jedes Klassifikationsnetzwerk wird über insgesamt 40 Epochen trainiert, wobei bei 
Anwendung von TF nach 20 Epochen gewechselt wird. Regressionsnetzwerke werden über 80 Epochen trainiert, mit einem TF-Wechsel nach 30 Epochen.

Die entsprechenden Graphen werden an dieser Stelle nicht dargestellt, da deren Anzahl zu groß ist und vergleichbare Darstellungen bereits an 
anderen Stellen verfügbar sind. Für die Einsichtnahme und Überprüfung dieser Ergebnisse verweisen wir auf das zugehörige GitHub-Repository 
unter \url{https://github.com/Lirras/BA_EvalTF_DDCN/tree/main/Plots/ba_plots/timing}. 

Es wird keine Early-Stopping-Metrik verwendet. Alle Klassifikationsnetzwerke erhalten identische Eingabedaten, ebenso wie alle 
Regressionsnetzwerke, um eine bessere Vergleichbarkeit innerhalb der jeweiligen Gruppe zu gewährleisten. Die verschiedenen Netzversionen – 
Cascade TF, Cascade und Complete – besitzen dabei dieselben Layer in gleicher Anzahl.

Im Folgenden wird eine Tabelle präsentiert, die die Trainingszeiten aller Netzwerke in einer vergleichbaren Form zusammenfasst:

\begin{table}[h!]
    \begin{center} 
        \begin{tabular}{l|l|l|l}
            \textbf{Netzwerk} & \textbf{Cascade TF} & \textbf{Cascade} & \textbf{Complete} \\
            \hline
            ConvMaxPool & 78 & 25 & 20 \\
            1DConv & 207 & 34 & 30 \\
            ClassOneDense & 79 & 28 & 13 \\
            RegressionTwo & 11 & 12 & 17 \\
            OneLayer & 16 & 18 & 11
        \end{tabular}
        \caption{
            \small{Dies stellt den Vergleich der Trainingsdauer zwischen den jeweiligen Netzwerken und deren Varianten dar. Die Zeitangaben erfolgen in Sekunden.}}
        \label{tab:time}
    \end{center}
\end{table}

In Tabelle \ref{tab:time} sind die Trainingszeiten der unterschiedlichen Netzwerkvarianten zusammengefasst. Die Spalte "Cascade TF" umfasst 
Kaskadennetzwerke mit Transfer Learning (TF). Die Spalte "Cascade" enthält Kaskadennetzwerke, die ausschließlich auf dem Target-Datensatz 
trainiert wurden. Die Spalte "Complete" schließlich zeigt die Trainingszeiten von Netzwerken, die weder TF verwenden noch kaskadiert sind, deren 
Layer vor dem Training festgelegt wurden und die vollständig auf dem Target-Datensatz trainiert wurden.

Auffällig ist, dass die Regressionsnetzwerke keine signifikanten Unterschiede in der Trainingsdauer aufweisen. Teilweise benötigt die Variante 
mit TF sogar weniger Zeit als ohne, was darauf zurückzuführen ist, dass der Target-Datensatz bei der Regression etwas größer als der 
Sourcedatensatz ist. Beide Datensätze umfassen jedoch mit etwas über 200 Trainingsbeispielen nur eine geringe Stichprobengröße.

Im Gegensatz dazu benötigen alle Klassifikationsnetzwerke mit TF eine längere Trainingszeit als ohne TF. Dies liegt daran, dass sie zunächst auf 
dem Sourcedatensatz trainieren, welcher mit etwa 48.000 Trainingsbeispielen deutlich größer ist, während die anderen Netzwerke direkt auf dem 
kleinen Target-Datensatz mit 732 Samples trainieren.

Da die Cascade-TF-Netzwerke auf dem Sourcedatensatz trainieren und die anderen Netzwerke ausschließlich auf dem Target-Datensatz, sind letztere 
in der Regel schneller. Innerhalb dieser Gruppe weisen die Complete-Netzwerke, die nicht kaskadiert sind, die kürzesten Trainingszeiten auf. Dies 
ist darauf zurückzuführen, dass bei jedem Kaskadennetzwerk das Output-Layer in jeder Stufe berechnet wird, während die Complete-Netzwerke nur ein 
einziges Output-Layer besitzen. Zudem entfallen hier zusätzliche Vorhersagen sowie die Berechnung von Augmented Vectors.

Nicht zuletzt sind die verwendeten Netzwerke nur moderat komplex und der betrachtete Target-Datensatz ist relativ klein, sodass die Trainingszeit 
weniger vom eigentlichen Training, sondern vielmehr von den damit verbundenen Berechnungen dominiert wird.


    \chapter{Klassifikation}  % 4.Klassifikation
    Hier werden die beiden Regressionsnetze vorgestellt. Beide haben als Input Tabellen mit drei Spalten. 
Welche das genau sind, wurde oben bereits erklärt. Sie haben ebenfalls beide den Adam Optimizer mit der Mean Squared Error-Lossfunction. 
Als Outputlayer wird für Regression typisch ein einzelnes Linear Layer mit einer Node und der Linear Activation Function genutzt. 

\begin{figure}[htpb]
    \centering
    \includegraphics[height=6cm]{../../Graphiken/regressiontwo_2.png}
    \caption{\label{fig:regr2} Vorstellung RegressionTwo Netzwerk}
\end{figure}

In Figure 5.1 ist das Regr2-Netzwerk mit allen seinen Layern. Dies ist ein Deep Cascade Netzwerk. Es wird also Layer für Layer trainiert. 
Dabei ist die Zahl hinter Linear die Anzahl der Nodes und die Zahl hinter 
Dropout die Prozente bezüglich dem Wert eins, die während des Trainings pro Epoche wegfallen. 

Das 1Lay-Netzwerk ist das Direct Cascade Regressionsnetz. Dieses hat nur ein Hiddenlayer mit einem Linearlayer mit 128 Nodes. Die 
Aktivierungsfunktion in diesem Layer ist Relu. Es wird iterativ genutzt und zwischen den Netzen Wissen mittels eines Augmented Vectors als 
neuen Input übertragen. 
Dieser wird mit der Prediction des vorherigen Netzes berechnet, indem diese als neue Spalte in der Inputtabelle des bisherigen Inputs hinzugefügt 
wird. Dies ist der Augmented Vector, der als neuer Input für das nächste Netz dient. 

    \section{Größe des Targetdatensatzes}
    In diesem Abschnitt wird die Auswirkung unterschiedlicher Größen des Target-Datensatzes auf die Performanz der Netzwerke untersucht. Es wird 
die Hypothese überprüft, dass die Leistung mit abnehmender Datenmenge schlechter wird.

Für eine vergleichbare Basis werden alle Netzwerke jeweils über insgesamt 40 Epochen trainiert, wobei nach 20 Epochen ein TF 
durchgeführt wird. Diese Untersuchung erfolgt anhand der Netzwerke CMP, 1DC und COD. Zusätzlich wird auch die Performanz auf dem Testdatensatz 
betrachtet.

\begin{figure}[htpb]
    \includegraphics[height=5cm]{../../Plots/ba_plots/targetgroesse/cmp_ts.png}
    \includegraphics[height=5cm]{../../Plots/ba_plots/targetgroesse/1dc_ts.png}
    \caption{\label{fig:targetgroessedeepdir} 
    \small{Abgebildet ist die Veränderung der Accuracy bei Convolutional-Netzwerken in Abhängigkeit von der Größe des Target-Datensatzes. Die 
    Plots zeigen jeweils die Ergebnisse auf dem Testdatensatz des Target-Datensatzes, wobei links das CMP-Netzwerk und rechts das 1DC-Netzwerk 
    dargestellt sind.
    Die dargestellten Tests umfassen für CMP die Varianten CMP:TF2/732/10, CMP:TF2/7k/10, CMP:TF2/21k/10, CMP:TF2/36k/10 und CMP:TF2/51k/10, 
    sowie für 1DC die Varianten 1DC:TF2/732/10, 1DC:TF2/7k/10, 1DC:TF2/21k/10, 1DC:TF2/36k/10 und 1DC:TF2/51k/10, jeweils dargestellt in den 
    Farben Gelb, Grün, Lila, Blau und Rot.
    }}
\end{figure}

In Abbildung \ref{fig:targetgroessedeepdir} sind die Testläufe in Abhängigkeit von der Größe des Trai-\\ningsdatensatzes dargestellt. Die erste 
Zahl in der Legende bezeichnet die Datenmenge, die zweite Zahl die Trainingsdauer. Die Datenmenge bezieht sich ausschließlich auf den 
Target-Datensatz. Es zeigt sich, dass die Trainingsdauer mit zunehmender Datenmenge ansteigt. Zudem bestätigt sich die Annahme, dass die 
Performanz bei Kaskadennetzwerken mit TF mit wachsender Datenmenge verbessert wird. Darüber hinaus weist das Deep Cascade 
Netzwerk eine leicht bessere Performanz auf als das Direct Cascade Netzwerk, was vermutlich darauf zurückzuführen ist, dass Direct Cascade 
Netzwerke lediglich ein Hidden Layer besitzen, während Deep Cascade Netzwerke aus mehreren Hidden Layern bestehen und daher in der Lage sind, 
komplexere Problemstellungen effizienter zu erlernen. 

\begin{figure}[htpb]
    \centering
    \includegraphics[height=5cm]{../../Plots/ba_plots/targetgroesse/cod_ts.png}
    \caption{\label{fig:targetgroesselinear} 
    \small{Abgebildet ist die Veränderung der Test-Accuracy eines linearen Netz-werks in Abhängigkeit von der Größe des Target-Datensatzes. 
    Die dargestellten Tests umfassen die Varianten COD:TF2/732/10, COD:TF2/7k/10, COD:TF2/21k/10, COD:TF2/36k/10 und COD:TF2/51k/10, 
    dargestellt in den Farben Gelb, Grün, Lila, Blau und Rot.}}
\end{figure}

Abbildung \ref{fig:targetgroesselinear} zeigt die Ergebnisse des Direct Cascade Netzwerks, das aus-schließlich lineare Hidden Layer verwendet. 
Auffällig ist, dass dieses Netzwerk unabhängig von der Anzahl der Trainingsdaten stets eine schlechtere Performanz aufweist als die beiden 
anderen Netzwerke, welche Convolutional Layer enthalten. Dies ist darauf zurückzuführen, dass das lineare Netzwerk die relevanten Merkmale für 
die Bilderkennung nicht adäquat extrahieren kann, da die Daten eine komplexe und nicht-lineare Struktur aufweisen.

Generell zeigt sich in allen Testplots, dass die erzielte Accuracy der Netz-werke mit TF selbst bei ausreichender Datenmenge, die ein direktes 
Training auf dem 
Target-Datensatz ermöglichen würde, nie eine zufriedenstellend hohe Accuracy erreicht. Das direkte Training auf dem Target-Datensatz mit dem 
Deep Cascade Netzwerk mit sehr vielen Trainingsbeispielen hingegen erzielt eine Accuracy von etwa 70\%, was deutlich über den Ergebnissen der 
TF Netzwerke liegt.

    \section{Bilddimensionalität}
    Bei den beiden Convolution Direct Cascade Netzwerken ist der einzige Unterschied, dass sie die Bilder in ein- beziehungsweise in 
zweidimensionaler Form sehen. Dabei fällt aber auf, dass es im zweidimensionalen Fall etwas besser ist. 
Dies liegt daran, dass das eindimensionale Netz in dem Filterlayer nur die Daten direkt rechts und links mit einbezieht. Das 
zweidimensionale Netzwerk hingegen nutzt bei der Operation jenes Layers nicht nur die direkt rechts und links, sondern auch die Daten, 
die oben und unten angrenzend sind, sowie die Daten, die in jede Richtung schräg vorkommen. 
Die Verbesserung ist aber nur minimal, wie in Figure \ref{fig:dim} zu sehen. 

\begin{figure}[htpb]
    \includegraphics[height=5cm]{../../Plots/ba_plots/dimensionality/1dim_tr.png}
    \includegraphics[height=5cm]{../../Plots/ba_plots/dimensionality/2dim_tr.png}
    \caption{\label{fig:dim} 
    \small{Hier ist links der Test 1DC:TF2/732/10 und rechts 2DC:TF2/732/10. Es geht hier darum einen möglichst geringen Unterschied 
    zwischen den beiden zu haben, damit nur noch 1DC betrachtet werden muss.}}
\end{figure}

Da das zweidimensionale Netzwerk mit nicht so vielen Daten genutzt werden kann, hat es hier eine sehr viel kürzere Zeit. Es kann deshalb 
nicht genutzt werden, da die Berechnung des Augmented Vectors zu Speicherplatzproblemen im Arbeitsspeicher führt. 

Weil diese Veränderung nur minimal ist, reicht es nur das eindimensionale Netz in den meisten Fällen zu betrachten, weshalb die technischen 
Probleme beim zweidimensionalen Netzwerk nicht so hinderlich für die Evaluierung von TF sind. 

    \section{Augmentierung}
    In diesem Abschnitt wird die Erstellung der Augmented Vectors für die Direct Cascade Netzwerke im Klassifikationskontext beschrieben. Jedes der 
drei betrachteten Netzwerke – COD, 1DC und 2DC – verwendet dabei eine eigene Methode zur Generierung dieser Vektoren. Die Darstellung erfolgt in 
der ge-\\nannten Reihenfolge: zunächst das COD-, anschließend das 1DC- und abschließend das 2DC-Netzwerk.

Allen Netzwerken gemeinsam ist, dass sowohl der Input des Netzwerks als auch dessen Prediction zur Berechnung des Augmented Vectors 
herangezogen werden. Als Input dient entweder der ursprüngliche Datensatz oder – ab der zweiten Iteration – der jeweils zuletzt erzeugte Augmented 
Vector. Nur bei der ersten Iteration wird der originale Datensatz verwendet, da zu diesem Zeitpunkt noch kein Augmented Vector existiert.

Mit jeder weiteren Iteration wächst der Augmented Vector, da er Informationen aus allen vorhergehenden Netzwerken integriert und so das bisherige 
Wissen an das nächste Netzwerk weitergibt. Die Prediction entspricht dabei der Ausgabe des trainierten Netzwerks auf Basis der 
jeweils aktuellen Eingabe.

Im Folgenden bezeichnen die Variablen N, W, H und C die Anzahl der Datensamples, die Bildbreite, die Bildhöhe sowie die Anzahl der Kanäle 
(Channels).

Für das COD-Netzwerk gilt: Der Input liegt in der Form (N,WxH) vor, d.h., die Bilddaten wurden vorab in ein Vektorformat 
überführt. Die Prediction hat die Form (N,10), entsprechend der zehn Klassenzugehörigkeiten. Beide Matrizen werden entlang der zweiten 
Dimension miteinander konkateniert, was zur Definition des Augmented Vectors gemäß Gleichung 4.1 führt.

\begin{equation}
    AugVec(Input(N, W*H), Prediction(N, 10)) = (N, (W*H).10)
\end{equation}

Beim 1DC-Netzwerk liegt der Input in der Form (N,WxH,C) vor. Da der Kanal in diesem Fall eindimensional ist, wird die Channel-Dimension 
zunächst entfernt. Anschließend erfolgt die Berechnung des Augmented Vectors gemäß Gleichung 4.1. Nach der Berechnung wird die Channel-Dimension 
wieder er-gänzt, um die ursprüngliche Struktur beizubehalten. In beiden bisher betrach-teten Netzwerken (COD und 1DC) wächst der Augmented Vector 
mit jeder Iteration linear um $N * 10$ Einträge.

Das 2DC-Netzwerk hingegen arbeitet mit einem komplexeren Eingabeformat: (N,W,H,C). Die Prediction liegt auch hier in der 
Form (N,10) vor. Zur Fusion von Input und Prediction werden für jedes Sample N zehn zusätzliche Arrays erzeugt, jeweils in der 
Form (W,H,C). Diese Arrays enthalten in jedem Eintrag die Wahrscheinlichkeitswerte der Prediction für jede der zehn Klassen in Bezug auf N. 
Anschließend erfolgt eine Konkatenation entlang der Channel-Dimension, wodurch für jedes N sowohl die Eingangsdaten als auch die 
Vorhersagewahrscheinlichkeiten aller zehn Klassen kanalweise zusammengeführt werden. Dieses Verfahren 
führt zur Definition des Augmented Vectors gemäß Glei-chung 4.2.

\begin{multline}
    AugVec(Input(N, W, H, C), Prediction(N, 10)) = Input(N, W, H, C.ConVec)\\
    ConVec(W, H, C)[0-9] = Prediction(10)[0-9]
\end{multline}

Der sogenannte ConVec bezeichnet den Vektor, in dem die einzelnen Werte der Prediction – entsprechend den Klassen eins bis zehn – jeweils in 
Form eines Arrays der Dimension (W,H,C) abgelegt sind. Bei jeder Iteration des Netz-werks führt dieses Verfahren jedoch zu einem 
speicherintensiven Wachstum des Augmented Vectors. Die Skalierung des Speicherbedarfs erfolgt gemäß der in Gleichung 4.3 dargestellten Beziehung.

\begin{equation}
    AugVecNew = N*W*H*C_{old} + N*W*H*10
\end{equation}

Daraus ergibt sich, dass der Arbeitsspeicherbedarf bei jeder Iteration mit einer Steigerungsrate wächst, die dem Zehnfachen der Größe des 
jeweiligen Datensatzes entspricht. Bei Datensamples, die bereits im Ausgangszustand eine Größe von 8192 Bytes pro Datenpunkt aufweisen, führt 
dieses Vorgehen zu einem sehr stark linear ansteigenden Speicherverbrauch. Daher ist die hier beschriebene Methode zur Erstellung des Augmented Vectors aus 
praktischer Sicht nicht sinnvoll einsetzbar. Aus diesem Grund wird das 2DC-Netzwerk im weiteren Verlauf der Arbeit nicht mehr berücksichtigt.

    \section{Mit und Ohne}
    \subsection{TF}
    Hier werden die Netze jeweils einmal mit und einmal ohne Transferlernen ausgetestet. Es werden nur die Direct Cascade Netzwerke betrachtet und 
sie werden mit deutlich mehr Netziterationen trainiert als bisher. Die Epochenanzahl pro Netzwerk bleibt aber gleich. Dabei werden jeweils nur 
wenig Targetdaten verwendet. 

\begin{figure}[htpb]
    \includegraphics[height=5cm]{../../Plots/ba_plots/classTF/1dc_tr.png}
    \includegraphics[height=5cm]{../../Plots/ba_plots/classTF/wo1dc_tr.png}
    \caption{\label{fig:1dc_tr} 
    \small{Hier ist links der Test 1DC:TF10/732/10/30 und rechts 1DC:732/10/30. Das eine ist mit TF, das andere ohne. Der mit TF dauert viel länger, 
    da der Sourcedatensatz groß ist. Bei beiden ist gut zu sehen, dass die Performanz des Netzes sehr schlecht ist. Egal ob mit oder ohne TF.}}
\end{figure}

Wie in Figure 4.4 zu sehen gibt es keinen Unterschied zwischen der Accuracy mit TF zu der ohne bei Convolutional Layern. In beiden Fällen ist diese 
extrem schlecht. Dies zeigt sich auch auf den Testdaten. 

\begin{figure}[htpb]
    \includegraphics[height=5cm]{../../Plots/ba_plots/classTF/cod_tr.png}
    \includegraphics[height=5cm]{../../Plots/ba_plots/classTF/wocod_tr.png}
    \caption{\label{fig:cod_tr} 
    \small{Hier ist links der Test COD:TF10/732/10/30, der mit TF ist und rechts COD:732/10/30, der ohne ist. Auch bei dem anderen Fall mit 
    anderen Hidden Layern im Netzwerk ist es sowohl mit als auch ohne TF schlecht.}}
\end{figure}

In Figure 4.5 zeigt sich dasselbe Bild nur auf Basis von Linear Layern. Dies kann zwei Gründe haben: Entweder funktioniert das Kaskadieren nicht 
oder es sind nicht genügend Targetdaten vorhanden. Letzteres wurde oben ausgetestet und lieferte zwar bessere, aber trotzdem nur mäßige Ergebnisse. 
Das Rauschen in den Plots kommt hier daher, dass alle zehn Epochen ein neues Netzwerk angefangen wird zu lernen. Dies hat zwar das Wissen aller 
vorherigen Netze im Input, aber nicht in der Art, dass die Gewichte direkt gleich gut sind. 

Daraus folgt also, dass es Probleme beim Kaskadieren geben muss. 

    \subsection{Kaskadierung}
    Im Folgenden wird untersucht, welche Auswirkungen auftreten, wenn auf das Kaskadieren verzichtet wird. Da ein zufriedenstellendes Ergebnis 
nur ohne Verwendung von TF zu erwarten ist, erfolgt das Training direkt auf dem Target-Datensatz. Die Architektur der 
vollständigen Netzwerke wird so angepasst, dass die Gesamtanzahl der Hidden Layer der Summe der Hidden Layer aller einzelnen Netzwerke im 
Direct-Cascade-Verfahren entspricht. Dabei wird die Gesamtanzahl der Trainings-Epochen beibehalten, um eine vergleichbare Trainingsdauer 
sicherzustellen.

\begin{figure}[htpb]
    \includegraphics[height=5cm]{../../Plots/ba_plots/classnocascade/1dc.png}
    \includegraphics[height=5cm]{../../Plots/ba_plots/classnocascade/cod.png}
    \caption{\label{fig:nocascade} 
    \small{Die dargestellten Ergebnisse zeigen Testläufe ohne Kaskadierung. Konkret sind links die Resultate für das 1DC-Netzwerk (1DC:Comp/732//30) 
    und rechts für das COD-Netzwerk (COD:Comp/732//30) dargestellt. Es ist ersichtlich, dass eines der Modelle in einem lokalen Maximum stecken bleibt. 
    Zudem lässt sich aus den Ergebnissen ableiten, welcher maximale Accuracy-Wert unter der gegebenen, begrenzten Datenmenge realistischerweise 
    erreicht werden kann. Dieser Wert wird ausschließlich durch den Einsatz der vollständigen, hier beschriebenen Netzwerkarchitekturen erzielt.}}
\end{figure}

In Abbildung \ref{fig:nocascade} fällt auf, dass während der meisten Epochen kein Lerneffekt eintritt. Zudem ist Overfitting 
erkennbar – ein zu erwartendes Verhalten angesichts der geringen Menge an Trainingsdaten. Obwohl in diesem Experiment kein TF eingesetzt wurde, 
zeigt einer der beiden Plots in der Mitte einen plötzlichen Anstieg der Accuracy. Dieser Anstieg wurde durch eine minimale Verbesserung des 
Trainingswerts bei gleichzeitig minimaler Verschlechterung des Validierungswerts ausgelöst; beide Änderungen lagen im Bereich von 
Zehntelprozenten. Dies deutet auf das Erreichen eines lokalen Maximums im Trainingsverlauf hin.

Im Vergleich dazu bleibt das andere Netzwerk dauerhaft auf dem Niveau dieses lokalen Maximums – beide Resultate zeigen exakt identische Werte. 
Aus Abbildung \ref{fig:nocascade} lässt sich zudem ableiten, dass unter den gegebenen Bedingungen eine maximale Accuracy von etwa 40\% erreichbar 
ist. Dieser Wert stellt das globale Maximum dar, da er die bestmögliche Performanz auf den Trainingsdaten widerspiegelt.

Weder das reine Kaskadieren noch die Kombination aus Kaskadierung und TF erreichen vergleichbare Ergebnisse – beide Varianten 
erreichen lediglich eine maximale Accuracy von etwa 20\% und damit nur etwa die Hälfte der möglichen Leistung.

Diese Beobachtungen legen nahe, dass die Ursache für die stark eingeschränkte Klassifikationsleistung im Direct-Cascade-Verfahren mit 
TF bereits in der Art der Kaskadierung selbst zu suchen ist. Mögliche Gründe hierfür könnten in der wiederholten Anwendung der 
Categorical-Crossentropy-Verlustfunktion liegen, die sich gegenseitig negativ beeinflussen könnte. Alternativ könnte auch die 
Softmax-Aktivierungsfunktion oder das konkrete Vorgehen beim Kaskadieren die Ursache darstellen.

Letztlich erweist sich der Einsatz von TF in Kombination mit dem Direct-Cascade-Verfahren unter Verwendung der in dieser Arbeit 
genutzten Datensätze und der beschriebenen Augmented Vectors als nicht zielführend, da die erzielten Accuracy-Werte selbst bei umfangreichen 
Trainingsdaten 60\% nicht überschreiten und im Vergleich zu einem vollständig trainierten Netzwerk signifikant schlechter ausfallen.


    % \section{Filternetze}
    % Die Klassifikation über Filternetze im Direct Cascade Ansatz. 
Es wird zudem zwischen eindimensionalen und zweidimensionalen Bilddaten unterschieden, um zu zeigen, dass die Dimensionalität für die 
Accuracy irrelevant ist und es im zweidimensionalen nur länger dauert. 

Der zweidimensionale Fall hat folgende Updateregel: 
Die Prediction wird ausgelesen und dessen Wert wird in ein Array mit demselben Shape hineingeschrieben und dies wird dann mit den Trainingsdaten 
auf der Channelachse konkateniert. Dies ergibt den Augmented Vector, der als Input für das nächste Netz genutzt wird.

Die Besonderheit des zweidimensionalen Netzes ist es, dass es in der Updateregel sehr viel Arbeitsspeicher benötigt wird. Deshalb wird nicht nur 
der Targetdatensatz auf 1\% verkleinert, sondern auch der Sourcedatensatz. 

Es wurde einmal ohne Metrik, einmal mit der Accuracy-Metrik und einmal mit der Loss-Letrik ausgetestet.

\begin{figure}[htpb]
    \includegraphics[height=5cm]{../../Plots/DirClass_LilConv/Dir2DLilConvTrainTen2Ten.png}
    \includegraphics[height=5cm]{../../Plots/DirClass_LilConv/Dir2DLilConvTestTen2Ten.png}
    \includegraphics[height=5cm]{../../Plots/DirClass_LilConv/Ten2TenTrain_ACC.png}
    \includegraphics[height=5cm]{../../Plots/DirClass_LilConv/Ten2TenTest_ACC.png}
    \includegraphics[height=5cm]{../../Plots/DirClass_LilConv/Ten2TenTrain_Loss.png}
    \includegraphics[height=5cm]{../../Plots/DirClass_LilConv/Ten2TenTest_Loss.png}
    \caption{\label{fig:2dconv}}
\end{figure}

Die Figure 2.1 zeigt die Plots der Tests zuerst ohne eine Metrik, dann mit der ACCM und zum Schluss mit der LM. 
Da der Sourcedatensatz verringert wurde, ist die Accuracy im ersten Bereich geringer als es erwartbar ist. 
Das Early Stopping ist hier nicht erkennbar, da die Berechnungszeit des Augmented Vector hier für die Zeit entscheidend ist.

Im eindimensionalen Fall gibt es folgende Updateregel: 
Die Bilddaten werden zuerst in eindimensionale Bilder verändert mit Channels. Dies ist der Input des Netzes. Die Prediction und der Netzinput, 
dessen Channelachse vorübergehend entfernt wird, werden direkt konkateniert und das Ergebnis, um die Channelachse erweitert. 

Hier wird nur der Targetdatensatz verkleinert und einem ohne Metrik, mit der ACCM und der LM ausgetestet. Dies zeigt die Figure 2.2.

\begin{figure}[htpb]
    \includegraphics[height=5cm]{../../Plots/DirClass_OneDConv/Ten2TenTrain.png}
    \includegraphics[height=5cm]{../../Plots/DirClass_OneDConv/Ten2TenTest.png}
    \includegraphics[height=5cm]{../../Plots/DirClass_OneDConv/DataTrain_ACC_Metr.png}
    \includegraphics[height=5cm]{../../Plots/DirClass_OneDConv/DataTest_ACC_Metr.png}
    \includegraphics[height=5cm]{../../Plots/DirClass_OneDConv/Ten2TenTrain_Loss.png}
    \includegraphics[height=5cm]{../../Plots/DirClass_OneDConv/Ten2TenTest_Loss.png}
    \caption{\label{fig:1dconv}}
\end{figure}

Hier werden die Metriken eindeutig gesehen, denn die Trainingszeit ist sehr verschieden. Es wird auch klar, dass der eindimensionale Fall 
schlechter ist als der Zweidimensionale.
Dies liegt daran, dass im eindimensionalen nur die Daten rechts und links von dem betrachteten Pixel in die Berechnung mit eingezogen werden 
können, während im zweidimensionalen zusätzlich auch die Daten oben, unten und die vier Ecken jenes Kreuzes betrachtet werden.
% da der eindimensionale Fall nicht die Verhältnisse der Pixel innerhalb einer Achse auf beiden Achsen betrachten kann. 
Beide Netze haben eine sehr schlechte Accuracy. Dies liegt aber nicht daran, dass auf dem Sourcedatensatz Overfitting passiert ist, denn es 
wird nicht besser, wenn der Wechsel der Datensätze beliebig nach vorne geschoben wird.

    % \section{Linearnetze}
    % Die Klassifikation über Linearnetze mittels eines Direct Cascade Ansatzes. 

Das Netz hat ein Linearlayer mit 512 Nodes und der ReLU-Activation function.

Für den augmented Vector werden alle inputdaten in eindimensionale Bilder verwandelt. Der Input wird mit der Prediction konkateniert und 
direkt an das nächste Netz weitergegeben.

Dieses Netz wurde ohne Metrik, mit der ACCM und LM ausgetestet. 

\begin{figure}[htpb]
    \includegraphics[height=5cm]{../../Plots/MnistLongDense/DataTrain.png}
    \includegraphics[height=5cm]{../../Plots/MnistLongDense/DataTest.png}
    \includegraphics[height=5cm]{../../Plots/MnistLongDense/Ten2Ten_Train_ACC.png}
    \includegraphics[height=5cm]{../../Plots/MnistLongDense/Ten2Ten_Test_ACC.png}
    \includegraphics[height=5cm]{../../Plots/MnistLongDense/Ten2Ten_Train_Loss.png}
    \includegraphics[height=5cm]{../../Plots/MnistLongDense/Ten2Ten_Test_Loss.png}
    \caption{\label{fig:linclass}}
\end{figure}

Die Figure 2.3 zeigt, dass ein Netzwerk mit nur Linearlayers etwas schlechter ist als ein zweidimensionales Filternetz. 
Hier werden die Metriken auch bereits gesehen, aber sie bringen kein erhofftes Ergebnis. 

Die Klassifikation funktioniert mit den Updateregeln und diesen Netzen nicht. Auch TF bringt dabei nichts. Das einzige, 
was halbwegs etwas bringt, ist, wenn mehr Targetdaten benutzt werden. Aber dann wird TF auch nicht mehr gebraucht. 


    \chapter{Regression}  % 5.Regression
    Hier werden die beiden Regressionsnetze vorgestellt. Beide haben als Input Tabellen mit drei Spalten. 
Welche das genau sind, wurde oben bereits erklärt. Sie haben ebenfalls beide den Adam Optimizer mit der Mean Squared Error-Lossfunction. 
Als Outputlayer wird für Regression typisch ein einzelnes Linear Layer mit einer Node und der Linear Activation Function genutzt. 

\begin{figure}[htpb]
    \centering
    \includegraphics[height=6cm]{../../Graphiken/regressiontwo_2.png}
    \caption{\label{fig:regr2} Vorstellung RegressionTwo Netzwerk}
\end{figure}

In Figure 5.1 ist das Regr2-Netzwerk mit allen seinen Layern. Dies ist ein Deep Cascade Netzwerk. Es wird also Layer für Layer trainiert. 
Dabei ist die Zahl hinter Linear die Anzahl der Nodes und die Zahl hinter 
Dropout die Prozente bezüglich dem Wert eins, die während des Trainings pro Epoche wegfallen. 

Das 1Lay-Netzwerk ist das Direct Cascade Regressionsnetz. Dieses hat nur ein Hiddenlayer mit einem Linearlayer mit 128 Nodes. Die 
Aktivierungsfunktion in diesem Layer ist Relu. Es wird iterativ genutzt und zwischen den Netzen Wissen mittels eines Augmented Vectors als 
neuen Input übertragen. 
Dieser wird mit der Prediction des vorherigen Netzes berechnet, indem diese als neue Spalte in der Inputtabelle des bisherigen Inputs hinzugefügt 
wird. Dies ist der Augmented Vector, der als neuer Input für das nächste Netz dient. 

    \section{Datenaugmentation}
    Der Sourcedatensatz Bost hat nur 506 Datensamples insgesamt und ist somit sehr klein. 
Davon werden im folgenden 51 Samples als Testset, 91 als Validationset und 364 als Trainingsset genutzt. 

Der Targetdatensatz Cali ist hingegen mit etwa 26 Tausend Samples sehr groß und wird nach Bedarf verkleinert. 
Es werden jeweils als Batch 16 Samples genutzt. 

    \subsection{Viele Daten}
    Hier werden alle Vergleiche bei vollem Targetdatensatz verwendet. 
Dadurch umfassen die Trainingsdaten etwa 8000 Samples und die Testdaten etwa 4000. 
Diese Vergleiche beinhalten Komplette, TF Cascade und Cascade Netzwerke. Dabei wird als komplettes Netzwerk ein Netzwerk, welches alle 
Layer vorher definiert hat und dieses eine Netzwerk mit einem einzigen Trainingsaufruf alles trainiert, was der normale Fall eines 
neuronalen Netzwerks ist. 

Es wird sowohl der Vergleich zwischen Deep Cascade, Direct Cascade und dem Kompletten Netzwerk gemacht.  

\begin{figure}[htpb]
    \includegraphics[height=5cm]{../../Plots/ba_plots/regression_large/onelayer_ts.png}
    \includegraphics[height=5cm]{../../Plots/ba_plots/regression_large/woonelayer_ts.png}
    \caption{\label{fig:largeregr2} Vergleich im OneLayer Netzwerk}
\end{figure}

In Figure 5.2 ist das Ergebnis des Direct Cascade. Auf der linken Seite ist die Versioin mit TF. Dabei fällt auf, dass es bei vielen 
Targetdaten besser ist auf dem Targetdatensatz direkt zu lernen, denn 
das Wissen, welches vom Sourcedatensatz übertragen wird, ist nicht so passend, wie das von den Targetdaten. Dies passiert aber nur, wenn es 
genügend Targetdaten gibt. Dabei kommt auch beim Deep Cascade ähnliche Werte heraus. 
Ein komplettes Netzwerk, wie es für Figure 5.3 genutzt wurde, hat etwa dieselbe Performanz, wie die beiden Kaskadenversionen. 

\begin{figure}[htpb]
    \includegraphics[height=5cm]{../../Plots/ba_plots/regression_large/onelayer_complete.png}
    \caption{\label{fig:largeregr2comp} Komplettes OneLayer}
\end{figure}

Dies liegt daran, dass die lineare Aktivierungsfunktion und der Mean Squared Error Loss für die Kaskadierung nicht störend sind. 
Also funktioniert Regression deutlich besser mit Kaskadennetzwerken als die Klassifikation, da sie an das Ergebnis des kompletten Netzes 
herankommt, was bei Klassifikation nie passiert ist. 

    \subsection{Wenig Daten}
    In diesem Unterkapitel wird der Targetdatensatz deutlich verkleinert und hat dann nur noch 240 Datensamples. Es werden dieselben Tests wie 
im vorherigen Unterkapitel durchgeführt. 

% Es wird zuerst angefangen mit Deep Cascade, da es Unterschiede zwischen den beiden Kaskadenversionen gibt. 
% Bei wenigen Targetdaten gibt es einen Unterschied zwischen Deep Cascade und Direct Cascade. Beides läuft zwar etwas schlechter als mit vielen 
% Daten ist aber mit TF unter Umständen besser als ohne. Dabei ist Deep Cascade etwas besser als Direct Cascade. 

\begin{figure}[htpb]
    \includegraphics[height=5cm]{../../Plots/ba_plots/regression_small/regr2_ts.png}
    \includegraphics[height=5cm]{../../Plots/ba_plots/regression_small/woregr2_ts.png}
    \caption{\label{fig:smallregr} Vergleich RegressionTwo Netzwerk}
\end{figure}


In Figure 5.4 sind die Ergebniss des Deep Cascade Netzwerks. Es ist ohne TF tatsächlich besser als mit. Dies liegt daran, dass die Gewichte der 
ersten Hälfte des Netzes nur auf dem Sourcedatensatz passend gelernt haben. 

Es gibt aber deutliche Unterschiede zu Direct Cascade, weshalb beide hier gezeigt werden. 

\begin{figure}[htpb]
    \includegraphics[height=5cm]{../../Plots/ba_plots/regression_small/onelayer_ts.png}
    \includegraphics[height=5cm]{../../Plots/ba_plots/regression_small/woonelayer_ts.png}
    \caption{\label{fig:smallonl} Vergleich OneLayer Netzwerk}
\end{figure}

Figure 5.5 bezieht sich auf das Direct Cascade Netzwerk. Es wird deutlich, dass in dieser Kaskadierungsvariante das Netz deutlich schlechter 
ohne TF ist als mit. Die liegt daran, dass das TF über den Augmented Vector als veränderten Input funktioniert und nicht über Gewichte, die 
von dem Sourcedatensatz fertig gelernt worden sind. Allerdings ist das Deep Cascade Netzwerk trotzdem besser. Dies kann aber auch 
an dem dahinter liegenden Netz liegen, da sie nicht nur die gleichen Layer haben. Noch besser ist die Variante, 
die weder Kaskadierung noch TF nutzt, wie in Figure 5.6 gezeigt. 

% Wieso ist die komplette Version trotz der wenigen Daten besser?
\begin{figure}
    \includegraphics[height=5cm]{../../Plots/ba_plots/regression_small/onelayer_complete.png}
    \caption{\label{fig:smallonlcomp} Komplettes OneLayer}
\end{figure}

Der MAE-Wert der Testdaten beläuft sich hier auf 53 Tausend Dollar, während dieser sonst bei so wenig Datensamples bei 60 bis 80 liegt. 

    \section{Early Stopping}
    Hier werden die Regressionsnetze mit Early Stopping verwendet und auch erklärt, warum das bei Klassifikation sinnlos ist. 
Dabei werden nur die Direct Cascade Netze betrachtet. 

Sowohl für die Regressionsnetze als auch für die Klassifikationsnetze wurde LM verwendet. 
Zudem für Regression noch MAEM und für Klassifikation ACCM. 

Bei der Klassifikation kommt es nur manchmal zu einem Abbruch der Epochen über das ACCM, aber es wird dadurch nicht besser. Mit LM kommt 
dieser Abbruch öfter vor und das Training geht somit 
zwar schneller, jedoch bleibt Klassifikation mit Kaskadierung so schlecht, dass es nicht genutzt werden kann. Dass weder LM noch ACCM 
funktioniert sieht man deutlich in Figure 5.7. ACCM ist die einzige der hier vorkommenden Metriken, dessen Ziel ein Maximum ist. 

\begin{figure}[htpb]
    \includegraphics[height=4.5cm]{../../Plots/ba_plots/earlystopping/lossmetric/1dconv_ts.png}
    \includegraphics[height=4.5cm]{../../Plots/ba_plots/earlystopping/intermetric/1dconv_ts.png}
    \caption{\label{fig:1dconvmetrics} LM und ACCM mit 1DConv}
\end{figure}

Deshalb wird sich hier eingehender mit dem Regressionsnetz OneLayer befasst. Die Metriken LM und MAEM suchen dabei ein Minimum. 

\begin{figure}[htpb]
    \includegraphics[height=4.5cm]{../../Plots/ba_plots/earlystopping/lossmetric/onelayer_ts.png}
    \includegraphics[height=4.5cm]{../../Plots/ba_plots/earlystopping/intermetric/onelayer_ts.png}
    \caption{\label{fig:onelayermetrics} LM und MAEM mit OneLayer}
\end{figure}

Dieses liefert mit den beiden Early-Stopping Metriken LM und MAEM 
halbwegs brauchbare Ergebnisse, jedoch sind diese deutlich schlechter als ein Training ohne diese, wie an den Werten von Figure 5.8 abgelesen 
werden kann. 

Diese Werte sind so schlecht als hätte man das OneLayer Netzwerk mit wenigen Targetdaten direkt auf diesen Datensatz lernen lassen. 
Das diese Early-Stopping Metriken so schleht sind, liegt daran, dass sie keine Verschlechterung im Validationset des Datensatzes dulden und ab 
der ersten das Netz der aktuellen Netziteration beenden. Dadurch ist selten das tatsächliche Minimum das, was über den Augmented Vector 
weitergegeben wird, sondern nur ein leicht abweichendes. Dazu kommt, dass diese Metriken nicht das globale Minimum finden können, wenn sie 
auf ein lokales treffen, denn sie werden Versuchen in diesem zu verbleiben. 

    
    \chapter{Weiterführendes}  % 6.Diskussion
    % Wenn ich alle Ergebnisse direkt erläutere und erkläre, warum was wie ist, dann kann das hier weg.
    % \section{Erkenntnisse}
    % \textbf{Bei Klassifikation und Regression:}\\
Es zeigt sich ein relativ schneller Overfitting-Effekt auf dem Source-Datensatz.

An der Stelle des TFs bricht die Performanz des Netzes deutlich ein. Anschließend kommt es zwar zu einer teilweisen Erholung, 
jedoch erreicht die Leistung nicht mehr das vorherige Niveau.

In der vorliegenden Implementierung sind die herkömmlichen Netzwerke bei kleinen Datensätzen und flacher Netzstruktur schneller als die Direct 
Cascade Netzwerke, welche wiederum schneller trainieren als die Deep Cascade Architekturen. Die Performanz variiert jedoch signifikant 
zwischen den einzelnen Netzwerktypen. 

Bei zunehmender Netzwerktiefe und einer hohen Anzahl an Trainingsepochen kehrt sich das Zeitverhältnis zwischen den Kaskadenversionen des 
Netzwerks und dem vollständigen Netzwerk um, sodass Letzteres im Vergleich deutlich langsamer ist als beide Kaskadenvarianten.

\textbf{Klassifikation:}\\
Bei der Klassifikation ist der Accuracy-Wert primär von der Größe des Target-Datensatzes abhängig: Je geringer die Datenmenge, desto 
schlechter fällt die Klassifikationsgenauigkeit aus.

Die Klassifikationsleistung ist derart unzureichend, dass ein Einsatz in diesem Kontext nicht sinnvoll erscheint, insbesondere wenn die 
Accuracy nur bei etwa 20\% liegt.

Bei TF zeigt sich eine leicht schlechtere Performanz im Vergleich zum Training ohne TF.

Ein Cascade-Netzwerk erzielt eine schlechtere Leistung als ein vollständig trainiertes Netzwerk ohne Kaskadierung.

Die Verarbeitung mehrdimensionaler Augmented Vectors kann zu Problemen mit dem Arbeitsspeicher führen.

Die eindimensionale Klassifikation ist geringfügig schlechter als die zweidimensionale; dieser Unterschied ist jedoch minimal und kann 
vernachlässigt werden.

\textbf{Regression:}\\
Bei der Regression führt der Einsatz von TF bei wenigen Trainingsdaten zu besseren Ergebnissen als ein Training des Netzwerks ausschließlich 
auf dem Target-Datensatz von Grund auf — dies gilt jedoch nur für Direct Cascade Netzwerke.

Die Leistungsfähigkeit der Regressionsnetzwerke mit eingesetztem TF ist ausreichend, um eine praktische Anwendbarkeit zu gewährleisten.

Der Performanz-Abfall bei der Regression fällt insgesamt deutlich geringer aus als bei der Klassifikation.

Die hier verwendeten einfachen Early-Stopping-Metriken verschlechtern die Ergebnisse, da sie dazu neigen, in lokalen Minima stecken zu bleiben.

    \section{Ausblick}
    Die Klassifikation sollte nicht weiter im Rahmen von Deep- oder Direct-Cascade-Architekturen untersucht werden, da die Performanz unabhängig von 
der Datenmenge konstant schlechter ist als bei einem vollständig trainierten Netzwerk. Eine potenzielle Verbesserung könnte nur durch die 
Verwendung alternativer Loss-Funktionen anstelle von Categorical Cross-Entropy oder durch eine modi-fizierte Konstruktion der Augmented Vectors 
zwischen den Netzwerkmodulen erzielt werden. Ebenso wäre die Erprobung anderer Kaskadierungsverfahren denkbar, wobei deren Erfolgsaussichten 
jedoch als gering eingeschätzt werden.

Für die Regressionsnetzwerke besteht die Möglichkeit, alternative Loss-Funk-tionen anstelle des MSEs einzusetzen. 
Darüber hinaus sollten auch Target-Datensätze mit weniger als einhundert Instanzen als Eingabe untersucht werden, um die Leistungsfähigkeit 
der Modelle unter noch limitierteren Bedingungen zu evaluieren.

In beiden Anwendungsfällen könnten zudem andere Early-Stopping-Krite-rien angewandt werden, die auch geringfügige Verschlechterungen im 
Validie-rungsmaß tolerieren. Zusätzlich sind Metriken denkbar, die die Anzahl der Netziterationen adaptiv steuern. Des Weiteren können 
Trainingsverfahren ohne feste Maximalanzahl an Epochen implementiert werden, wobei hier das Risiko besteht, dass das Training unendlich lange 
andauert.

Darüber hinaus kann die Verwendung alternativer Optimierungsalgorithmen anstelle von Adam evaluiert werden, ebenso wie der Einsatz 
unterschiedlicher Source- und Target-Datensätze. Ebenso ist eine veränderte Vorverarbeitung der bestehenden Datensätze für die Netzwerke 
denkbar.

Weiterhin sollte untersucht werden, wie sich TF bei einem Wechsel der Aufgabenstellung verhält, ebenso wie bei einem 
kombinierten Task- und Domainwechsel.

    \section{Fazit}
    Die Klassifikation ist in Kombination mit Direct Cascade und TF unter Verwendung der hier eingesetzten Augmented Vektoren nicht praktikabel. 
Deep Cascade Architekturen erzielen zwar eine bessere Klassifikationsleistung, sind jedoch ebenfalls nicht sinnvoll einsetzbar.

Für die Regression erweist sich der Einsatz von TF in der Direct Cascade Variante als zielführend, da die Performanz entweder gleichwertig oder 
überlegen im Vergleich zu vollständig trainierten Netzwerken ist. Einzig bei sehr kleinen Datenmengen kann es zu Leistungseinbußen gegenüber dem 
vollständigen Netzwerk kommen.

Das Verfahren funktioniert auch mit Deep Cascade Netzwerken und zeigt selbst in diesem Ausnahmefall eine mit dem vollständigen Netzwerk 
vergleichbaren Performanz.

Vollständig trainierte Netzwerke mit der gleichen Gesamtanzahl an Hidden Layer sind insbesondere bei sehr kleinen Datensätzen schneller als Deep 
Cascade Modelle, welche wiederum schneller als Direct Cascade Architekturen sind. Aus diesem Grund ist es bereits bei geringen Datenmengen 
sinnvoller, ein vollständiges Netzwerk anstelle einer kaskadierten Architektur zu verwenden.


    \printbibliography
\end{document}
