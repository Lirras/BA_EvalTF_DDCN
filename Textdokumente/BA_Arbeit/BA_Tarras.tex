\documentclass[ngerman]{report}

\usepackage[
        bibencoding=utf8, 
        style=alphabetic
    ]{biblatex}

\usepackage{graphicx}
\usepackage{amsmath}
\usepackage{caption}

\bibliography{bibliography}

\title{Evaluierung von Transferlernen mit Deep Direct Cascade Networks}

\author{Simon Tarras}

\begin{document}
    \maketitle
    \tableofcontents
    % Vollständige Struktur anlegen
    % Einleitung
    % Auch etwas über den Code schreiben (Methoden der Vorstellung (Bsp. Klassendiagram))
    \chapter{Einleitung}  % 1.Einleitung
    \section{Einführung}
    Künstliche Intelligenz (KI) ist mittlerweile selbst außerhalb der Informatik einer breiten Öffentlichkeit bekannt. Die zugrunde liegende 
Technologie basiert in der Regel auf künstlichen neuronalen Netzen, die üblicherweise in einem einzigen Schritt vollständig konstruiert und 
trainiert werden. Dieser Prozess ist jedoch zeit- und rechenintensiv, weshalb alternative Verfahren wie die Cascade-Correlation-Methode 
entwickelt wurden \cite{cascor}. Erste Untersuchungen zeigten, dass diese Kaskadierungsstrategie bereits bei kleineren Netzarchitekturen 
zufriedenstellende Ergebnisse liefert. Darauf aufbauend wurden verschiedene erweiterte Netzwerkstrukturen und Kaskadierungsverfahren 
entwickelt \cite{cascade_network_architectures}, \cite{Constructive_Cascade}, \cite{deep_cascade_learning}.

Ein wesentlicher Vorteil kaskadierender Netzwerke liegt in ihrer modularen Struktur, die eine flexible Anpassung an unterschiedliche 
Datensätze und Aufgabenstellungen ermöglicht \cite{phd_deep_cascade}, \cite{transfer_learning}, \cite{survey_transfer}. Die vorliegende 
Arbeit ist im Kontext dieser Transferfähigkeit einzuordnen. Ziel ist es, die Leistungsfähigkeit verschiedener Netzwerkarchitekturen zu 
evaluieren und spezifische Herausforderungen im Zusammenhang mit Transferlernen (TF) und Kaskadierung zu analysieren.

    \section{Motivation}
    In vielen Anwendungsfällen, für die der Einsatz von KI grundsätzlich sinnvoll wäre, stehen lediglich sehr kleine Datensätze 
zur Verfügung. Diese Datenmengen sind oft so begrenzt, dass ein neuronales Netzwerk nicht ausreichend Informationen erhält, um ein robustes und 
generalisierbares Modell zu erlernen. % – selbst bei verlängerter Trainingsdauer bleibt die Modellleistung unzureichend.

Zusätzlich stellt die lange Trainingszeit klassischer neuronaler Netzwerke eine weitere Herausforderung dar. Beide Probleme – unzureichende 
Datenverfügbarkeit und hoher Trainingsaufwand – sollen in dieser Arbeit adressiert und verbessert werden.

Zur Bewältigung der Problematik kleiner Datensätze wird ein auf TF basierender Ansatz verfolgt. Dabei lernt das Modell zunächst 
auf einer verwandten, aber besser verfügbaren Datenbasis, um dieses Vorwissen anschließend auf die eigentliche Zielaufgabe zu übertragen. Dieses 
Vorgehen orientiert sich an kognitiven Lernprozessen beim Menschen, bei denen bereits erlernte Konzepte genutzt werden, um neue, ähnliche 
Inhalte zu erschließen. % – vergleichbar mit der Funktionsweise von Eselsbrücken.

Die Netzwerkarchitektur wird derart gestaltet, dass jeweils nur ein geringer Teil des Modells gleichzeitig trainiert wird. Dadurch soll der 
Trainingsaufwand reduziert und eine signifikante zeitliche Effizienzsteigerung erzielt werden. Zur Umsetzung dieses Ansatzes kommen die 
Kaskadierungsverfahren Deep Cascade und Direct Cascade zum Einsatz.

    \section{Related Work}
    Diese Arbeit baut auf den CasCor-Algorithmus \cite{cascor}, der wegen der langen Trainingsdauer entwickelt wurde, auf und nutzt diesen, 
um Direct Cascade \cite{cascade_network_architectures} 
durchzuführen, wie es in der Thesis von Marquez \cite{phd_deep_cascade} verwendet wurde. Zudem wird Deep Cascade verwendet als 
Vergleichsbasis \cite{deep_cascade_learning}. Dabei sind Deep Cascade Netzwerke solche, die iterativ aufgebaut werden \cite{Constructive_Cascade}. 
Es wird nur Domain-Wechsel angewandt, während es noch den Taskwechsel \cite{transfer_learning} gibt. Zudem gibt es bei Transfer Learning drei 
Probleme \cite{survey_transfer}, die hier auch bearbeitet werden. 

% Alle wissenschaftlichen Arbeiten, die hier dazu gehören und verwandt sind aufzählen. 

% Also die Direct Cascade Arbeiten von Ritter und Littmann zum Beispiel. 

% Ebenso Cascor von Fahlmann und Lebiere.

% Deep Cascade Learning von Enrico S. Marquez. 

% Wie sieht das mit dem Bachelorvortrag aus? Wieviel später kommt der?
% Passt das hier von der Länge und dem Inhalt? 


    \chapter{Methodik}  % 2.Methodik
    \section{Transferlernen}
    Transferlernen (TF) ist das Prinzip des Lernens über einer Eselsbrücke. 
Es gibt mehrere Varianten, wie TF verwendet werden kann. Nur wenn keine davon genutzt wird, wird nicht von TF gesprochen. 
Hier wird nur der Domain-Wechsel vorgestellt werden, da nur dieser genutzt wird. Ein Domain-Wechsel ist hier der Wechsel 
zwischen zwei verschiedenen Datensätzen, während diegleichen Netzarten genutzt wird. 
Dies wird Transductive Transferlernen\cite{survey_transfer} genannt. 
Das Wissen vom ersten Datensatz wird auf den zweiten übertragen. Der erste Datensatz ist dabei die Source, der Zweite das Target. 
Es gibt dabei drei Stellschrauben, bei denen nicht klar ist, was besser ist: What, How, When to Transfer \cite{survey_transfer}. 
Da es sowohl eine Klassifikation als auch eine Regression ausgetestet wird, werden jeweils zwei Source- und Targetdatensätze benötigt. 
Für Klassifikation wird die Source der Modified National Institute of Standards and Technology \cite{handwritten_digit} (MNIST) Datensatzes  
und der 
Street View House Numbers (SVHN) \cite{house_numbers} der Targetdatensatz sein. Beide müssen für das Transfer ein wenig 
verändert werden. Der MNIST wird von 28x28 Pixel auf 32x32 erweitert, während der SVHN von farbig auf schwarz-weiß verändert wird. 
Dies ist notwendig, da beide Datensätze als Input denselben Shape, also diegleichen Dimensionalitäten vorweisen, haben müssen. 
Bei der Regression ist der Sourcedatensatz der Boston Housing Prices (Bost) \cite{Boston_housing} und der Targetdatensatz der 
California Housing Prices (Cali) \cite{California_housing}. 

Beide Datensätze müssen stark reduziert werden. Von den Acht beziehungsweise Dreizehn Spalten bleiben nur Drei übrig. Dies hat den Grund, dass 
nur Spalten als Sinnvoll geachtet werden, die ein passendes gegenüber haben. Der Bost-Datensatz hat allerdings ein ethnisches Problem, da 
dieser eine Spalte enthält, die diskriminierend ist. Diese wird entfernt. 
Die einzigen Spalten des Bost-Datensatzes, die übrig bleiben sind: RM, AGE, LSTAT. RM ist die durchschnittliche Zimmeranzahl pro Wohnung, AGE 
ist die Anzahl der Häuser, die vor 1940 bewohnt wurden und LSTAT ist der prozentuale Anteil der Bevölkerung mit niedrigerem Status. 
Der Datensatz Cali behält nur die Spalten MedInc und HouseAge. MedInc ist das durchschnittliche Einkommen des Häuserblocks und HouseAge das 
durchschnittliche Alter. Aus den Spalten AveRooms und Households wird die durchschnittliche Anzahl von Zimmers pro Haushalt berechnet. AveRooms 
ist dabei die durchschnittliche Anzahl an Räumen innerhalb eines Häuserblocks, während Households die Anzahl der Haushalte innerhalb des 
Häuserblocks ist. Dadurch ist die berechnete Spalte zu der RM-Spalte von Bost passend. 
Da LSTAT und MedInc wahrscheinlich abhängig sind, da es vermutet wird, dass diejenigen Menschen, die einen niedrigeren Status vorweisen, weniger 
Einnahmen haben. Deshalb dürfte es über diese beiden Spalten möglich zu sein TF zu nutzen. Allerdings sind sie zueinander antiproportional, 
weshalb die LSTAT Spalte invertiert wird, damit es zur Proportionalität kommt. Komplexer ist die Berechnung des Alters der Häuser, da 
AGE nur die Anzahl der Häuser, die vor 1940 gebaut wurden, beeinhaltet, aber HouseAge das durchschnittliche Alter des Häuserblocks ist. 
Die Maximalanzahl der betrachteten Häuser im Bost-Datensatz ist einhundert und das Alter der Häuser vor 1940 ist 85, wenn man auf 2025 rechnet. 
Dadurch kann AGE auf die Art von HouseAge mit folgender Formel umgerechnet werden: 
\begin{equation}
    \frac{AGE * 85}{Maximalanzahl}
\end{equation}
Dadurch sind alle Source- und Targetdatensätze zueinander kompatibel. Damit ist ausreichend geklärt, mit was TF verwendet wird. 

Die nächste Frage, die geklärt werden muss ist das How to transfer. Dies wird jeweils ohne Veränderung der Weights der Netze gemacht. 
Es wird das neuronale Netz zuerst auf dem Sourcedatensatz trainiert und dann ohne irgendetwas zu tun auf den Targetdatensatz gewechselt, 
welcher auf demselbem Netz oder einem gleichen Netz wie zuvor ist. Wenn es dasselbe Netz ist, dann verändert sich nur aus welchem 
Datensatz der Input kommt, was bei Deep Cascade ist. Während bei dem gleichen Netz der Input immer vergrößert wird und das TF über diese 
Vergrößerung passiert, was bei Direct Cascade ist. 

Wann TF sinnvoll ist, ist nicht klar, weshalb es mal mit früherem und späteren TF probiert wird. 

    \section{Kaskadierung}
    Im Folgenden wird untersucht, welche Auswirkungen auftreten, wenn auf das Kaskadieren verzichtet wird. Da ein zufriedenstellendes Ergebnis 
nur ohne Verwendung von TF zu erwarten ist, erfolgt das Training direkt auf dem Target-Datensatz. Die Architektur der 
vollständigen Netzwerke wird so angepasst, dass die Gesamtanzahl der Hidden Layer der Summe der Hidden Layer aller einzelnen Netzwerke im 
Direct-Cascade-Verfahren entspricht. Dabei wird die Gesamtanzahl der Trainings-Epochen beibehalten, um eine vergleichbare Trainingsdauer 
sicherzustellen.

\begin{figure}[htpb]
    \includegraphics[height=5cm]{../../Plots/ba_plots/classnocascade/1dc.png}
    \includegraphics[height=5cm]{../../Plots/ba_plots/classnocascade/cod.png}
    \caption{\label{fig:nocascade} 
    \small{Die dargestellten Ergebnisse zeigen Testläufe ohne Kaskadierung. Konkret sind links die Resultate für das 1DC-Netzwerk (1DC:Comp/732//30) 
    und rechts für das COD-Netzwerk (COD:Comp/732//30) dargestellt. Es ist ersichtlich, dass eines der Modelle in einem lokalen Maximum stecken bleibt. 
    Zudem lässt sich aus den Ergebnissen ableiten, welcher maximale Accuracy-Wert unter der gegebenen, begrenzten Datenmenge realistischerweise 
    erreicht werden kann. Dieser Wert wird ausschließlich durch den Einsatz der vollständigen, hier beschriebenen Netzwerkarchitekturen erzielt.}}
\end{figure}

In Abbildung \ref{fig:nocascade} fällt auf, dass während der meisten Epochen kein Lerneffekt eintritt. Zudem ist Overfitting 
erkennbar – ein zu erwartendes Verhalten angesichts der geringen Menge an Trainingsdaten. Obwohl in diesem Experiment kein TF eingesetzt wurde, 
zeigt einer der beiden Plots in der Mitte einen plötzlichen Anstieg der Accuracy. Dieser Anstieg wurde durch eine minimale Verbesserung des 
Trainingswerts bei gleichzeitig minimaler Verschlechterung des Validierungswerts ausgelöst; beide Änderungen lagen im Bereich von 
Zehntelprozenten. Dies deutet auf das Erreichen eines lokalen Maximums im Trainingsverlauf hin.

Im Vergleich dazu bleibt das andere Netzwerk dauerhaft auf dem Niveau dieses lokalen Maximums – beide Resultate zeigen exakt identische Werte. 
Aus Abbildung \ref{fig:nocascade} lässt sich zudem ableiten, dass unter den gegebenen Bedingungen eine maximale Accuracy von etwa 40\% erreichbar 
ist. Dieser Wert stellt das globale Maximum dar, da er die bestmögliche Performanz auf den Trainingsdaten widerspiegelt.

Weder das reine Kaskadieren noch die Kombination aus Kaskadierung und TF erreichen vergleichbare Ergebnisse – beide Varianten 
erreichen lediglich eine maximale Accuracy von etwa 20\% und damit nur etwa die Hälfte der möglichen Leistung.

Diese Beobachtungen legen nahe, dass die Ursache für die stark eingeschränkte Klassifikationsleistung im Direct-Cascade-Verfahren mit 
TF bereits in der Art der Kaskadierung selbst zu suchen ist. Mögliche Gründe hierfür könnten in der wiederholten Anwendung der 
Categorical-Crossentropy-Verlustfunktion liegen, die sich gegenseitig negativ beeinflussen könnte. Alternativ könnte auch die 
Softmax-Aktivierungsfunktion oder das konkrete Vorgehen beim Kaskadieren die Ursache darstellen.

Letztlich erweist sich der Einsatz von TF in Kombination mit dem Direct-Cascade-Verfahren unter Verwendung der in dieser Arbeit 
genutzten Datensätze und der beschriebenen Augmented Vectors als nicht zielführend, da die erzielten Accuracy-Werte selbst bei umfangreichen 
Trainingsdaten 60\% nicht überschreiten und im Vergleich zu einem vollständig trainierten Netzwerk signifikant schlechter ausfallen.

    \subsection{Deep Cascade}
    Hier wird die Variante des Deep Cascade vorgestellt. 
Die Deep Cascade Netze werden iterativ während dem Training aufgebaut. Es bleibt dabei ein einziges Netz. Es wird zuerst 
definiert, welcher Optimizer und welcher Loss in dem Netz genutzt wird. 

\begin{figure}[htpb]
    \includegraphics[height=10cm]{../../Graphiken/deepcascade_2.png}
    \caption{\label{fig:deepcascade} Vorstellung Deep Cascade Aufbau}
\end{figure}

Sobald dies beides gemacht wurde, wird im Netz das erste Layer definiert. Dieses wird ergänzt durch ein Output Layer und dann trainiert. 
Wenn das Training beendet wird, wird das Output Layer gelöscht und ein neues Layer hinzugefügt, wie es in Figure 2.1 gezeigt wird. Zudem wird 
das gerade trainierte Layer gefreezt, damit dieses keine weiteren Aktualisierungen mehr bekommt. 
Dann wiederholt sich das Training, das Löschen, das Freezing und weitere Hinzufügen von Layern. 
An einer beliebigen Stelle kann TF gemacht werden, indem, statt in der Trainingsphase den Sourcedatensatz zu nutzen, der Targetdatensatz 
genutzt wird. 

% Sollte ich nicht vorher Kaskadierung erklären? Oder geht das hier? Hier könnte ich auch Graphen bauen. Ist glaube ich sogar besser, wenn 
% ich es tue...

    \subsection{Direct Cascade}
    In diesem Abschnitt wird die Kaskadierungsvariante des Direct Cascade Netzwerks vorgestellt. Das Netzwerk ist hierbei vollständig vorab 
definiert und besteht aus einem einzelnen Hidden Layer sowie einem Output Layer. Die Gesamtstruktur setzt sich aus mehreren identischen 
Subnetzwerken zusammen, zwischen denen während des Trainings ein Wissenstransfer stattfindet.

\begin{figure}[htpb]
    \centering
    \includegraphics[height=10cm]{../../Graphiken/direct_cascade.png}
    \caption{\label{fig:directcascade} 
    \small{Hier wird das Direct Cascade Verfahren dargestellt. Dieses Verfahren verwendet mehrere einzelne Netzwerke 
    (hier als Modelle bezeichnet), die jeweils nur wenige Hidden Layer aufweisen, in der Regel lediglich einen. Nach der 
    Initialisierung wird jedes Modell einmal ohne weiteres Training angewendet, und dessen Ausgangssignal wird mit dem ursprünglichen 
    Eingabesignal kombiniert. Diese Kombination bildet den neuen Eingabedatensatz für das nachfolgende Modell. Durch diese sukzessive 
    Verknüpfung der Ausgaben mit den Eingaben kann das Verfahren eine Wissensweitergabe und -integration zwischen den einzelnen Modellen 
    realisieren.}}
\end{figure}

Der Ablauf beginnt, wie in Abbildung \ref{fig:directcascade} dargestellt, mit dem vorbereiteten Quell-Datensatz (Sourcedatensatz), der als 
Eingabe in die erste Instanz des Netzwerks gegeben wird. Diese Netzwerkinstanz wird anschließend trainiert. Nach Abschluss des Trainings 
erfolgt eine einmalige Anwendung des fixierten Netzwerks, deren Ergebnis die Vorhersage (Prediction) darstellt. Diese Prediction wird mit 
dem ursprünglichen Eingabesignal desselben Netzwerks kombiniert, wodurch ein sogenannter Augmented Vector entsteht. Die genaue Bildung dieses 
Augmented Vectors variiert dabei leicht je nach spezifischer Implementierung des jeweiligen Direct Cascade Netzwerks und wird an späterer 
Stelle detaillierter erläutert.

Der Augmented Vector dient als Input für die nächste Instanz des Netzwerks. Dieser Zyklus aus Netzwerkinstanz, Training, Prediction und 
Augmented Vector Berechnung wird beliebig oft wiederholt. Durch die Einbindung der Prediction in den Augmented Vector kann das Netzwerk 
Wissen aus den zuvor trainierten Instanzen übernehmen und integrieren.

Eine Transfer-Learning-Phase (TF) kann jederzeit innerhalb eines Trainingsschritts durchgeführt werden, indem anstelle des Quell-Datensatzes 
ein Ziel-Datensatz (Targetdatensatz) als Input verwendet wird. Dabei können beliebig viele Netzwerkinstanzen vor und nach der 
Transfer-Learning-Phase genutzt werden. Der einzige Unterschied besteht darin, dass der Augmented Vector mit jedem weiteren Netzwerk etwas 
größer wird, da er sowohl das Wissen aller bisher trainierten Netzwerke als auch die ursprünglichen Eingabedaten enthält.

Für die Implementierung bedeutet dies, dass von Beginn an sowohl der Quell- als auch der Ziel-Datensatz in das feste Netzwerk eingespeist 
werden müssen. Dies ist notwendig, um die Prediction auf dem Ziel-Datensatz – die während der Trainingsphase mit dem Quell-Datensatz generiert 
wurde – im Augmented Vector zu integrieren. Somit wird sichergestellt, dass die während des Trainings auf dem Quell-Datensatz gelernten 
Netzwerkkomponenten auch bei der Anpassung an den Ziel-Datensatz berücksichtigt werden.

    \section{Setup}
    Alle Test wurden auf einem Erazer Gaming Notebook P15601 unter Windows 10 durchgeführt.
Die Neuronalen Netze laufen dabei ausschließlich auf 
der CPU und wurden nur trainiert, während der Rechner am Stromnetz angeschlossen war. 
Dieser Rechner hat einen intel Core i5 der neunten Generation mit 4 Kernen auf 8 
logischen Prozessoren. Die Betriebsgeschwindigkeit liegt bei 2,4-5,1 GHz und die 
RAM-Größe liegt bei 15,8 GB bei einer Geschwindigkeit von 2667 MHz. 

Es wurde mit PyCharm und der library Keras programmiert. Die Texte sind mit BibTex 
erstellt worden und die Plots mit der MatPlotLib library.

Dabei sind MNIST und Bost die Sourcedaten und SVHN und Cali die Targetdaten. Jeder Targetdatensatz 
wird händisch verkleinert, da es darum geht, nicht genügend Daten für sie allein zu haben und deshalb eine andere 
Methode genutzt werden muss.

    % \section{Durchführungen}
    % Es wurden sowohl für Klassifikation als auch für Regression drei verschiedene Ansätze der Kaskadierung genutzt. 
Ebenso wurde Direct TF mit Domainwechsel durchgeführt. 

Die drei Ansätze sind Deep Cascade, Direct Cascade und eine Kaskadierung von einem Netz im Netz mit mehreren Inputs.

Bei Deep Cascade wird ein Netz Layer für Layer aufgebaut und jedes Layer einzelnd trainiert und gefreezt. 
Bei Direct Cascade werden ganze Netze trainiert und dessen Prediction als zusätzlichen Input für das nächste Netz zu nutzen.
Bei der dritten Variante wird ein Netz trainiert, dann auf einem Teilnetz davon die Prediction gemacht, um mit dieser das 
ganze Netz außer das vorher erwähnte Teilnetz zu trainieren. 
Nur Deep Cascade wird genauer betrachtet, denn die beiden anderen Ansätze sind nur zum Vergleichen da.

Es wurde bei jedem gleichbleibende Epochenanzahlen, zufällige und von einer Metrik abhängige durchgeführt. 

Bei allen Neuronalen Netzen wurden die dafür benötigten Daten in ein Trainings-, Validation- und Testdatensatz aufgeteilt. 
Ebenfalls wurde MNIST auf 32x32 erweitert, sowie SVHN in graue Bilder mit einem Channel verändert. 
Es wurde erweitert, da keine Daten unnötig verloren werden sollten. Die Reduzierung von SVHN liegt daran, dass MNIST nur 
Schwarz-Weiß-Bilder sind und es nicht möglich ist, dies in bunte Bilder zu verändern.
Die Veränderungen der Datensätze kommt daher, dass sie technisch gleich aussehen müssen, da sie sonst nicht als Input 
desselben Netzes genutzt werden können.

Für die Regressionsnetze müssen alle Spalten weggenommen werden, die kein Gegenüber im anderen Datensatz besitzen. 
Somit fielen die Spalten: Verbrechensrate, Anteil der Wohngebiete über 25000 Fuß, Nicht-Einzelhandelanteil der Gewerbeflächen, 
Flussgrundstück, Stickoxidkonzentration, Entfernung zu Arbeitsvermittlungszentren, Erreichbarkeit von Autobahnen, 
Vollwertsteuersatz, Schüler-Lehrer-Verhältnis und die Anzahl von Schwarzen im Bosten weg, während im California die 
folgenden Spalten wegfielen: Längengrad, Breitengrad, Schlafzimmer und Bevölkerung. Aus der Gesamtanzahl der Räume und der Haushalte 
wird die durchschnittliche Anzahl an Räumen pro Wohnung errechnet.
Diese Spalten haben alle keinen Gegenüber im anderen Datensatz und eine Spalte ist aus ethnischen Gründen nicht nutzbar, was daran liegt, 
dass der Datensatz aus den Siebzigern stammt.
Übrig blieben von Boston nur noch die durchschnittliche Anzahl der Räume pro Wohnung, die Menge der Häuser, die vor 1940 
erbaut worden sind und der prozentuale Anteil der Bevölkerung mit niedrigem Status.
Bei California blieben das Errechnete und das Hausalter, sowie das durchschnittliche Einkommen. 
Die Anzahl der Räume pro Wohnung passen offensichtlich zueinander, während der prozentuale Anteil der Bevölkerung mit niedrigem Status 
antiproportional zu dem durchschnittlichen Einkommen ist. Dies wird vorher zu einer Proportionalität umgewandelt.
Als etwas komplizierter erweist sich das Alter. Mit Prozentrechnung kann man aber das ungefäre Alter der Häuser aus dem 
Boston Datensatz abschätzen. Da immer eine Häuseranzahl von einhundert betrachtet wird, ist dies die Gesamtmenge und folgende Formel 
löst das Problem: 
\begin{equation}
    \frac{Hausanzahl * Hausalter}{Gesamtmenge}
\end{equation}
Die Hausanzahl ist hier die Menge der Häuser, die vor 1940 erbaut worden sind. Das Hausalter bezieht sich auf das Alter der eben 
erwähnten Häuser und ist auf Heute angedacht; sind also 85 Jahre.

Die Hypothese war, dass man mit TF bei zu wenig Daten eine verhältnismäßig gute Performanz der Netze erwarten kann, sowie, 
dass durch einen Kaskadierungsansatz das Training der Netze sehr kurz ist.

Generell wird zuerst eine Weile auf dem Sourcedatensatz trainiert und dann auf den Targetdatensatz gewechselt ohne die 
bisherigen Netze zu verändern. Bis auf den Direct Cascade Ansatz werden auch die Inputs während des ganzen Prozesses nicht 
verändert.
Bei Direct Cascade werden die Inputs immer größer, denn die Prediction des vorherigen Netzes wird zum Input hinzugefügt.

    \section{Metrik}
    Es wurden drei Metriken erstellt. 
Die Accuracy- (ACCM), Loss- (LM) und MAE-Metrik (MAEM). MAE heißt dabei Mean absolute Error. 
Alle drei Metriken sind für Early Stopping und entscheiden, wieviele Epochen genutzt werden. 
Die Accuracy-Metrik bricht immer dann ab, wenn die Validation-Accuracy mindestens um 10\% 
schlechter ist als die Trainingsaccuracy, da dann in dem Netzwerk Overfitting herrscht.

Die Loss- und die MAE-Metrik brechen beide dann ab, wenn der Validation-Wert der aktuellen 
Epoche schlechter ist als in der Epoche davor. Dies hat zur Folge, dass die Netze in lokale 
Minima hineinlaufen und nicht wieder herauskommen. Dabei unterliegt die Anzahl der Netze für 
das Direct Cascade keiner Metrik.

% Evtl eine Metrik bauen, die über einen Max-Value für ACC geht und dann den Abbruch macht, wenn es besser 
% wird oder schlechter als dieser Wert(Muss rel. enger Bereich sein)
% Ebenso für Min-Value für Loss und MAE

    \section{Liste der Tests}
    
Liste aller hier vorkommenden Netzen mit ihren Kürzeln:

\begin{enumerate}
    \item ConvMaxPool (CMP)
    \item 1DConv (1DC)
    \item 2DConv (2DC)
    \item ClassOneDense (COD)
    \item RegressionTwo (Regr2)
    \item OneLayer (1Lay)
\end{enumerate}

Davon sind ConvMaxPool und RegressionTwo Deep Cascade Netzwerke, während alle anderen Direct Cascade Netzwerke sind. 
Ebenso sind nur RegressionTwo und OneLayer Regressionsnetze, während der Rest Klassifikationsnetze sind. 

Alle Netze werden mit dem Adam-Optimizer mit der Lernrate 1e-3 gelernt. Klassifikationsnetze haben als Loss den 
CategoricalCrossEntropy und Softmax als Aktivierungsfunktion, während die Regressionsnetze MeanSquaredError und Linear als 
Aktivierungsfunktion vorweisen. 

Mit allen Direct Cascade Netzwerken wurden zusätzlich Early Stopping Metriken durchgeführt mit MAEM, LM und ACCM. 

Für alle Klassifikationsnetze gilt, dass sie mit fünf verschiedenen Größen des Targetdatensatzes trainiert wurden. Die Ausnahme ist das 
2DC-Netzwerk, welches nur mit sehr wenigen Source- und Targetdaten trainiert werden kann, da es technisch auf derselben Hardware mit mehr 
Daten unmöglich ist. 

Bei den Regressionsnetzen wird jeweils einmal mit vielen und wenigen Targetdaten trainiert. 

Es wurde mit allen Netzwerken ein Vergleich sowohl zwischen mit TF und ohne als auch zwischen ohne TF und Kompletten angefertigt. 
Komplett heißt hier, dass es ein Netzwerk ohne TF und ohne Kaskadierung ist und dieses deshalb in einem komplett trainiert wird. 

Mit allen Netzwerken wurde der Zeitpunkt für das TF frei ausgetestet. 

Alle Direct Cascade Netzwerke haben jeweils nur ein Hidden Layer. In manchen Fällen sind sie noch mit einem Hilfslayer, um den Wechsel 
zwischen Filterlayern und Linearlayern zu bewerkstelligen. 

Für alle Netzwerke wurde derselbe Seed für die Initialisierung der Weights genutzt. 

In Tabelle 2.1 sind die Tests bezüglich Klassifikation und in Tabelle 2.2 die für die Regression. 
In beiden Tabellen sind die Tests, die sich mit der Zeitdauer befassen mit der Endung Time. Dabei gilt, dass CasTF Kaskadierung mit TF, Cas allein 
Kaskadierung ohne TF und Comp bedeutet, dass es weder TF noch Kaskadierung gab. ACCM, LM und MAEM sind die Tests bezüglich der Early-Stopping 
Metriken. 
Vor dem ersten Schrägstrich steht, wann TF gemacht wurde, welches mit TF im Eintrag gekennzeichnet ist. Wenn kein TF gemacht wurde, ist dieser 
erste Bereich nicht existent. Dahinter steht die Datenmenge 
des Targetdatensatzes und danach die Menge an Epochen pro Trainingsiteration. Wenn es noch etwas viertes gibt, dann zeigt dieses an, wieviele 
Epochen in Zehnern es insgesamt gab. 

\begin{table}[!ht]
    \centering
    \begin{tabular}{l|l|l|l}
        \textbf{CMP} & \textbf{COD} & \textbf{1DC} & \textbf{2DC} \\
        \hline
        TF0/732/10 & CasTFTime & CasTFTime & CasTFTime \\
        TF1/732/10 & CasTime & CasTime & CasTime \\
        TF2/732/10 & CompTime & CompTime & CompTime \\
        TF3/732/10 & TF2/732/10 & TF2/732/10 & TF2/732/10 \\
        TF4/732/10 & TF2/7k/10 & TF2/7k/10 & \\
        TF5/732/10 & TF2/21k/10 & TF2/21k/10 & \\
        732/10 & TF2/36k/10 & TF2/36k/10 & \\
        CasTFTime & TF2/51k/10 & TF2/51k/10 & \\
        CasTime & TF10/732/10/30 & TF10/732/10/30 & \\
        CompTime & 732/10/30 & 732/10/30 & \\
        TF2/7k/10 & Comp/732//30 & Comp/732//30 & \\
        TF2/21k/10 & ACCM/732/10 & ACCM/732/10 & \\
        TF2/36k/10 & LM/732/10 & LM/732/10 & \\
        TF2/51k/10 & & & \\
        & & & \\
    \end{tabular}
    \caption{\label{tab:classtests} Liste aller Klassifikationstests}
\end{table}

\begin{table}[!ht]
    \centering
    \begin{tabular}{l|l}
        \textbf{Regr2} & \textbf{1Lay} \\
        \hline
        TF0/240/25 & CasTFTime \\
        TF1/240/25 & CasTime \\
        TF4/240/25 & CompTime \\
        CasTFTime & TF11/8k/10/8 \\
        CasTime & 8k/10/8 \\
        CompTime & Comp/8k/10/8 \\
        TF3/8k/10/8 & TF11/240/10/20 \\
        8k/10/8 & 240/10 \\
        Comp/8k/8 & Comp/240/8 \\
        TF3/240/8 & MAEM/240/10 \\
        240/8 & LM/240/10 \\
        Comp/240/8 & TF4/206/10/8/ts \\
        TF4/206/10/8/ts & 206/10/8/ts \\
        206/10/8/ts & Comp/206/10/8/ts \\
        Comp/206/10/8/ts &
    \end{tabular}
    \caption{\label{tab:regrtests} Liste aller Regressionstests}
\end{table}

Eine Referenz zu einer dieser Listen ist CMP:TF0/732/10. Diese bedeutet, dass es um den Test mit der Kennung TF0/732/10 des CMP-Netzwerkes geht. 
Bei diesem gibt es die Besonderheit, dass TF nach dem ersten Layer gemacht wird, dieses Layer jedoch im 
Gegensatz zum restlichen Netzwerk nur mit einer Epoche trainiert wird.
Selbiges gilt für Regr2:TF0/240/25.

Die Tests mit der Endung ts sind diejenigen, die einen explizit großen Testdatensatz haben. 


    \chapter{Allgemeines}  % 3.Allgemeines
    \section{ConvMaxPool}
    \subsection{Einbruch bei TF}
    \subsection{Wenig Epochen für Stabilisierung}
    \subsection{Overfitting auf Sourcedatensatz}
    \section{Zeitnahme}

    \chapter{Klassifikation}  % 4.Klassifikation
    \section{Größe des Targetdatensatzes}
    \section{Bilddimensionalität}
    \section{Augmentierung}
    \section{Mit und Ohne}
    \subsection{TF}
    \subsection{Kaskadierung}

    % \section{Filternetze}
    % Die Klassifikation über Filternetze im Direct Cascade Ansatz. 
Es wird zudem zwischen eindimensionalen und zweidimensionalen Bilddaten unterschieden, um zu zeigen, dass die Dimensionalität für die 
Accuracy irrelevant ist und es im zweidimensionalen nur länger dauert. 

Der zweidimensionale Fall hat folgende Updateregel: 
Die Prediction wird ausgelesen und dessen Wert wird in ein Array mit demselben Shape hineingeschrieben und dies wird dann mit den Trainingsdaten 
auf der Channelachse konkateniert. Dies ergibt den Augmented Vector, der als Input für das nächste Netz genutzt wird.

Die Besonderheit des zweidimensionalen Netzes ist es, dass es in der Updateregel sehr viel Arbeitsspeicher benötigt wird. Deshalb wird nicht nur 
der Targetdatensatz auf 1\% verkleinert, sondern auch der Sourcedatensatz. 

Es wurde einmal ohne Metrik, einmal mit der Accuracy-Metrik und einmal mit der Loss-Letrik ausgetestet.

\begin{figure}[htpb]
    \includegraphics[height=5cm]{../../Plots/DirClass_LilConv/Dir2DLilConvTrainTen2Ten.png}
    \includegraphics[height=5cm]{../../Plots/DirClass_LilConv/Dir2DLilConvTestTen2Ten.png}
    \includegraphics[height=5cm]{../../Plots/DirClass_LilConv/Ten2TenTrain_ACC.png}
    \includegraphics[height=5cm]{../../Plots/DirClass_LilConv/Ten2TenTest_ACC.png}
    \includegraphics[height=5cm]{../../Plots/DirClass_LilConv/Ten2TenTrain_Loss.png}
    \includegraphics[height=5cm]{../../Plots/DirClass_LilConv/Ten2TenTest_Loss.png}
    \caption{\label{fig:2dconv}}
\end{figure}

Die Figure 2.1 zeigt die Plots der Tests zuerst ohne eine Metrik, dann mit der ACCM und zum Schluss mit der LM. 
Da der Sourcedatensatz verringert wurde, ist die Accuracy im ersten Bereich geringer als es erwartbar ist. 
Das Early Stopping ist hier nicht erkennbar, da die Berechnungszeit des Augmented Vector hier für die Zeit entscheidend ist.

Im eindimensionalen Fall gibt es folgende Updateregel: 
Die Bilddaten werden zuerst in eindimensionale Bilder verändert mit Channels. Dies ist der Input des Netzes. Die Prediction und der Netzinput, 
dessen Channelachse vorübergehend entfernt wird, werden direkt konkateniert und das Ergebnis, um die Channelachse erweitert. 

Hier wird nur der Targetdatensatz verkleinert und einem ohne Metrik, mit der ACCM und der LM ausgetestet. Dies zeigt die Figure 2.2.

\begin{figure}[htpb]
    \includegraphics[height=5cm]{../../Plots/DirClass_OneDConv/Ten2TenTrain.png}
    \includegraphics[height=5cm]{../../Plots/DirClass_OneDConv/Ten2TenTest.png}
    \includegraphics[height=5cm]{../../Plots/DirClass_OneDConv/DataTrain_ACC_Metr.png}
    \includegraphics[height=5cm]{../../Plots/DirClass_OneDConv/DataTest_ACC_Metr.png}
    \includegraphics[height=5cm]{../../Plots/DirClass_OneDConv/Ten2TenTrain_Loss.png}
    \includegraphics[height=5cm]{../../Plots/DirClass_OneDConv/Ten2TenTest_Loss.png}
    \caption{\label{fig:1dconv}}
\end{figure}

Hier werden die Metriken eindeutig gesehen, denn die Trainingszeit ist sehr verschieden. Es wird auch klar, dass der eindimensionale Fall 
schlechter ist als der Zweidimensionale.
Dies liegt daran, dass im eindimensionalen nur die Daten rechts und links von dem betrachteten Pixel in die Berechnung mit eingezogen werden 
können, während im zweidimensionalen zusätzlich auch die Daten oben, unten und die vier Ecken jenes Kreuzes betrachtet werden.
% da der eindimensionale Fall nicht die Verhältnisse der Pixel innerhalb einer Achse auf beiden Achsen betrachten kann. 
Beide Netze haben eine sehr schlechte Accuracy. Dies liegt aber nicht daran, dass auf dem Sourcedatensatz Overfitting passiert ist, denn es 
wird nicht besser, wenn der Wechsel der Datensätze beliebig nach vorne geschoben wird.

    % \section{Linearnetze}
    % Die Klassifikation über Linearnetze mittels eines Direct Cascade Ansatzes. 

Das Netz hat ein Linearlayer mit 512 Nodes und der ReLU-Activation function.

Für den augmented Vector werden alle inputdaten in eindimensionale Bilder verwandelt. Der Input wird mit der Prediction konkateniert und 
direkt an das nächste Netz weitergegeben.

Dieses Netz wurde ohne Metrik, mit der ACCM und LM ausgetestet. 

\begin{figure}[htpb]
    \includegraphics[height=5cm]{../../Plots/MnistLongDense/DataTrain.png}
    \includegraphics[height=5cm]{../../Plots/MnistLongDense/DataTest.png}
    \includegraphics[height=5cm]{../../Plots/MnistLongDense/Ten2Ten_Train_ACC.png}
    \includegraphics[height=5cm]{../../Plots/MnistLongDense/Ten2Ten_Test_ACC.png}
    \includegraphics[height=5cm]{../../Plots/MnistLongDense/Ten2Ten_Train_Loss.png}
    \includegraphics[height=5cm]{../../Plots/MnistLongDense/Ten2Ten_Test_Loss.png}
    \caption{\label{fig:linclass}}
\end{figure}

Die Figure 2.3 zeigt, dass ein Netzwerk mit nur Linearlayers etwas schlechter ist als ein zweidimensionales Filternetz. 
Hier werden die Metriken auch bereits gesehen, aber sie bringen kein erhofftes Ergebnis. 

Die Klassifikation funktioniert mit den Updateregeln und diesen Netzen nicht. Auch TF bringt dabei nichts. Das einzige, 
was halbwegs etwas bringt, ist, wenn mehr Targetdaten benutzt werden. Aber dann wird TF auch nicht mehr gebraucht. 


    \chapter{Regression}  % 5.Regression
    \section{Datenaugmentation}
    \subsection{Wenig Daten}
    \subsection{Viele Daten}
    \section{Early Stopping}
    % Diskussion
    % Schreibe alles auf, was du gelernt hast: Das ist dann die Diskussion -> Wie kommt man dahin, was braucht man dafür?
    % Struktur -> Steil
    % Eine Liste aller Experimente, die relevant sind, die ich gemacht habe. -> Heiko (Und KFOLD-Cross bei wichtigen Durchführen)
    % Achsenbeschriftung verbessern (Einheit + Größe)
    
    \chapter{Diskussion}  % 6.Diskussion
    \section{Erkenntnisse}
    \textbf{Bei Klassifikation und Regression:}\\
Es zeigt sich ein relativ schneller Overfitting-Effekt auf dem Source-Datensatz.

An der Stelle des TFs bricht die Performanz des Netzes deutlich ein. Anschließend kommt es zwar zu einer teilweisen Erholung, 
jedoch erreicht die Leistung nicht mehr das vorherige Niveau.

In der vorliegenden Implementierung sind die herkömmlichen Netzwerke bei kleinen Datensätzen und flacher Netzstruktur schneller als die Direct 
Cascade Netzwerke, welche wiederum schneller trainieren als die Deep Cascade Architekturen. Die Performanz variiert jedoch signifikant 
zwischen den einzelnen Netzwerktypen. 

Bei zunehmender Netzwerktiefe und einer hohen Anzahl an Trainingsepochen kehrt sich das Zeitverhältnis zwischen den Kaskadenversionen des 
Netzwerks und dem vollständigen Netzwerk um, sodass Letzteres im Vergleich deutlich langsamer ist als beide Kaskadenvarianten.

\textbf{Klassifikation:}\\
Bei der Klassifikation ist der Accuracy-Wert primär von der Größe des Target-Datensatzes abhängig: Je geringer die Datenmenge, desto 
schlechter fällt die Klassifikationsgenauigkeit aus.

Die Klassifikationsleistung ist derart unzureichend, dass ein Einsatz in diesem Kontext nicht sinnvoll erscheint, insbesondere wenn die 
Accuracy nur bei etwa 20\% liegt.

Bei TF zeigt sich eine leicht schlechtere Performanz im Vergleich zum Training ohne TF.

Ein Cascade-Netzwerk erzielt eine schlechtere Leistung als ein vollständig trainiertes Netzwerk ohne Kaskadierung.

Die Verarbeitung mehrdimensionaler Augmented Vectors kann zu Problemen mit dem Arbeitsspeicher führen.

Die eindimensionale Klassifikation ist geringfügig schlechter als die zweidimensionale; dieser Unterschied ist jedoch minimal und kann 
vernachlässigt werden.

\textbf{Regression:}\\
Bei der Regression führt der Einsatz von TF bei wenigen Trainingsdaten zu besseren Ergebnissen als ein Training des Netzwerks ausschließlich 
auf dem Target-Datensatz von Grund auf — dies gilt jedoch nur für Direct Cascade Netzwerke.

Die Leistungsfähigkeit der Regressionsnetzwerke mit eingesetztem TF ist ausreichend, um eine praktische Anwendbarkeit zu gewährleisten.

Der Performanz-Abfall bei der Regression fällt insgesamt deutlich geringer aus als bei der Klassifikation.

Die hier verwendeten einfachen Early-Stopping-Metriken verschlechtern die Ergebnisse, da sie dazu neigen, in lokalen Minima stecken zu bleiben.

    \section{Ausblick}
    \section{Fazit}
    \printbibliography
\end{document}
