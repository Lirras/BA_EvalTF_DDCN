Die Klassifikation über Linearnetze mittels eines Direct Cascade Ansatzes. 

Das Netz hat ein Linearlayer mit 512 Nodes und der ReLU-Activation function.

Für den augmented Vector werden alle inputdaten in eindimensionale Bilder verwandelt. Der Input wird mit der Prediction konkateniert und 
direkt an das nächste Netz weitergegeben.

Dieses Netz wurde ohne Metrik, mit der ACCM und LM ausgetestet. 

\begin{figure}[htpb]
    \includegraphics[height=5cm]{../../Plots/MnistLongDense/DataTrain.png}
    \includegraphics[height=5cm]{../../Plots/MnistLongDense/DataTest.png}
    \includegraphics[height=5cm]{../../Plots/MnistLongDense/Ten2Ten_Train_ACC.png}
    \includegraphics[height=5cm]{../../Plots/MnistLongDense/Ten2Ten_Test_ACC.png}
    \includegraphics[height=5cm]{../../Plots/MnistLongDense/Ten2Ten_Train_Loss.png}
    \includegraphics[height=5cm]{../../Plots/MnistLongDense/Ten2Ten_Test_Loss.png}
    \caption{\label{fig:linclass}}
\end{figure}

Die Figure 2.3 zeigt, dass ein Netzwerk mit nur Linearlayers etwas schlechter ist als ein zweidimensionales Filternetz. 
Hier werden die Metriken auch bereits gesehen, aber sie bringen kein erhofftes Ergebnis. 

Die Klassifikation funktioniert mit den Updateregeln und diesen Netzen nicht. Auch TF bringt dabei nichts. Das einzige, 
was halbwegs etwas bringt, ist, wenn mehr Targetdaten benutzt werden. Aber dann wird TF auch nicht mehr gebraucht. 
