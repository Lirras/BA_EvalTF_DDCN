\textbf{Bei Klassifikation und Regression:}\\
Es zeigt sich ein relativ schneller Overfitting-Effekt auf dem Source-Datensatz.

An der Stelle des TFs bricht die Performanz des Netzes deutlich ein. Anschließend kommt es zwar zu einer teilweisen Erholung, 
jedoch erreicht die Leistung nicht mehr das vorherige Niveau.

Die Deep Cascade Architektur zeigt durchgängig eine mindestens vergleichbare, in vielen Fällen jedoch leicht überlegene Leistungsfähigkeit im 
Vergleich zur Direct Cascade Variante.

In der vorliegenden Implementierung sind die herkömmlichen Netzwerke bei kleinen Datensätzen und flacher Netzstruktur schneller als die Direct 
Cascade Netzwerke, welche wiederum schneller trainieren als die Deep Cascade Architekturen. Die Performanz variiert jedoch signifikant 
zwischen den einzelnen Netz-werktypen. 

Bei zunehmender Netzwerktiefe und einer hohen Anzahl an Trainingsepo-chen kehrt sich das Zeitverhältnis zwischen den Kaskadenversionen des 
Netz-werks und dem vollständigen Netzwerk um, sodass Letzteres im Vergleich deutlich langsamer ist als beide Kaskadenvarianten.

\textbf{Klassifikation:}\\
Bei der Klassifikation ist der Accuracy-Wert primär von der Größe des Target-Datensatzes abhängig: Je geringer die Datenmenge, desto 
schlechter fällt die Klassifikationsgenauigkeit aus.

Die Klassifikationsleistung ist derart unzureichend, dass ein Einsatz in die-sem Kontext nicht sinnvoll erscheint, insbesondere wenn die 
Accuracy nur bei etwa 20\% liegt.

Bei TF zeigt sich eine leicht schlechtere Performanz im Vergleich zum Trai-ning ohne TF.

Ein Cascade-Netzwerk erzielt eine deutlich schlechtere Leistung als ein voll-ständig trainiertes Netzwerk ohne Kaskadierung.

Die Verarbeitung mehrdimensionaler Augmented Vectors kann zu Problemen mit dem Arbeitsspeicher führen.

Die eindimensionale Klassifikation ist geringfügig schlechter als die zweidimensionale; dieser Unterschied ist jedoch minimal und kann 
vernachlässigt werden.

\textbf{Regression:}\\
Bei der Regression führt der Einsatz von TF bei wenigen Trainingsdaten zu besseren Ergebnissen als ein Training des Netzwerks ausschließlich 
auf dem Target-Datensatz von Grund auf — dies gilt jedoch nur für Direct Cascade Netzwerke.

Die Leistungsfähigkeit der Regressionsnetzwerke mit eingesetztem TF ist ausreichend, um eine praktische Anwendbarkeit zu gewährleisten.

Der Performanz-Abfall bei der Regression fällt insgesamt deutlich geringer aus als bei der Klassifikation.

Die hier verwendeten einfachen Early-Stopping-Metriken verschlechtern die Ergebnisse, da sie dazu neigen, in lokalen Minima stecken zu bleiben.
