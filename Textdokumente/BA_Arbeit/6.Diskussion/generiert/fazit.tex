Die Klassifikation ist in Kombination mit Direct Cascade und TF unter Verwendung der hier eingesetzten Augmented Vektoren nicht praktikabel. 
Deep Cascade Architekturen erzielen zwar eine bessere Klassifikationsleistung, sind jedoch ebenfalls nicht sinnvoll einsetzbar.

Für die Regression erweist sich der Einsatz von TF in der Direct Cascade Variante als zielführend, da die Performanz entweder gleichwertig oder 
überlegen im Vergleich zu vollständig trainierten Netzwerken ist. Einzig bei sehr kleinen Datenmengen kann es zu Leistungseinbußen gegenüber dem 
vollständigen Netzwerk kommen.

Das Verfahren funktioniert auch mit Deep Cascade Netzwerken und zeigt selbst in diesem Ausnahmefall eine mit dem vollständigen Netzwerk 
vergleichbaren Performanz.

Vollständig trainierte Netzwerke mit der gleichen Gesamtanzahl an Hidden Layer sind insbesondere bei sehr kleinen Datensätzen schneller als Deep 
Cascade Modelle, welche wiederum schneller als Direct Cascade Architekturen sind. Aus diesem Grund ist es bereits bei geringen Datenmengen 
sinnvoller, ein vollständiges Netzwerk anstelle einer kaskadierten Architektur zu verwenden.
