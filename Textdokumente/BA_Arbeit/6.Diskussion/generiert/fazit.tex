Die Klassifikation ist in Kombination mit Direct Cascade und TF unter Verwendung der hier eingesetzten Augmented Vectors nicht praktikabel. 
Deep Cascade Architekturen erzielen zwar eine bessere Klassifikationsleistung, sind jedoch ebenfalls nicht sinnvoll einsetzbar.

Im Regressionskontext ist der Einsatz von Direct Cascade-Netzwerken mit aktiviertem TF grundsätzlich möglich, da die resultierende Performanz 
lediglich geringfügig unter der vollständiger Netzwerke liegt und vergleichbar mit derjenigen von Deep Cascade-Netzwerken unter Verwendung von 
TF ist.

Das Verfahren funktioniert auch mit Deep Cascade Netzwerken und zeigt selbst in diesem Ausnahmefall eine mit dem vollständigen Netzwerk 
vergleichbaren Performanz.

Vollständig trainierte Netzwerke mit der gleichen Gesamtanzahl an Hidden Layer sind insbesondere bei sehr kleinen Datensätzen und flacher 
Netzstruktur schneller als Deep Cascade Modelle, welche wiederum schneller als Direct Cascade Architekturen sind. 

Bei tiefen Netzwerkarchitekturen und einer großen Anzahl an Trainings-epochen verändert sich das Bild dahingehend, dass vollständige Netzwerke 
deutlich langsamer sind als sämtliche Kaskadierungsvarianten. Dieser Effekt tritt unabhängig davon auf, ob TF verwendet wird oder nicht.
