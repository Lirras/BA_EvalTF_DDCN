Alle Test wurden auf einem Erazer Gaming Notebook P15601 unter Windows 10 durchgeführt.
Die Neuronalen Netze laufen dabei ausschließlich auf 
der CPU und wurden nur trainiert, wenn der Rechner am Stromnetz angeschlossen war. 
Dieser Rechner hat einen intel Core i5 der neunten Generation mit 4 Kernen auf 8 
logischen Prozessoren. Die Betriebsgeschwindigkeit liegt bei 2,4-5,1 GHz und die 
RAM-Größe liegt bei 15,8 GB bei einer Geschwindigkeit von 2667 MHz. 

Es wurde mit PyCharm und der library Keras programmiert. Die Texte sind mit BibTex 
erstellt worden und die Plots mit der MatPlotLib library.

Die genutzten Datensätze sind Modified National Institute of Standards and Technology Dataset (MNIST), 
Streetview House Numbers (SVHN), Boston Housing Prices (Boston) und California Hausing Prices (California).

Dabei sind MNIST und Boston die Sourcedaten und SVHN und California die Targetdaten. Jeder Targetdatensatz 
wird händisch verkleinert, da es darum geht, nicht genügend Daten für sie allein zu haben und eine andere 
Methode genutzt werden muss.
