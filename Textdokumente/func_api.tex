Es wird hier ein Versuch gemacht, Direct Cascade mit TF 
durchzuführen. Jeweils einmal mit Klassifikation und Regression, 
sowie mit und ohne TF. Zudem wird überprüft wie lange die 
Gesamtzeit ist. Alle Netze werden zweimal mit je Vier Epochen 
trainiert. 

\section{Klassifikation}
Der Optimizer ist hier Adam und die learningrate schrumpft pro 
Epoche nach folgender Regel:
\begin{equation}
    (1e-4) * 10^{\frac{10}{epoch + 10}}
\end{equation}
Das Netz liegt in folgender Form vor: 
\begin{figure}[htpb]
    \includegraphics[height=15cm]{../Plots/functional_api_tests/Classification_Network.png}
    \caption{\label{fig:figure16} Aufbau eines Klassifikationsnetzwerk für Model as Layer learning}
\end{figure}
Hier wird zuerst von MNIST mit dem Model 1 ein predict gemacht. Das Ergebnis davon 
ist der zweite Input des Gesamtnetzes. Dieses wird danach komplett 
auf MNIST trainiert. Sobald es fertigt ist, wird Model 1 gefreezt und 
ein predict darauf mit den SVHN-Daten durchgeführt. 
Dieses zweite predict ist wieder der zweite Input, sodass dann 
das gesamte Netz nun auf SVHN trainiert wird. Mit TF dauert das 
$515.33$ Sekunden. Wenn direkt auf SVHN trainiert wird, dauert es etwas 
länger mit $578.66$ Sekunden, hat aber eine um etwa 10\% höhere Accuracy.
\begin{figure}[htpb]
    \includegraphics[height=5cm]{../Plots/functional_api_tests/TFClass_515_33sec.png}
    \includegraphics[height=5cm]{../Plots/functional_api_tests/Class_578_66sec.png}
    \caption{\label{fig:figure17} Classification mit und ohne TF}
\end{figure}


\section{Regression}
Der Optimizer ist hier RMSprop und die learningrate schrumpft pro Epoche nach 
folgender Regel: 
\begin{equation}
    (1e-4) * 100^{\frac{10}{epoch + 10}}
\end{equation}
Das Netz liegt in folgender Form vor:
\begin{figure}[htpb]
    \includegraphics[height=10cm]{../Plots/functional_api_tests/Regression_Network.png}
    \caption{\label{fig:figure18} Aufbau Regression Network}
\end{figure}
Hier wird exakt das Gleiche wie in der Klassifikation oben gemacht nur auf dem 
Boston- und dem California-Datensatz. Die Dauer mit TF ist $23.88$ Sekunden, während 
es ohne $39.81$ Sekunden ist. Es gibt kaum einen Unterschied zwischen den beiden 
Werten im Mean Absolute Error.
\begin{figure}
    \includegraphics[height=5cm]{../Plots/functional_api_tests/TFRegr_23_88sec.png}
    \includegraphics[height=5cm]{../Plots/functional_api_tests/Regr_39_81sec.png}
    \caption{\label{fig:figure19} Regression mit und ohne TF}
\end{figure}