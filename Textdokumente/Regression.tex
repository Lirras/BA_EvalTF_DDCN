Bei einer Regression kommt es dazu, dass das NN sich mithilfe 
der Daten der korrekten Funktion annähert.

\section{Datensätze}
    Die genutzten Datensätze sollen zunächst Boston Housing und California Housing sein, da 
    beide dasselbe Ziel mit der Annäherung an den Preis des Hauses haben. 
    Boston Housing hat ein ethnisches Problem, da es eine Spalte gibt, die den Preis in 
    Abhängigkeit zur Anzahl der in der Nähe lebenden Schwarzen setzt. Diese wird hier entfernt. 
    Ebenso entfernt werden die Spalten, die kein vernünftiges Gegenüber im California Datensatz 
    haben. Dadurch reduziert sich die Spaltenanzahl auf 3. Bei Boston bleiben RM, AGE und LSTAT 
    über, die im Vergleich halbwegs zu den Spalten AveRoom, HouseAge und MedInc vom California 
    Datensatz passen. 
    RM ist dabei die durchschnittliche Anzahl der Räume pro Wohnung, während AveRoom die 
    durchschnittliche Anzahl pro Haushalt ist. 
    AGE ist dabei der Anteil der Wohnungen, die vor 1940 erbaut worden sind. HouseAge ist 
    hingegen das Durchschnittsalter der Häuser eines Wohnblocks. Diese Spalten müssen 
    aneinander angepasst werden.
    LSTAT ist der prozentuale Anteil der niedrigeren Bevölkerung und MedInc ist das mittlere 
    Einkommen des Wohnblocks. Hier müssen ebenfalls leichte Anpassungen durchgeführt werden. 
    Alle anderen Spalten beider Datensätze sind nicht ähnlich genug, dass zwischen ihnen 
    gelernt werden sollte. Die Erwartung ist, dass alle TF-Netze schlechter abschneiden als die 
    normalen schon allein deshalb, weil durch die Spaltenreduktion sehr viele Daten nicht 
    betrachtet werden. Des Weiteren ist genau das bei der Klassifikation ebenfalls passiert.